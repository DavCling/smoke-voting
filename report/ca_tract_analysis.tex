\documentclass[12pt]{article}
\usepackage[margin=1in]{geometry}
\usepackage{amsmath,amssymb}
\usepackage{booktabs}
\usepackage{graphicx}
\usepackage{natbib}
\usepackage{hyperref}
\usepackage{caption}
\usepackage{subcaption}
\usepackage{setspace}
\usepackage{float}
\usepackage{tabularx}
\usepackage{multirow}
\usepackage{threeparttable}
\usepackage{pdflscape}

\onehalfspacing

\graphicspath{{../output/california/figures/}}

\title{Wildfire Smoke and Voting Behavior: \\
  Evidence from California Census Tracts}
\author{Working Draft}
\date{\today}

\begin{document}
\maketitle

\begin{abstract}
We extend the national county-level analysis of wildfire smoke effects on voting behavior to census-tract resolution within California. Using Stanford ECHO Lab daily tract-level smoke PM$_{2.5}$ predictions and precinct-level election returns allocated to ${\sim}$8,000 census tracts via SWDB precinct-to-block crosswalks, we estimate two-way fixed effects panel regressions that exploit fine-grained within-county variation in smoke exposure. Smoke exposure in the 30 days before an election \emph{decreases} Democratic vote share ($\beta = -0.0012$, $p < 0.001$), \emph{increases} incumbent vote share ($\beta = 0.0197$, $p < 0.001$), and \emph{increases} turnout ($\beta = 0.0049$, $p < 0.001$). The DEM and turnout effects survive county$\times$year fixed effects---the most demanding specification---while the incumbent effect does not. Threshold analysis reveals a dose-response pattern: the DEM shift toward Republicans grows monotonically with smoke intensity. Compared to national county-level estimates, the CA tract-level incumbent and turnout effects are an order of magnitude larger, suggesting that county-level averaging attenuates localized smoke impacts.
\end{abstract}

\newpage
\tableofcontents
\newpage

% ============================================================
\section{Introduction}
% ============================================================

The national analysis establishes relationships between pre-election wildfire smoke exposure and voting outcomes at the county level using a two-way fixed effects (TWFE) design. This companion analysis pursues three goals:

\begin{enumerate}
  \item \textbf{Finer spatial resolution.} Census tracts (${\sim}$4,000 residents) offer much finer variation in both smoke exposure and voting outcomes than counties (${\sim}$100,000 residents). Within-county variation in smoke exposure---driven by topography, wind patterns, and proximity to fire sources---provides identifying variation that county-level data cannot capture.

  \item \textbf{Within-state identification.} Restricting to a single state eliminates state-level confounders entirely: state policy, campaign spending, and voter registration rules are constant across all tracts. County trends and county$\times$year fixed effects serve as demanding robustness checks, analogous to state trends and state$\times$year FE in the national analysis.

  \item \textbf{Direct comparison.} Matching the national analysis structure---build-up table, temporal dynamics, threshold comparisons, House elections---enables direct coefficient comparison and assessment of scale-dependence in the estimated effects.
\end{enumerate}

California is an ideal setting: it experiences the most wildfire smoke in the continental US, has substantial within-state variation in exposure (from the heavily smoked northern and central regions to the relatively clear coastal and southern tracts), and maintains high-quality precinct-level election data through the Statewide Database (SWDB).


% ============================================================
\section{Data}
% ============================================================

\subsection{Smoke Exposure}

We use the Stanford ECHO Lab v2 daily tract-level smoke PM$_{2.5}$ predictions \citep{childs2022daily}, which provide estimates of wildfire-attributable PM$_{2.5}$ for every US census tract from 2006 through 2023. The source data records only smoke days (days with positive wildfire-attributable PM$_{2.5}$); non-smoke days are absent and treated as zero exposure. We filter to California tracts (GEOID prefix ``06''), yielding 8,057 tracts and approximately 4 million tract-day observations.

For each election, we compute smoke exposure in cumulative pre-election windows (7, 14, 21, 28, 30, 60, 90 days, and the full fire season from June~1 through Election Day). Within each window we calculate:
\begin{itemize}
  \item Mean smoke PM$_{2.5}$ ($\mu$g/m$^3$; primary treatment variable)
  \item Number of smoke days (any positive smoke PM$_{2.5}$)
  \item Fraction of days exceeding the haze visibility threshold ($>$20~$\mu$g/m$^3$)
  \item Fraction exceeding ``Unhealthy for Sensitive Groups'' ($>$35.5~$\mu$g/m$^3$)
  \item Fraction exceeding ``Unhealthy'' ($>$55.5~$\mu$g/m$^3$)
\end{itemize}

Table~\ref{tab:summary} reports summary statistics. Mean 30-day smoke PM$_{2.5}$ across the presidential panel is 1.75~$\mu$g/m$^3$ (SD~=~3.37), with substantial right-skew: the maximum reaches 150.6~$\mu$g/m$^3$. Approximately 19\% of tract-election observations have at least one haze-threshold day in the 30-day window, while only 3.9\% and 1.5\% exceed the USG and unhealthy thresholds, respectively.

\begin{table}[H]
\centering
\caption{Summary Statistics (Presidential Panel, 30-Day Window)}
\label{tab:summary}
\begin{threeparttable}
\small
\begin{tabular}{lrrrrr}
\toprule
Variable & $N$ & Mean & SD & Min & Max \\
\midrule
\multicolumn{6}{l}{\emph{Smoke Exposure}} \\
Mean PM$_{2.5}$ ($\mu$g/m$^3$) & 32,061 & 1.749 & 3.371 & 0.000 & 150.581 \\
Smoke days & 32,061 & 5.532 & 8.145 & 0 & 31 \\
Frac.\ haze days ($>$20) & 32,061 & 0.020 & 0.051 & 0.000 & 1.000 \\
Frac.\ USG days ($>$35.5) & 32,061 & 0.003 & 0.017 & 0.000 & 0.833 \\
Frac.\ unhealthy days ($>$55.5) & 32,061 & 0.001 & 0.008 & 0.000 & 0.733 \\
[4pt]
\multicolumn{6}{l}{\emph{Outcomes}} \\
DEM two-party vote share & 32,061 & 0.662 & 0.174 & 0.000 & 1.000 \\
Incumbent party vote share & 32,061 & 0.506 & 0.238 & 0.000 & 1.000 \\
Log total votes & 32,061 & 7.347 & 0.704 & 0.473 & 9.713 \\
[4pt]
\multicolumn{6}{l}{\emph{Controls}} \\
Unemployment rate & 22,795 & 0.090 & 0.052 & 0.000 & 1.000 \\
Log median income & 22,728 & 11.085 & 0.485 & 7.824 & 12.429 \\
Log population & 22,826 & 8.356 & 0.494 & 0.693 & 10.612 \\
October mean temperature ($^\circ$C) & 32,058 & 19.842 & 2.461 & 6.194 & 28.088 \\
October precipitation (mm) & 32,058 & 16.743 & 37.013 & 0.000 & 897.613 \\
\bottomrule
\end{tabular}
\begin{tablenotes}
\small
\item \textit{Notes:} Presidential panel includes 8,030 tracts $\times$ 4 elections (2008, 2012, 2016, 2020). ACS controls available for 2012--2020 vintages (22,727 obs); weather controls available for all years. DEM vote share = DEM / (DEM + REP).
\end{tablenotes}
\end{threeparttable}
\end{table}


\subsection{Election Data}

Precinct-level election returns come from the California Statewide Database (SWDB) for general elections 2006--2022. We include:
\begin{itemize}
  \item Presidential races: 2008, 2012, 2016, 2020
  \item US House races: 2006, 2008, 2012--2022 (biennial; 2010 excluded due to data limitations)
\end{itemize}

California's top-two primary system (effective 2012) means some House races feature two candidates of the same party. We flag these same-party races and exclude them from Democratic vote share analysis while retaining them for turnout.

\subsection{Precinct-to-Tract Crosswalk}

Precinct votes are allocated to census tracts using SWDB precinct-to-block mapping files (\texttt{sr\_blk\_map}), which provide the share of each precinct's registered voters residing in each census block (\texttt{PCTSRPREC}). Since census blocks nest within tracts, we aggregate block-level shares to the tract level and apply these weights to allocate precinct vote counts proportionally.

We validate the crosswalk against the independently constructed tract-level vote allocations from \citet{fekrazad2025}, which covers 2016 and 2020. Our allocated tract-level Democratic votes correlate at $r = 0.98$ (2016) and $r = 0.97$ (2020) with the Fekrazad totals, and statewide vote totals agree within 0.14\%.

\subsection{Controls}

Tract-level time-varying controls from the American Community Survey (ACS) 5-year estimates (available 2012+):
\begin{itemize}
  \item Unemployment rate, log median household income, log total population
\end{itemize}

Weather controls from PRISM 4km monthly rasters (available all years):
\begin{itemize}
  \item October mean temperature, October total precipitation
\end{itemize}

ACS controls are available for 71\% of observations (2012--2020 elections). Specifications with controls use this subsample; specifications without controls use the full panel.


% ============================================================
\section{Empirical Strategy}
% ============================================================

Our baseline specification is a TWFE panel regression:
\begin{equation}
  Y_{it} = \beta \cdot \text{Smoke}_{it} + \mathbf{X}_{it}'\gamma + \alpha_i + \delta_t + \varepsilon_{it}
\end{equation}
where $Y_{it}$ is the outcome for tract $i$ in election year $t$, $\text{Smoke}_{it}$ is the 30-day pre-election smoke measure, $\mathbf{X}_{it}$ is a vector of time-varying controls, $\alpha_i$ are tract fixed effects, and $\delta_t$ are year fixed effects. Standard errors are clustered by tract.

\textbf{Outcomes:} (1) Democratic two-party vote share, (2) incumbent party vote share, (3) log total votes (turnout).

\textbf{Build-up specifications:}
\begin{enumerate}
  \item Raw OLS (no fixed effects)
  \item Tract + Year FE (TWFE baseline)
  \item TWFE + Controls (main specification)
  \item TWFE + Controls + County linear trends
\end{enumerate}

With tract and year FE, identification comes from within-tract, across-election variation in smoke exposure, isolating the effect of differential pre-election smoke from permanent tract characteristics and common time shocks.


% ============================================================
\section{Results}
% ============================================================

\subsection{Build-Up Table}

Table~\ref{tab:buildup} presents presidential election results (30-day window) across four specifications.

\begin{table}[H]
\centering
\caption{Build-Up Specifications: Presidential Elections (30-Day Window)}
\label{tab:buildup}
\begin{threeparttable}
\small
\begin{tabular}{lcccc}
\toprule
 & (1) & (2) & (3) & (4) \\
 & Raw OLS & TWFE & +Controls & +Cty.\ Trends \\
\midrule
\multicolumn{5}{l}{\emph{Panel A: DEM Vote Share}} \\[2pt]
Smoke PM$_{2.5}$ (30d) & $-$0.00177$^{***}$ & $-$0.00117$^{***}$ & $-$0.00121$^{***}$ & 0.00052 \\
 & (0.00020) & (0.00038) & (0.00031) & (0.00032) \\
$N$ & 32,061 & 32,061 & 22,727 & 22,727 \\[6pt]
\multicolumn{5}{l}{\emph{Panel B: Incumbent Vote Share}} \\[2pt]
Smoke PM$_{2.5}$ (30d) & $-$0.02288$^{***}$ & 0.01054$^{***}$ & 0.01973$^{***}$ & 0.01503$^{***}$ \\
 & (0.00142) & (0.00253) & (0.00173) & (0.00206) \\
$N$ & 32,061 & 32,061 & 22,727 & 22,727 \\[6pt]
\multicolumn{5}{l}{\emph{Panel C: Log Total Votes}} \\[2pt]
Smoke PM$_{2.5}$ (30d) & 0.02312$^{***}$ & 0.00403$^{***}$ & 0.00489$^{***}$ & 0.00620$^{***}$ \\
 & (0.00127) & (0.00107) & (0.00090) & (0.00133) \\
$N$ & 32,061 & 32,061 & 22,727 & 22,727 \\
\midrule
Tract FE & & Yes & Yes & Yes \\
Year FE & & Yes & Yes & Yes \\
Controls & & & Yes & Yes \\
County trends & & & & Yes \\
\bottomrule
\end{tabular}
\begin{tablenotes}
\small
\item \textit{Notes:} Standard errors clustered by tract in parentheses. $^{*}p<0.10$, $^{**}p<0.05$, $^{***}p<0.01$. Treatment is mean smoke PM$_{2.5}$ in the 30 days before Election Day. Controls: unemployment rate, log median income, log population, October temperature, October precipitation.
\end{tablenotes}
\end{threeparttable}
\end{table}

Several patterns emerge. First, the raw OLS estimates (column~1) conflate smoke exposure with permanent tract characteristics---smokier tracts tend to be more rural and Republican, producing a large negative correlation with DEM vote share. The TWFE specification (column~2) absorbs these permanent confounders and reduces the DEM coefficient by a third while flipping the sign on incumbent vote share from negative to positive.

Second, controls (column~3) sharpen the estimates. The DEM effect is stable at $-0.0012$, meaning a 1~$\mu$g/m$^3$ increase in 30-day mean smoke PM$_{2.5}$ is associated with a 0.12 percentage point decrease in Democratic vote share. The incumbent effect strengthens substantially from 0.011 to 0.020, and the turnout effect increases from 0.004 to 0.005.

Third, county linear trends (column~4) eliminate the DEM effect but the incumbent and turnout effects survive. This is consistent with slow-moving county-level partisan trends that correlate with smoke exposure patterns.

\subsection{Temporal Dynamics}

Figure~\ref{fig:temporal_mean} shows how estimated effects evolve as the smoke exposure window expands from 7 to 91 days before the election.

\begin{figure}[H]
\centering
\includegraphics[width=\textwidth]{ca_temporal_7day_mean_controls.png}
\caption{Temporal dynamics of smoke effects on voting outcomes (CA tract-level). Each point shows the coefficient from a separate TWFE+controls regression using the cumulative smoke window indicated on the x-axis. Shaded bands show 95\% confidence intervals. Treatment: mean smoke PM$_{2.5}$.}
\label{fig:temporal_mean}
\end{figure}

The DEM vote share effect peaks in the 14--21 day window ($\beta \approx -0.002$) and attenuates toward zero at longer windows, suggesting that recent smoke exposure matters most for partisan shifts. The incumbent vote share effect builds steadily through 49 days before stabilizing around $\beta \approx 0.01$--$0.02$, indicating that cumulative smoke exposure reinforces status-quo voting. The turnout effect is concentrated in the 14--28 day window before turning negative at longer horizons.

Figure~\ref{fig:temporal_frac} shows the same analysis using the fraction of days exceeding the haze visibility threshold ($>$20~$\mu$g/m$^3$) as the treatment variable.

\begin{figure}[H]
\centering
\includegraphics[width=\textwidth]{ca_temporal_7day_frac_controls.png}
\caption{Temporal dynamics using fraction of haze-threshold days ($>$20~$\mu$g/m$^3$). Same specification as Figure~\ref{fig:temporal_mean} but with a binary threshold treatment. The DEM effect is concentrated in the first 28 days; the incumbent effect is large and persistent.}
\label{fig:temporal_frac}
\end{figure}


\subsection{Smoke Exposure Maps}

Figure~\ref{fig:map_byyear} displays the geographic distribution of 30-day pre-election mean smoke PM$_{2.5}$ across California tracts for each presidential election year.

\begin{figure}[H]
\centering
\includegraphics[width=\textwidth]{ca_smoke_map_by_year.png}
\caption{Tract-level mean smoke PM$_{2.5}$ in the 30 days before each presidential election (2008, 2012, 2016, 2020). The 2020 election saw the most intense and widespread smoke exposure due to the August Complex and Creek fires. Northern California and the Central Valley are consistently the most exposed regions.}
\label{fig:map_byyear}
\end{figure}

\begin{figure}[H]
\centering
\includegraphics[width=0.85\textwidth]{ca_smoke_map_residualized.png}
\caption{Residualized smoke exposure after removing tract and year means (within-tract, within-year variation). This represents the identifying variation in the TWFE regressions. The 2020 fire season contributes the most residualized variation.}
\label{fig:map_resid}
\end{figure}

Figure~\ref{fig:map_norcal} zooms into Northern California, where the bulk of identifying variation originates.

\begin{figure}[H]
\centering
\includegraphics[width=0.85\textwidth]{ca_smoke_map_norcal.png}
\caption{Northern California regional detail of tract-level smoke exposure by election year. The 2020 fire season dominates, with extreme tract-to-tract variation driven by topography and fire proximity.}
\label{fig:map_norcal}
\end{figure}


% ============================================================
\section{Comparison with National County-Level Results}
% ============================================================

Table~\ref{tab:national} presents Specification~3 (TWFE + controls, 30-day window) estimates from both the national county-level and CA tract-level analyses. Figure~\ref{fig:national_comp} visualizes the comparison.

\begin{table}[H]
\centering
\caption{National County-Level vs.\ CA Tract-Level Estimates (Spec.\ 3, 30d)}
\label{tab:national}
\begin{threeparttable}
\small
\begin{tabular}{lcc}
\toprule
Outcome & National (County) & CA (Tract) \\
\midrule
DEM vote share & 0.00119$^{***}$ & $-$0.00121$^{***}$ \\
 & (0.00022) & (0.00031) \\[4pt]
Incumbent vote share & 0.00073 & 0.01973$^{***}$ \\
 & (0.00094) & (0.00173) \\[4pt]
Log total votes & 0.00228$^{***}$ & 0.00489$^{***}$ \\
 & (0.00044) & (0.00090) \\
\midrule
Unit FE & County & Tract \\
Year FE & Yes & Yes \\
Controls & Yes & Yes \\
Observations & 12,429 & 22,727 \\
\bottomrule
\end{tabular}
\begin{tablenotes}
\small
\item \textit{Notes:} Both specifications use 30-day mean smoke PM$_{2.5}$, TWFE with controls. National clusters by county; CA clusters by tract.
\end{tablenotes}
\end{threeparttable}
\end{table}

\begin{figure}[H]
\centering
\includegraphics[width=0.75\textwidth]{ca_national_comparison.png}
\caption{Coefficient comparison: national county-level vs.\ CA tract-level estimates. Points show $\hat{\beta}$ with 95\% confidence intervals. The incumbent and turnout effects are an order of magnitude larger at the tract level, consistent with county-level averaging diluting localized smoke impacts.}
\label{fig:national_comp}
\end{figure}

Three findings stand out. First, the DEM vote share effects have \emph{opposite signs}: smoke increases DEM share nationally but decreases it in California. This likely reflects California's unusual political geography---the smokiest tracts are in the rural, Republican-leaning Central Valley and Sierra foothills, where smoke may reinforce anti-environmentalist sentiment rather than increasing demand for climate action.

Second, the incumbent vote share effect is an order of magnitude larger at the tract level (0.020 vs.\ 0.001), and is highly significant at the tract level where it was null nationally. This suggests that county-level averaging severely dilutes the localized rally-around-the-flag or status-quo-bias effects of smoke exposure.

Third, the turnout effect is roughly twice as large at the tract level (0.005 vs.\ 0.002), both highly significant, with the same positive sign---smoke-exposed tracts turn out more.


% ============================================================
\section{House Elections}
% ============================================================

Table~\ref{tab:house} compares presidential and House results within the CA tract-level analysis.

\begin{table}[H]
\centering
\caption{House vs.\ Presidential Comparison (CA Tract-Level, Spec.\ 3, 30d)}
\label{tab:house}
\begin{threeparttable}
\small
\begin{tabular}{lcc}
\toprule
Outcome & Presidential & House \\
\midrule
DEM vote share & $-$0.00121$^{***}$ & 0.00193$^{***}$ \\
 & (0.00031) & (0.00021) \\[4pt]
Incumbent vote share & 0.01973$^{***}$ & 0.02128$^{***}$ \\
 & (0.00173) & (0.00143) \\[4pt]
Log total votes & 0.00489$^{***}$ & 0.01673$^{***}$ \\
 & (0.00090) & (0.00387) \\
\midrule
Elections & 2008--2020 & 2006--2022 \\
Observations & 22,727 & --- \\
\bottomrule
\end{tabular}
\begin{tablenotes}
\small
\item \textit{Notes:} Both use Spec.\ 3 (TWFE + controls, 30-day window), clustered by tract. House DEM vote share excludes same-party races (top-two primary). House turnout includes all races.
\end{tablenotes}
\end{threeparttable}
\end{table}

The DEM vote share effects are opposite-signed: smoke \emph{decreases} Democratic presidential vote share but \emph{increases} Democratic House vote share. The incumbent effect is remarkably consistent across races (${\sim}$0.02), reinforcing the interpretation as a general status-quo bias rather than an office-specific effect. The turnout effect is much larger for House elections, potentially because House races have lower baseline turnout and are more sensitive to mobilization shocks.


% ============================================================
\section{Robustness}
% ============================================================

\subsection{County$\times$Year Fixed Effects}

Table~\ref{tab:countyyear} reports the most demanding specification, which absorbs all county-level time-varying confounders and identifies effects solely from within-county, within-year variation across tracts.

\begin{table}[H]
\centering
\caption{County$\times$Year Fixed Effects (30d Mean PM$_{2.5}$)}
\label{tab:countyyear}
\begin{threeparttable}
\small
\begin{tabular}{lccc}
\toprule
 & DEM Vote Share & Incumbent Vote Share & Log Total Votes \\
\midrule
Smoke PM$_{2.5}$ (30d) & $-$0.00130$^{**}$ & $-$0.00201 & 0.00626$^{***}$ \\
 & (0.00052) & (0.00141) & (0.00234) \\
$p$-value & 0.012 & 0.155 & 0.008 \\
$N$ & 32,061 & 32,061 & 32,061 \\
\midrule
Tract FE & Yes & Yes & Yes \\
County$\times$Year FE & Yes & Yes & Yes \\
\bottomrule
\end{tabular}
\begin{tablenotes}
\small
\item \textit{Notes:} Standard errors clustered by tract. $^{*}p<0.10$, $^{**}p<0.05$, $^{***}p<0.01$. No additional controls included (county$\times$year FE absorbs county-level variation).
\end{tablenotes}
\end{threeparttable}
\end{table}

The DEM vote share effect ($-0.0013$, $p = 0.012$) and turnout effect (0.0063, $p = 0.008$) survive even this most demanding specification. The incumbent effect, however, becomes small, negative, and insignificant---suggesting it was partly driven by county-level confounders correlated with both smoke exposure and incumbency advantage.

This is a notable contrast with the national analysis, where state$\times$year FE eliminated all significance. The tract-level analysis retains statistical power because within-county, within-year variation in smoke is substantial (see Figure~\ref{fig:map_resid}).


\subsection{Threshold Comparison}

Table~\ref{tab:threshold} compares results across three smoke intensity thresholds, testing for dose-response patterns.

\begin{table}[H]
\centering
\caption{Threshold Comparison: Fraction of Days Above Threshold (30d, Spec.\ 3)}
\label{tab:threshold}
\begin{threeparttable}
\small
\begin{tabular}{lp{1.7cm}ccc}
\toprule
 & & Haze ($>$20) & USG ($>$35.5) & Unhealthy ($>$55.5) \\
\midrule
\multicolumn{2}{l}{\% with exposure} & 19.2\% & 3.9\% & 1.5\% \\[4pt]
DEM vote share & $\hat{\beta}$ & 0.001 & 0.086$^{***}$ & 0.189$^{***}$ \\
 & (SE) & (0.011) & (0.020) & (0.045) \\[4pt]
Incumbent & $\hat{\beta}$ & 1.284$^{***}$ & 2.601$^{***}$ & 4.013$^{***}$ \\
 & (SE) & (0.064) & (0.321) & (1.199) \\[4pt]
Log total votes & $\hat{\beta}$ & 0.196$^{***}$ & $-$0.029 & $-$0.281$^{*}$ \\
 & (SE) & (0.036) & (0.088) & (0.167) \\
\bottomrule
\end{tabular}
\begin{tablenotes}
\small
\item \textit{Notes:} Each cell reports a separate TWFE+controls regression. Treatment is the fraction of days in the 30-day window exceeding each threshold. Standard errors clustered by tract. $^{*}p<0.10$, $^{**}p<0.05$, $^{***}p<0.01$. ``Haze'' = visible haze onset; ``USG'' = EPA ``Unhealthy for Sensitive Groups''; ``Unhealthy'' = EPA ``Unhealthy.''
\end{tablenotes}
\end{threeparttable}
\end{table}

A clear dose-response pattern emerges for DEM vote share: the coefficient grows monotonically from near-zero at the haze threshold to 0.086 (USG) to 0.189 (unhealthy). At high intensity, smoke \emph{increases} Democratic vote share---the opposite of the mean PM$_{2.5}$ result---suggesting that severe, visible smoke events may activate pro-climate voting while moderate haze has the opposite effect.

The incumbent effect scales dramatically with intensity (1.28 $\to$ 2.60 $\to$ 4.01), while the turnout effect reverses: moderate smoke increases turnout but severe smoke decreases it, potentially because unhealthy air quality deters physical polling station visits.


\subsection{Controls Robustness}

Table~\ref{tab:controls_robust} reports leave-one-out sensitivity of the Spec.\ 3 estimates.

\begin{table}[H]
\centering
\caption{Controls Sensitivity: Leave-One-Out (Spec.\ 3, 30d)}
\label{tab:controls_robust}
\begin{threeparttable}
\small
\begin{tabular}{lccc}
\toprule
Specification & DEM & Incumbent & Turnout \\
\midrule
No controls & $-$0.00117$^{***}$ & 0.01054$^{***}$ & 0.00403$^{***}$ \\
Full controls & $-$0.00113$^{***}$ & 0.01942$^{***}$ & 0.00505$^{***}$ \\
Drop unemployment & $-$0.00126$^{***}$ & 0.01943$^{***}$ & 0.00534$^{***}$ \\
Drop log income & $-$0.00110$^{***}$ & 0.02062$^{***}$ & 0.00541$^{***}$ \\
Drop log population & $-$0.00114$^{***}$ & 0.01943$^{***}$ & 0.00492$^{***}$ \\
Drop Oct.\ temperature & $-$0.00113$^{***}$ & 0.01978$^{***}$ & 0.00495$^{***}$ \\
Drop Oct.\ precipitation & $-$0.00152$^{***}$ & 0.01909$^{***}$ & 0.00504$^{***}$ \\
Drop \% bachelor's+ & $-$0.00120$^{***}$ & 0.01959$^{***}$ & 0.00504$^{***}$ \\
Drop \% white non-Hispanic & $-$0.00116$^{***}$ & 0.01927$^{***}$ & 0.00508$^{***}$ \\
Drop \% Hispanic & $-$0.00115$^{***}$ & 0.01939$^{***}$ & 0.00498$^{***}$ \\
\bottomrule
\end{tabular}
\begin{tablenotes}
\small
\item \textit{Notes:} Each row drops one control variable. All $^{***}$ denote $p<0.01$. Coefficients are highly stable across specifications; no single control drives the results.
\end{tablenotes}
\end{threeparttable}
\end{table}

All three outcomes are robust to the exclusion of any individual control. The DEM coefficient ranges from $-0.0010$ to $-0.0015$; the incumbent coefficient from 0.019 to 0.021; and the turnout coefficient from 0.004 to 0.005. No single control is responsible for the estimated effects.


\subsection{Sensitivity to 2020}

The 2020 election coincided with historically extreme wildfire smoke in California (Figure~\ref{fig:map_byyear}). Figure~\ref{fig:drop2020} compares temporal dynamics for the full sample against a sample excluding 2020.

\begin{figure}[H]
\centering
\includegraphics[width=\textwidth]{ca_temporal_drop2020.png}
\caption{Temporal dynamics: full sample (blue) vs.\ excluding 2020 (red). The DEM effect largely disappears without 2020. The incumbent effect reverses sign, and confidence intervals widen substantially, reflecting the importance of 2020 for statistical power.}
\label{fig:drop2020}
\end{figure}

The 2020 fire season clearly dominates the identifying variation. Without 2020, the DEM effect is attenuated and imprecise, the incumbent effect reverses sign (suggesting that the pro-incumbent effect is specific to the 2020 context), and the turnout effect loses significance. This underscores both the power and the limitation of the California analysis: the extreme 2020 smoke events provide strong identification but also raise questions about external validity to more typical fire seasons.


% ============================================================
\section{Discussion}
% ============================================================

The tract-level analysis reveals several findings that are masked at the county level:

\begin{enumerate}
  \item \textbf{Larger incumbent and turnout effects.} The incumbent vote share effect is 27$\times$ larger at the tract level (0.020 vs.\ 0.001), suggesting that county-level averaging severely attenuates localized smoke impacts. This is consistent with smoke exposure being hyperlocal---tracts within the same county can experience dramatically different smoke levels.

  \item \textbf{County$\times$year FE preserves significance.} Unlike the national analysis where state$\times$year FE eliminated all effects, the tract-level DEM and turnout effects survive county$\times$year FE. This demonstrates that within-county spatial variation in smoke provides genuine identifying variation.

  \item \textbf{Dose-response in DEM share.} Moderate smoke ($>$20~$\mu$g/m$^3$) has no effect on partisan voting, but severe smoke ($>$55.5~$\mu$g/m$^3$) increases DEM vote share by 19 percentage points per unit fraction. This nonlinearity suggests different behavioral mechanisms at different exposure levels.

  \item \textbf{2020 dependence.} The extreme 2020 fire season provides most of the identifying variation. Results are less robust without it, particularly for the incumbent effect which reverses sign.
\end{enumerate}

\textbf{Limitations.} (1)~Single-state results may not generalize. (2)~Precinct-to-tract allocation introduces measurement error that likely attenuates estimates. (3)~ACS controls are available only for 2012+, leaving 29\% of observations without economic controls. (4)~Post-2012 same-party House races reduce the effective sample for House vote share analysis.


\newpage
\bibliographystyle{apalike}
\bibliography{references}

\end{document}
