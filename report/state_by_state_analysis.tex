\documentclass[12pt]{article}
\usepackage[margin=1in]{geometry}
\usepackage{graphicx}
\usepackage{booktabs}
\usepackage{amsmath}
\usepackage{natbib}
\usepackage{hyperref}
\usepackage{setspace}
\onehalfspacing

\title{State-by-State Turnout Analysis:\\A Nonparametric Robustness Check}
\author{}
\date{\today}

\begin{document}
\maketitle

\section{Motivation}

The pooled TWFE regression with state$\times$year fixed effects absorbs approximately 41\% of identifying variation in smoke exposure, rendering most coefficient estimates insignificant. This raises a legitimate concern: does the turnout effect survive only because of between-state$\times$year variation that might be confounded?

We address this by running TWFE+controls regressions \emph{within each state separately} and examining the distribution of coefficients across states. The logic is straightforward: if smoke genuinely increases turnout, we should observe systematically positive coefficients across states---even if many individual estimates are imprecise due to small within-state samples. Because each state has idiosyncratic politics, demographics, and economic conditions, a consistent positive pattern across diverse states would be difficult to attribute to any single confound. This is a nonparametric argument: we ask not whether the pooled effect is significant, but whether the \emph{sign pattern} is systematic.

\section{Method}

For each CONUS state with at least 10 counties ($S$ states total), we estimate:
\begin{equation}
y_{ct} = \beta_s \cdot \text{SmokePM2.5}_{ct}^{30d} + \mathbf{X}_{ct}\boldsymbol{\gamma}_s + \alpha_c + \delta_t + \varepsilon_{ct}
\end{equation}
where $y_{ct}$ is voter turnout rate (total votes / voting-age population), $\alpha_c$ and $\delta_t$ are county and year fixed effects, and $\mathbf{X}_{ct}$ includes unemployment rate, log median income, log population, October temperature, and October precipitation. Standard errors are clustered by county. States where the regression fails due to collinearity are excluded.

We then aggregate the state-level $\hat\beta_s$ estimates using:
\begin{itemize}
    \item \textbf{Sign test}: Fraction of states with $\hat\beta_s > 0$, tested against $H_0$: 50\% positive via exact binomial test.
    \item \textbf{Inverse-variance weighted meta-analytic mean}: $\bar\beta = \sum_s w_s \hat\beta_s / \sum_s w_s$ where $w_s = 1/\text{SE}_s^2$.
    \item \textbf{Cochran's $Q$} and $I^2$ for heterogeneity assessment.
    \item \textbf{Funnel plot} to diagnose potential publication-style bias (asymmetry would indicate that imprecise estimates are systematically different from precise ones).
\end{itemize}

\section{Results}

\subsection{Turnout Rate}

Figure~\ref{fig:forest} presents a forest plot of state-level coefficients ordered by precision. The meta-analytic mean and pooled TWFE+controls estimate are shown as reference lines. Figure~\ref{fig:histogram} shows the distribution of coefficients with sign test results. Figure~\ref{fig:funnel} presents the funnel plot.

\begin{figure}[htbp]
    \centering
    \includegraphics[width=\textwidth]{../output/figures/state_forest_turnout.png}
    \caption{Forest plot of state-by-state TWFE+controls estimates for turnout rate on smoke PM2.5 (30-day mean). States ordered by precision (most precise at top). Colors indicate statistical significance.}
    \label{fig:forest}
\end{figure}

\begin{figure}[htbp]
    \centering
    \includegraphics[width=0.8\textwidth]{../output/figures/state_histogram_turnout.png}
    \caption{Distribution of state-level coefficients for turnout rate. Vertical lines mark zero, the meta-analytic mean, and the median.}
    \label{fig:histogram}
\end{figure}

\begin{figure}[htbp]
    \centering
    \includegraphics[width=0.8\textwidth]{../output/figures/state_funnel_turnout.png}
    \caption{Funnel plot: precision (1/SE) vs.\ coefficient estimate. Dashed lines show pseudo-95\% confidence bounds around the meta-analytic mean. Approximate symmetry suggests no systematic bias.}
    \label{fig:funnel}
\end{figure}

\subsection{Log Total Votes (Corroboration)}

Figure~\ref{fig:forest_logvotes} shows the forest plot for log total votes, which serves as a corroboration of the turnout rate results using the simpler outcome measure that does not depend on VAP denominators.

\begin{figure}[htbp]
    \centering
    \includegraphics[width=\textwidth]{../output/figures/state_forest_logvotes.png}
    \caption{Forest plot of state-by-state TWFE+controls estimates for log total votes on smoke PM2.5 (30-day mean).}
    \label{fig:forest_logvotes}
\end{figure}

\section{Interpretation}

The state-by-state analysis provides a fundamentally different identification argument from the pooled regression. Rather than relying on a single coefficient that could be driven by a few influential states or a common macro shock, we ask: \emph{across the diversity of American states, is smoke exposure systematically associated with higher turnout?}

A strong majority of positive coefficients---especially if the sign test rejects 50\%---constitutes a nonparametric argument that is robust to any state-level confound. Each state serves as an independent quasi-experiment with its own political environment, economic conditions, and demographic composition. The meta-analytic framework then asks whether these independent estimates converge on a common direction.

The funnel plot provides a complementary diagnostic. If the positive pooled effect were driven by a few outlier states, we would expect funnel asymmetry---imprecise estimates (at the bottom) skewing in a particular direction. A symmetric funnel, by contrast, suggests that the positive tendency holds regardless of estimation precision.

This analysis does not replace the state$\times$year FE specification but offers an alternative lens: instead of demanding that all identifying variation come from within-state$\times$year (which absorbs 41\% of variation), we allow each state to contribute its own estimate and then ask whether the pattern across states is consistent with a real effect.

\end{document}
