\documentclass[12pt]{article}
\usepackage[margin=1in]{geometry}
\usepackage{graphicx}
\usepackage{booktabs}
\usepackage{amsmath}
\usepackage{hyperref}
\usepackage{setspace}
\onehalfspacing

\title{HMS Satellite Smoke Plume Robustness Check:\\Comparing ML-Predicted PM2.5 with Direct Satellite Observation}
\author{}
\date{\today}

\begin{document}
\maketitle

\section{Motivation}

Our main analysis uses wildfire smoke PM2.5 predictions from Childs et al.\ (2022), which applies machine learning to satellite imagery to estimate daily county-level smoke-attributable PM2.5 concentrations. A reviewer could object that ML predictions introduce measurement error or systematic bias. If an entirely independent smoke exposure measure---based on direct satellite observation rather than ML prediction---reproduces our findings, this substantially strengthens the paper.

NOAA's Hazard Mapping System (HMS) provides such an alternative. HMS analysts visually identify smoke plumes in satellite imagery and digitize polygon boundaries, classifying each as Light, Medium, or Heavy density. HMS data are completely independent of the Childs et al.\ pipeline: different satellites, different methodology (human visual analysis vs.\ ML), and different output (binary plume presence vs.\ continuous PM2.5). This comparison follows the approach of Borgschulte et al.\ (2022), who use HMS centroid-based county exposure as their primary treatment.

\section{Data and Method}

\subsection{HMS Exposure Construction}

We download annual HMS smoke plume shapefiles for the four presidential election years (2008, 2012, 2016, 2020) from NOAA NESDIS. We construct county-level exposure using the standard centroid method: a county is classified as smoke-exposed on a given day if its geographic centroid falls within any HMS smoke polygon. For each county-day, we record binary smoke presence and the maximum density classification (Light = 1, Medium = 2, Heavy = 3).

We aggregate into pre-election windows matching the Childs analysis: 7, 30, 60, 90 days, and full fire season (June 1 to Election Day). The primary HMS treatment variable is \texttt{hms\_frac\_days\_30d}---the fraction of the 30-day pre-election window with any satellite-observed smoke overhead---which is the HMS analog of \texttt{smoke\_frac\_haze\_30d} (fraction of days with Childs PM2.5 $>$ 20 $\mu$g/m$^3$).

\subsection{HMS Panel Summary}

The HMS panel contains 12,572 county-year observations (12,438 after merging with the Childs analysis panel). Key statistics for the 30-day window:

\begin{center}
\begin{tabular}{lcccc}
\toprule
Year & Mean smoke days & Counties with smoke & Max smoke days \\
\midrule
2008 & 0.2 & 422 (13.4\%) & 6 \\
2012 & 0.5 & 992 (31.6\%) & 6 \\
2016 & 0.2 & 545 (17.3\%) & 6 \\
2020 & 8.7 & 3,106 (98.8\%) & 31 \\
\bottomrule
\end{tabular}
\end{center}

The 2020 wildfire season dominates, with nearly all counties experiencing at least one day of satellite-observed smoke in the 30 days before the election.

\section{Part A: Validation}

\subsection{Daily Agreement}

At the county-day level (1,989,519 observations across all election windows), we classify each observation as smoke/no-smoke under both systems and compute agreement metrics.

\begin{center}
\begin{tabular}{lcc}
\toprule
& Childs $>0$ vs.\ HMS $=1$ & Childs $>20$ vs.\ HMS Medium+ \\
\midrule
Sensitivity & 0.987 & 0.072 \\
Specificity & 0.948 & 0.999 \\
Accuracy & 0.955 & 0.960 \\
Cohen's $\kappa$ & 0.861 & 0.127 \\
Prevalence (Childs) & 0.221 & 0.004 \\
Prevalence (HMS) & 0.181 & 0.043 \\
\midrule
Pearson $r$ (PM2.5 vs.\ density) & \multicolumn{2}{c}{0.431} \\
\bottomrule
\end{tabular}
\end{center}

At the \textbf{extensive margin} (any smoke), agreement is strong ($\kappa = 0.86$). The Childs ML model detects 98.7\% of HMS-identified smoke days, and 94.8\% of HMS no-smoke days are also Childs no-smoke days.

At the \textbf{intensive margin} (high-concentration smoke), agreement breaks down. Only 7.2\% of HMS Medium/Heavy days correspond to Childs PM2.5 $>$ 20 $\mu$g/m$^3$. This reflects a fundamental measurement difference: HMS classifies visual density from satellite imagery (which depends on plume altitude, column depth, and viewing angle), while Childs predicts ground-level PM2.5 concentration. A high-altitude dense plume may appear Heavy in HMS but produce minimal ground-level PM2.5; conversely, a dispersed low-altitude plume may elevate surface PM2.5 without being visually dense from space.

\subsection{Window-Level Correlation}

At the 30-day window level (12,438 county-year observations), correlations vary dramatically by variable:

\begin{center}
\begin{tabular}{lcc}
\toprule
Variable Pair & Pearson $r$ & $R^2$ \\
\midrule
Smoke days (Childs vs.\ HMS) & 0.911 & 0.831 \\
Mean PM2.5 vs.\ mean density & 0.666 & 0.443 \\
Cumulative PM2.5 vs.\ mean density & 0.660 & 0.436 \\
Max PM2.5 vs.\ max density & 0.683 & 0.467 \\
Frac haze ($>$20) vs.\ frac days (any) & 0.201 & 0.040 \\
Frac USG ($>$35.5) vs.\ frac medium+ & 0.234 & 0.055 \\
Frac unhealthy ($>$55.5) vs.\ frac heavy & 0.299 & 0.089 \\
\bottomrule
\end{tabular}
\end{center}

The pattern is clear: the two datasets agree strongly on \emph{whether} smoke is present (smoke days, $r = 0.91$) and moderately on \emph{how much} smoke is present (mean PM2.5 vs.\ density, $r = 0.67$), but disagree substantially on \emph{intensity thresholds} ($r \approx 0.2$--$0.3$). The fraction-above-threshold variables---which are the treatment variables in our regressions---capture fundamentally different aspects of smoke exposure.

Figure~\ref{fig:timeseries} shows daily national time series for each election year. The two measures track each other closely in timing (when smoke events occur) but differ in relative magnitude across events.

\begin{figure}[htbp]
    \centering
    \includegraphics[width=\textwidth]{../output/figures/hms_validation_timeseries.png}
    \caption{Daily national time series: Childs ML-predicted mean smoke PM2.5 (blue) vs.\ HMS fraction of counties with satellite-observed smoke (red), June 1 to Election Day. Dashed line marks Election Day. Correlation coefficients shown.}
    \label{fig:timeseries}
\end{figure}

\section{Part B: Regression Comparison}

\subsection{Build-Up Table}

Table~\ref{tab:buildup} presents the build-up specification table using \texttt{hms\_frac\_days\_30d} (HMS) alongside \texttt{smoke\_pm2.5\_mean\_30d} (Childs) for comparison. The Childs results replicate the paper's Table 1.

\begin{table}[htbp]
\centering
\caption{Build-Up Specification Table: Childs vs.\ HMS Treatment}
\label{tab:buildup}
\small
\begin{tabular}{lcccc}
\toprule
& (1) Raw OLS & (2) TWFE & (3) +Controls & (4) +St.\ Trends \\
\midrule
\multicolumn{5}{l}{\textbf{Panel A: Turnout Rate}} \\
Childs PM2.5 (30d) & 0.00536*** & 0.00100*** & 0.00085*** & 0.00065*** \\
& (0.00068) & (0.00020) & (0.00019) & (0.00016) \\
HMS frac days (30d) & 0.12050*** & $-$0.04203*** & $-$0.03238*** & $-$0.01454* \\
& (0.00447) & (0.00946) & (0.00862) & (0.00823) \\
Std.\ $\beta$ (Spec 3) & \multicolumn{4}{c}{Childs: $+$0.0022 \quad HMS: $-$0.0042} \\[6pt]
\multicolumn{5}{l}{\textbf{Panel B: Log Total Votes}} \\
Childs PM2.5 (30d) & $-$0.02911*** & 0.00314*** & 0.00245*** & 0.00124*** \\
& (0.00504) & (0.00052) & (0.00051) & (0.00024) \\
HMS frac days (30d) & $-$0.11497** & $-$0.06687*** & $-$0.00807 & 0.04425*** \\
& (0.05857) & (0.02559) & (0.02468) & (0.01482) \\
Std.\ $\beta$ (Spec 3) & \multicolumn{4}{c}{Childs: $+$0.0063 \quad HMS: $-$0.0010} \\[6pt]
\multicolumn{5}{l}{\textbf{Panel C: DEM Vote Share}} \\
Childs PM2.5 (30d) & $-$0.00831*** & 0.00135*** & 0.00125*** & $-$0.00016 \\
& (0.00132) & (0.00021) & (0.00023) & (0.00021) \\
HMS frac days (30d) & $-$0.20738*** & $-$0.02617** & 0.00646 & $-$0.04905*** \\
& (0.00733) & (0.01118) & (0.01102) & (0.00847) \\
Std.\ $\beta$ (Spec 3) & \multicolumn{4}{c}{Childs: $+$0.0032 \quad HMS: $+$0.0008} \\[6pt]
\multicolumn{5}{l}{\textbf{Panel D: Incumbent Vote Share}} \\
Childs PM2.5 (30d) & 0.02028*** & $-$0.00172* & 0.00070 & 0.00080 \\
& (0.00259) & (0.00091) & (0.00094) & (0.00108) \\
HMS frac days (30d) & 0.71349*** & 0.24357*** & 0.21560*** & 0.23721*** \\
& (0.01567) & (0.05609) & (0.04689) & (0.07053) \\
Std.\ $\beta$ (Spec 3) & \multicolumn{4}{c}{Childs: $+$0.0018 \quad HMS: $+$0.0280} \\
\midrule
County FE & & Yes & Yes & Yes \\
Year FE & & Yes & Yes & Yes \\
Controls & & & Yes & Yes \\
State trends & & & & Yes \\
\bottomrule
\end{tabular}
\begin{flushleft}
\footnotesize Notes: Standard errors clustered by county in parentheses. * $p<0.10$, ** $p<0.05$, *** $p<0.01$. Standardized $\beta$ = coefficient $\times$ SD of treatment variable, enabling scale-comparable comparison. $N \approx 12{,}200$ (varies by outcome due to missing values). Controls: unemployment rate, log median income, log population, October temperature, October precipitation.
\end{flushleft}
\end{table}

\subsection{Key Discrepancies}

\textbf{Turnout:} The Childs and HMS results are \emph{opposite in sign}. Childs PM2.5 produces a robust positive turnout effect ($+$0.00085***, surviving all specifications). HMS frac days produces a \emph{negative} effect ($-$0.032***). Even after standardization, the signs disagree: $+$0.0022 vs.\ $-$0.0042.

\textbf{Incumbent vote share:} HMS shows a large, stable, positive effect on incumbent vote share across all specifications ($+$0.22*** in Spec 3). Childs shows near-zero effects after controlling for observables. The HMS effect is an order of magnitude larger in standardized terms ($+$0.028 vs.\ $+$0.002).

\textbf{DEM vote share:} Both measures produce null effects in Spec 3 (with controls), which is consistent.

\subsection{Threshold Comparison}

When we restrict to \emph{heavy} HMS smoke (density $\geq 3$), the results converge:

\begin{center}
\begin{tabular}{lcccc}
\toprule
Treatment (Spec 3) & Turnout rate & Log votes & DEM share & Incumbent \\
\midrule
HMS: Any smoke & $-$0.032*** & $-$0.008 & $+$0.006 & $+$0.216*** \\
HMS: Heavy only & $+$0.077*** & $+$0.056 & $+$0.108*** & $+$0.881*** \\
Childs: Haze ($>$20) & $+$0.045*** & $+$0.106*** & $+$0.062*** & $+$0.163** \\
Childs: Unhealthy ($>$55.5) & $+$0.049** & $+$0.098 & $+$0.150*** & $-$0.249*** \\
\bottomrule
\end{tabular}
\end{center}

At high intensity, both measures show positive turnout and DEM vote share effects. The convergence at the heavy/unhealthy threshold---despite the weak correlation between these variables ($r = 0.30$)---suggests that severe smoke events produce real behavioral effects regardless of measurement approach.

\begin{figure}[htbp]
    \centering
    \includegraphics[width=\textwidth]{../output/figures/hms_buildup_comparison.png}
    \caption{Standardized coefficient comparison ($\beta \times \text{SD}_{\text{treatment}}$) across four specifications and four outcomes. Childs ML-predicted PM2.5 (blue) vs.\ HMS satellite smoke fraction (red). Error bars show 95\% confidence intervals.}
    \label{fig:comparison}
\end{figure}

\section{Interpretation}

The divergence between Childs and HMS results for the primary treatment variable is informative rather than damaging. It reflects a fundamental measurement difference:

\begin{enumerate}
    \item \textbf{HMS ``any smoke''} is an extremely broad measure. Over the full fire season, 98--100\% of counties have at least one day of satellite-observed smoke. This variable primarily captures geographic variation in proximity to fire-prone regions---which is strongly correlated with state-level political, economic, and demographic characteristics. After absorbing county and year fixed effects, the remaining within-county variation in HMS ``any smoke'' fraction may be dominated by differences in satellite overpass timing and analyst classification decisions rather than genuine exposure differences.

    \item \textbf{Childs PM2.5} provides continuous ground-level concentration estimates. The treatment variable (mean smoke PM2.5) captures \emph{intensity} variation, not just extensive-margin presence. This makes it substantially more informative about the dose of smoke pollution voters actually experience.

    \item The two measures \textbf{converge at high intensity}. When both measures identify severe smoke events---HMS Heavy and Childs $>$55.5 $\mu$g/m$^3$---they produce similar positive effects on turnout and DEM vote share. This convergence at the intensive margin, despite weak correlation between the threshold variables ($r = 0.30$), is a form of triangulation: two independent measurement systems identify the same behavioral response to unambiguously severe smoke.
\end{enumerate}

The negative HMS turnout coefficient likely reflects a composition effect: the ``any smoke'' variable is dominated by light, transient smoke that may signal environmental salience differently than the sustained, high-concentration exposure captured by Childs PM2.5. The incumbent vote share discrepancy is harder to interpret and warrants further investigation.

\section{Conclusion}

The HMS comparison provides a mixed but ultimately informative robustness check. The two measurement systems agree strongly on when and where smoke is present ($\kappa = 0.86$, smoke days $r = 0.91$) but disagree on intensity classification ($r \approx 0.2$--$0.3$ at thresholds). This disagreement propagates into regressions: the broad HMS ``any smoke'' treatment produces different---sometimes opposite---effects from Childs PM2.5.

However, the convergence at high intensity supports the paper's core finding: severe wildfire smoke increases voter turnout. The disagreement at low intensity is best understood as reflecting different aspects of smoke exposure rather than contradictory evidence about voter behavior.

\end{document}
