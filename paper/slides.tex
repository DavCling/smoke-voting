\documentclass[aspectratio=169,12pt]{beamer}
\usetheme{metropolis}
\usepackage{booktabs}
\usepackage{graphicx}
\usepackage{natbib}
\usepackage{appendixnumberbeamer}

\graphicspath{{../output/figures/}}
\bibliographystyle{apalike}

\title{Wildfire Smoke and Voting Behavior\\in the United States}
\subtitle{Preliminary Results}
\author{David Clingingsmith\\\small Case Western Reserve University}
\date{\today}

\begin{document}

\maketitle

%----------------------------------------------------------------------
\begin{frame}{Motivation}

\begin{itemize}
  \item Wildfire smoke is a very widespread \textit{experiential} consequence of climate change in the U.S.
  \begin{itemize}
    \item Unlike ambient air pollution, smoke events are visible, sudden, and directly attributable to wildfires --- making them potentially more \textit{salient} as climate signals
  \end{itemize}
  \item Does smoke exposure change whether and how people vote?
  \item Prior work:
  \begin{itemize}
    \item \textbf{Fire proximity} $\rightarrow$ pro-environment voting in CA, but only among Democrats \citep{hazlett2020wildfire}
    \item \textbf{Overall air pollution (PM\textsubscript{10})} $\rightarrow$ anti-incumbent voting in Germany \citep{bellani2024air}
    \item \textbf{Rain on election day} $\rightarrow$ lower turnout \citep{gomez2007weather}
  \end{itemize}
  \item \textbf{Gap:} No study has linked wildfire-specific smoke PM\textsubscript{2.5} to U.S.\ election outcomes
\end{itemize}

\end{frame}

%----------------------------------------------------------------------
\begin{frame}{Data}

\begin{enumerate}
  \item \textbf{Wildfire smoke PM\textsubscript{2.5}} --- Stanford Echo Lab \citep{childs2022daily}
  \begin{itemize}
    \item Daily, county-level, 2006--2020
    \item ML separation of wildfire smoke from background PM\textsubscript{2.5}
  \end{itemize}
  \vspace{0.3cm}
  \item \textbf{Election returns} --- MIT Election Data Lab \citep{medsl2024county}
  \begin{itemize}
    \item Presidential: county-level, 2000--2024
    \item House: precinct-level returns aggregated to county, 2016--2020
  \end{itemize}
  \vspace{0.3cm}
  \item \textbf{Analysis samples:}
  \begin{itemize}
    \item Presidential: 12,429 county $\times$ election obs (2008, 2012, 2016, 2020)
    \item House: 9,171 county $\times$ election obs (2016, 2018, 2020)
  \end{itemize}
\end{enumerate}

\end{frame}

%----------------------------------------------------------------------
\begin{frame}{Smoke Exposure Varies Dramatically Across Elections}

\begin{figure}
\centering
\includegraphics[width=\textwidth]{smoke_exposure_map_panel.png}
\end{figure}

\vspace{-0.3cm}
{\small Mean wildfire smoke PM\textsubscript{2.5} ($\mu$g/m$^3$) in the 30 days before election day.}

\end{frame}

%----------------------------------------------------------------------
\begin{frame}{Empirical Strategy}

Two-way fixed effects:
\begin{equation*}
Y_{ct} = \alpha_c + \gamma_t + \beta \cdot \text{SmokePM}_{ct} + \varepsilon_{ct}
\end{equation*}

\begin{itemize}
  \item $\alpha_c$: County FE --- absorb all time-invariant confounders
  \item $\gamma_t$: Election year FE --- absorb national swings
  \item SEs clustered by county
  \item Treatment: mean smoke PM\textsubscript{2.5} in the 60 days before election
\end{itemize}

\vspace{0.3cm}
\textbf{Identifying assumption:} Conditional on county and year FE, variation in smoke exposure is uncorrelated with unobserved determinants of voting. This is plausible because smoke plume direction is determined by wind, not by county politics or demographics.

\end{frame}

%----------------------------------------------------------------------
\begin{frame}{Identification: Threats and Estimator Choice}

\textbf{Potential threats:}
\begin{itemize}
  \item Spatially correlated shocks (e.g., drought affects both fires and local economy)
  \begin{itemize}
    \item Mitigated: smoke travels hundreds of miles from fire origin
  \end{itemize}
  \item Secular trends in fire-prone vs.\ non-fire-prone regions
  \begin{itemize}
    \item Mitigated: county FE absorb levels; year FE absorb national trends
  \end{itemize}
\end{itemize}

\vspace{0.3cm}
\textbf{TWFE with continuous treatment:}
\begin{itemize}
  \item \citet*{callaway2024continuous} show TWFE with a continuous treatment can produce coefficients with ambiguous causal interpretation due to heterogeneous dose--response weighting
  \item Our setting mitigates this: treatment is atmospherically assigned (limiting selection into dose); we estimate a linear slope (non-negative ACRT weights)
  \item Robustness: results hold when treatment is dichotomized or binned
\end{itemize}

\end{frame}

%----------------------------------------------------------------------
\begin{frame}{Main Results: Presidential Elections}

\begin{table}
\centering
\small
\begin{tabular}{lccc}
\toprule
& (1) & (2) & (3) \\
& DEM Vote Share & Incumbent Share & Log Turnout \\
\midrule
Smoke PM\textsubscript{2.5} (60d) & 0.00087*** & $-$0.00399*** & 0.00242*** \\
& (0.00009) & (0.00044) & (0.00018) \\[6pt]
\midrule
County FE & Yes & Yes & Yes \\
Year FE & Yes & Yes & Yes \\
$N$ & 12,429 & 12,429 & 12,429 \\
\bottomrule
\end{tabular}
\end{table}

\vspace{0.3cm}
\begin{itemize}
  \item \textbf{+10 $\mu$g/m$^3$ smoke $\rightarrow$ +0.9 pp DEM vote share}
  \item Anti-incumbent effect is $\sim$4x larger than pro-DEM effect
  \item No evidence of turnout suppression
\end{itemize}

\end{frame}

%----------------------------------------------------------------------
\begin{frame}{Effect Across the Partisan Spectrum}

\begin{table}
\centering
\small
\begin{tabular}{lccc}
\toprule
& R-Leaning & Swing & D-Leaning \\
\midrule
Smoke PM\textsubscript{2.5} (60d) & 0.00066*** & 0.00049*** & 0.00082*** \\
& (0.00021) & (0.00014) & (0.00013) \\[6pt]
$N$ & 4,144 & 4,141 & 4,143 \\
\bottomrule
\end{tabular}
\end{table}

\vspace{0.3cm}
\begin{itemize}
  \item Effect is \textbf{present in all terciles} of prior partisanship
  \item Somewhat larger in D-leaning counties
  \item Contrast with \citet{hazlett2020wildfire}: fire proximity affects \textit{only} Democratic areas
  \item Smoke is a broader, less politically sorted treatment
\end{itemize}

\end{frame}

%----------------------------------------------------------------------
\begin{frame}{Temporal Dynamics}

\begin{columns}[T]
\begin{column}{0.55\textwidth}
\begin{figure}
\centering
\includegraphics[width=\textwidth]{event_study_windows.png}
\end{figure}
\end{column}
\begin{column}{0.42\textwidth}
\vspace{1cm}
\begin{itemize}
  \item Effect significant at all windows
  \item Strongest at 30 days
  \item Consistent with recency / salience mechanism
  \item Not just election-day disruption
\end{itemize}
\end{column}
\end{columns}

\end{frame}

%----------------------------------------------------------------------
\begin{frame}{Binscatter: Smoke and Democratic Vote Share}

\begin{figure}
\centering
\includegraphics[width=0.65\textwidth]{binscatter_smoke_dem_share.png}
\end{figure}

\vspace{-0.2cm}
{\small County and year FE residualized. 50 equal-sized bins of smoke exposure.}

\end{frame}

%----------------------------------------------------------------------
\begin{frame}{House Elections: County-Level Analysis}

\begin{table}
\centering
\small
\begin{tabular}{lcc}
\toprule
& (1) County House & (2) Presidential \\
\midrule
\textit{DEM Vote Share} & 0.00038*** & 0.00087*** \\
& (0.00013) & (0.00009) \\[4pt]
\textit{Incumbent Share} & $-$0.00153*** & $-$0.00399*** \\
& (0.00045) & (0.00044) \\[4pt]
\midrule
Unit & County & County \\
$N$ (contested) & 8,391 & 12,429 \\
Elections & 2016--2020 & 2008--2020 \\
\bottomrule
\end{tabular}
\end{table}

\vspace{0.2cm}
\begin{itemize}
  \item County-level House confirms both pro-DEM and anti-incumbent effects
  \item Magnitudes smaller than presidential, consistent with candidate-driven races
  \item Same county-level unit avoids crosswalk measurement error
\end{itemize}

\end{frame}

%----------------------------------------------------------------------
\begin{frame}{What Mechanism?}

\begin{table}
\centering
\small
\begin{tabular}{lccc}
\toprule
Mechanism & Turnout? & Partisan pattern & Our evidence \\
\midrule
\textbf{Salience} & No & Pro-environment & $\checkmark$ DEM shift \\
\textbf{Negative affect} & No & Anti-incumbent & $\checkmark$ Large anti-incumb. \\
\textbf{Disruption} & Suppression & Differential & $\times$ No suppression \\
\bottomrule
\end{tabular}
\end{table}

\vspace{0.3cm}
Evidence is most consistent with \textbf{both} salience and negative affect channels operating simultaneously.

\end{frame}

%----------------------------------------------------------------------
\begin{frame}{Limitations and Next Steps}

\textbf{Current limitations:}
\begin{itemize}
  \item Only 4 presidential elections; 3 House elections (smoke data: 2006--2020)
  \item County-level aggregation; no individual-level variation
  \item Turnout measure is crude (log total votes without population denominator)
\end{itemize}

\vspace{0.3cm}
\textbf{Planned extensions:}
\begin{itemize}
  \item NOAA HMS smoke plumes for extended coverage through 2024
  \item State legislative elections
  \item Wind direction as instrument for smoke exposure
  \item Conley spatial SEs for inference robust to spatial correlation
\end{itemize}

\end{frame}

%----------------------------------------------------------------------
\begin{frame}{Summary}

\begin{enumerate}
  \item Wildfire smoke \textbf{increases Democratic vote share} and \textbf{punishes incumbents}
  \item Effects are \textbf{nationally representative} and \textbf{cross the partisan spectrum}
  \item Smoke is \textbf{plausibly exogenous} (wind-driven) and affects \textbf{far more people} than fire proximity
  \item Consistent with both climate salience and negative affect mechanisms
\end{enumerate}

\end{frame}

%----------------------------------------------------------------------
\appendix
\begin{frame}[allowframebreaks]{References}
\small
\bibliography{references}
\end{frame}

\end{document}
