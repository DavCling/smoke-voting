\documentclass[aspectratio=169,12pt]{beamer}
\usetheme{metropolis}
\usepackage{booktabs}
\usepackage{graphicx}
\usepackage{natbib}
\usepackage{appendixnumberbeamer}

\graphicspath{{}}
\bibliographystyle{apalike}

\title{Wildfire Smoke and Voting Behavior\\in the United States}
\subtitle{Preliminary Results}
\author{}
\date{\today}

\begin{document}

\maketitle

%----------------------------------------------------------------------
\begin{frame}{Motivation}

\begin{itemize}
  \item Wildfire smoke is the most widespread \textit{experiential} consequence of climate change in the U.S.
  \item Does smoke exposure change how people vote?
  \item Prior work:
  \begin{itemize}
    \item \textbf{Fire proximity} $\rightarrow$ pro-environment voting in CA, but only among Democrats \citep{hazlett2020wildfire}
    \item \textbf{Air pollution (PM\textsubscript{10})} $\rightarrow$ anti-incumbent voting in Germany \citep{bellani2024air}
    \item \textbf{Rain on election day} $\rightarrow$ lower turnout \citep{gomez2007weather}
  \end{itemize}
  \item \textbf{Gap:} Nobody has linked wildfire-specific smoke PM\textsubscript{2.5} to U.S.\ election outcomes
\end{itemize}

\end{frame}

%----------------------------------------------------------------------
\begin{frame}{Why Smoke $>$ Fire Proximity}

\begin{columns}[T]
\begin{column}{0.48\textwidth}
\textbf{Fire perimeters}
\begin{itemize}
  \item Treatment: $\sim$1,300 block groups near fire lines
  \item California only
  \item Confounded by property destruction, displacement, insurance
  \item Endogenous to land use
\end{itemize}
\end{column}
\begin{column}{0.48\textwidth}
\textbf{Wildfire smoke}
\begin{itemize}
  \item Treatment: \textit{every county in the U.S.}
  \item National scope
  \item Isolates experiential/health channel
  \item Plausibly exogenous (wind-driven)
\end{itemize}
\end{column}
\end{columns}

\end{frame}

%----------------------------------------------------------------------
\begin{frame}{Data}

\begin{enumerate}
  \item \textbf{Wildfire smoke PM\textsubscript{2.5}} --- Stanford Echo Lab \citep{childs2022daily}
  \begin{itemize}
    \item Daily, county-level, 2006--2020
    \item ML separation of wildfire smoke from background PM\textsubscript{2.5}
  \end{itemize}
  \vspace{0.3cm}
  \item \textbf{Presidential election returns} --- MIT Election Data Lab \citep{medsl2024county}
  \begin{itemize}
    \item County-level, 2000--2024
  \end{itemize}
  \vspace{0.3cm}
  \item \textbf{Analysis sample:} 12,429 county $\times$ election observations\\
  3,108 counties $\times$ 4 elections (2008, 2012, 2016, 2020)
\end{enumerate}

\end{frame}

%----------------------------------------------------------------------
\begin{frame}{Smoke Exposure Varies Dramatically Across Elections}

\begin{figure}
\centering
\includegraphics[width=\textwidth]{smoke_exposure_map_panel.png}
\end{figure}

\vspace{-0.3cm}
{\small Mean wildfire smoke PM\textsubscript{2.5} ($\mu$g/m$^3$) in the 30 days before election day.}

\end{frame}

%----------------------------------------------------------------------
\begin{frame}{Empirical Strategy}

Two-way fixed effects:
\begin{equation*}
Y_{ct} = \alpha_c + \gamma_t + \beta \cdot \text{SmokePM}_{ct} + \varepsilon_{ct}
\end{equation*}

\begin{itemize}
  \item $\alpha_c$: County FE --- absorb all time-invariant confounders
  \item $\gamma_t$: Election year FE --- absorb national swings
  \item SEs clustered by county
  \item Treatment: mean smoke PM\textsubscript{2.5} in the 60 days before election
\end{itemize}

\vspace{0.3cm}
\textbf{Identification:} Smoke plume direction is determined by wind, not by county politics or demographics.

\end{frame}

%----------------------------------------------------------------------
\begin{frame}{Main Results}

\begin{table}
\centering
\small
\begin{tabular}{lccc}
\toprule
& (1) & (2) & (3) \\
& DEM Vote Share & Incumbent Share & Log Turnout \\
\midrule
Smoke PM\textsubscript{2.5} (60d) & 0.00087*** & $-$0.00399*** & 0.00242*** \\
& (0.00009) & (0.00044) & (0.00018) \\[6pt]
\midrule
County FE & Yes & Yes & Yes \\
Year FE & Yes & Yes & Yes \\
$N$ & 12,429 & 12,429 & 12,429 \\
\bottomrule
\end{tabular}
\end{table}

\vspace{0.3cm}
\begin{itemize}
  \item \textbf{+10 $\mu$g/m$^3$ smoke $\rightarrow$ +0.9 pp DEM vote share}
  \item Anti-incumbent effect is $\sim$4x larger than pro-DEM effect
  \item No evidence of turnout suppression
\end{itemize}

\end{frame}

%----------------------------------------------------------------------
\begin{frame}{Effect Across the Partisan Spectrum}

\begin{table}
\centering
\small
\begin{tabular}{lccc}
\toprule
& R-Leaning & Swing & D-Leaning \\
\midrule
Smoke PM\textsubscript{2.5} (60d) & 0.00066*** & 0.00049*** & 0.00082*** \\
& (0.00021) & (0.00014) & (0.00013) \\[6pt]
$N$ & 4,144 & 4,141 & 4,143 \\
\bottomrule
\end{tabular}
\end{table}

\vspace{0.3cm}
\begin{itemize}
  \item Effect is \textbf{present in all terciles} of prior partisanship
  \item Somewhat larger in D-leaning counties
  \item Contrast with \citet{hazlett2020wildfire}: fire proximity affects \textit{only} Democratic areas
  \item Smoke is a broader, less politically sorted treatment
\end{itemize}

\end{frame}

%----------------------------------------------------------------------
\begin{frame}{Temporal Dynamics}

\begin{columns}[T]
\begin{column}{0.55\textwidth}
\begin{figure}
\centering
\includegraphics[width=\textwidth]{event_study_windows.png}
\end{figure}
\end{column}
\begin{column}{0.42\textwidth}
\vspace{1cm}
\begin{itemize}
  \item Effect significant at all windows
  \item Strongest at 30 days
  \item Consistent with recency / salience mechanism
  \item Not just election-day disruption
\end{itemize}
\end{column}
\end{columns}

\end{frame}

%----------------------------------------------------------------------
\begin{frame}{Binscatter: Smoke and Democratic Vote Share}

\begin{figure}
\centering
\includegraphics[width=0.65\textwidth]{binscatter_smoke_dem_share.png}
\end{figure}

\vspace{-0.2cm}
{\small County and year FE residualized. 20 equal-sized bins of smoke exposure.}

\end{frame}

%----------------------------------------------------------------------
\begin{frame}{What Mechanism?}

\begin{table}
\centering
\small
\begin{tabular}{lccc}
\toprule
Mechanism & Turnout? & Partisan pattern & Our evidence \\
\midrule
\textbf{Salience} & No & Pro-environment & $\checkmark$ DEM shift \\
\textbf{Negative affect} & No & Anti-incumbent & $\checkmark$ Large anti-incumb. \\
\textbf{Disruption} & Suppression & Differential & $\times$ No suppression \\
\bottomrule
\end{tabular}
\end{table}

\vspace{0.3cm}
Evidence is most consistent with \textbf{both} salience and negative affect channels operating simultaneously.

\end{frame}

%----------------------------------------------------------------------
\begin{frame}{Limitations and Next Steps}

\textbf{Current limitations:}
\begin{itemize}
  \item Only 4 presidential elections (smoke data: 2006--2020)
  \item County-level aggregation; no individual-level variation
  \item Turnout measure is crude (no population denominator)
\end{itemize}

\vspace{0.3cm}
\textbf{Planned extensions:}
\begin{itemize}
  \item NOAA HMS smoke plumes for extended coverage through 2024
  \item Congressional and state legislative elections
  \item Wind direction as instrument for smoke exposure
  \item State $\times$ year FE; Conley spatial SEs
\end{itemize}

\end{frame}

%----------------------------------------------------------------------
\begin{frame}{Summary}

\begin{enumerate}
  \item Wildfire smoke \textbf{increases Democratic vote share} and \textbf{punishes incumbents}
  \item Effects are \textbf{nationally representative} and \textbf{cross the partisan spectrum}
  \item Smoke is \textbf{plausibly exogenous} (wind-driven) and affects \textbf{far more people} than fire proximity
  \item Consistent with both climate salience and negative affect mechanisms
\end{enumerate}

\end{frame}

%----------------------------------------------------------------------
\appendix
\begin{frame}[allowframebreaks]{References}
\small
\bibliography{references}
\end{frame}

\end{document}
