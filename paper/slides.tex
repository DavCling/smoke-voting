\documentclass[aspectratio=169,12pt]{beamer}
\usetheme{metropolis}
\usepackage{booktabs}
\usepackage{graphicx}
\usepackage{natbib}
\usepackage{appendixnumberbeamer}

\graphicspath{{../output/figures/}}
\bibliographystyle{apalike}

\title{Wildfire Smoke and Voting Behavior\\in the United States}
\subtitle{Preliminary Results}
\author{David Clingingsmith\\\small Case Western Reserve University}
\date{\today}

\begin{document}

\maketitle

%----------------------------------------------------------------------
\begin{frame}{Motivation}

\begin{itemize}
  \item Wildfire smoke is a very widespread \textit{experiential} consequence of climate change in the U.S.
  \begin{itemize}
    \item Unlike ambient air pollution, smoke events are visible, sudden, and directly attributable to wildfires --- making them potentially more \textit{salient} as climate signals
    \item Tens of millions of Americans experience unhealthy air from wildfire smoke each year; awareness is high and rising
  \end{itemize}
  \item Does smoke exposure change whether and how people vote?
\end{itemize}

\end{frame}

%----------------------------------------------------------------------
\begin{frame}{Prior Work}

\begin{itemize}
  \item \textbf{Fire proximity} $\rightarrow$ pro-environment voting in CA, but only among Democrats \citep{hazlett2020wildfire}
  \item \textbf{Overall air pollution (PM\textsubscript{10})} $\rightarrow$ anti-incumbent voting in Germany \citep{bellani2024air}
  \item \textbf{Rain on election day} $\rightarrow$ lower turnout \citep{gomez2007weather}
\end{itemize}

\vspace{0.4cm}
\textbf{Gap:} No study has linked wildfire-specific smoke PM\textsubscript{2.5} to U.S.\ election outcomes

\vspace{0.2cm}
\begin{itemize}
  \item Smoke differs from generic air pollution: episodic, visible, attributable to a specific cause
  \item Smoke differs from fire proximity: affects far more people, plausibly exogenous (wind-driven), isolates the experiential channel from property destruction
\end{itemize}

\end{frame}

%----------------------------------------------------------------------
\begin{frame}{Data}

\begin{enumerate}
  \item \textbf{Wildfire smoke PM\textsubscript{2.5}} --- \citet{childs2022daily}
  \begin{itemize}
    \item Daily, county-level, 2006--2020
    \item Uses NOAA HMS satellite smoke plume classifications combined with machine learning to isolate wildfire-attributed PM\textsubscript{2.5} from background pollution
  \end{itemize}
  \vspace{0.3cm}
  \item \textbf{Election returns} --- MIT Election Data Lab \citep{medsl2024county}
  \begin{itemize}
    \item Presidential: county-level (2008, 2012, 2016, 2020)
    \item House: precinct-level returns aggregated to county (2016, 2018, 2020)
  \end{itemize}
  \vspace{0.3cm}
  \item \textbf{Analysis samples:}
  \begin{itemize}
    \item Presidential: 12,429 county $\times$ election obs (4 cycles)
    \item House: 9,171 county $\times$ election obs (3 cycles)
  \end{itemize}
\end{enumerate}

\end{frame}

%----------------------------------------------------------------------
\begin{frame}{Smoke Exposure Varies Dramatically Across Elections}

\begin{figure}
\centering
\includegraphics[width=\textwidth,trim=0 0 0 0,clip]{smoke_exposure_map_panel.png}
\end{figure}

\end{frame}

%----------------------------------------------------------------------
\begin{frame}{Empirical Strategy}

Two-way fixed effects:
\begin{equation*}
Y_{ct} = \alpha_c + \gamma_t + \beta \cdot \text{SmokePM}_{ct} + \varepsilon_{ct}
\end{equation*}

\begin{itemize}
  \item $\alpha_c$: County FE --- absorb all time-invariant confounders
  \item $\gamma_t$: Election year FE --- absorb national swings
  \item SEs clustered by county
  \item Treatment: mean smoke PM\textsubscript{2.5} in the 60 days before election
\end{itemize}

\vspace{0.3cm}
\textbf{Identifying assumption:} Conditional on county and year FE, variation in smoke exposure is uncorrelated with unobserved determinants of voting. This is plausible because smoke plume direction is determined by wind, not by county politics or demographics.

\end{frame}

%----------------------------------------------------------------------
\begin{frame}{Identification: Threats}

\textbf{Potential threats:}
\begin{itemize}
  \item Spatially correlated shocks (e.g., drought affects both fires and local economy)
  \begin{itemize}
    \item Mitigated: smoke travels hundreds of miles from fire origin
  \end{itemize}
  \item Secular trends in fire-prone vs.\ non-fire-prone regions
  \begin{itemize}
    \item Mitigated: county FE absorb levels; year FE absorb national trends
  \end{itemize}
\end{itemize}

\end{frame}

%----------------------------------------------------------------------
\begin{frame}{TWFE with Continuous Treatment}

\citet*{callaway2024continuous} show TWFE with a continuous treatment can produce coefficients with ambiguous causal interpretation due to heterogeneous dose--response weighting.

\vspace{0.3cm}
\textbf{Why this is less concerning here:}
\begin{itemize}
  \item Treatment is atmospherically assigned --- limiting selection into dose levels
  \item We estimate a linear slope corresponding to the average causal response (ACRT) decomposition, in which weights are non-negative
  \item Robustness: results are qualitatively similar when treatment is dichotomized or discretized into dose bins
\end{itemize}

\end{frame}

%----------------------------------------------------------------------
\begin{frame}{Main Results: Presidential and House Elections}

\begin{table}
\centering
\small
\begin{tabular}{lccc}
\toprule
& (1) & (2) & (3) \\
& DEM Vote Share & Incumbent Share & Log Turnout \\
\midrule
\multicolumn{4}{l}{\textit{Panel A: Presidential (2008--2020)}} \\[3pt]
Smoke PM\textsubscript{2.5} (60d) & 0.00087*** & $-$0.00399*** & 0.00242*** \\
& (0.00009) & (0.00044) & (0.00018) \\
$N$ & 12,429 & 12,429 & 12,429 \\[6pt]
\multicolumn{4}{l}{\textit{Panel B: County-Level House (2016--2020)}} \\[3pt]
Smoke PM\textsubscript{2.5} (60d) & 0.00038*** & $-$0.00153*** & 0.00177*** \\
& (0.00013) & (0.00045) & (0.00066) \\
$N$ & 8,391 & 8,391 & 9,165 \\[3pt]
\bottomrule
\multicolumn{4}{l}{\footnotesize County and year FE. SEs clustered by county. *** $p<0.01$.}
\end{tabular}
\end{table}

\end{frame}

%----------------------------------------------------------------------
\begin{frame}{Interpreting the Main Results}

\begin{itemize}
  \item \textbf{+10 $\mu$g/m$^3$ smoke $\rightarrow$ +0.9 pp DEM vote share} (presidential)
  \item Anti-incumbent effect is $\sim$4x larger than pro-DEM effect
  \item No evidence of turnout suppression
  \item House effects are present and significant, but smaller in magnitude --- consistent with the more candidate-driven nature of House races
\end{itemize}

\end{frame}

%----------------------------------------------------------------------
\begin{frame}{Effect Across the Partisan Spectrum}

\begin{table}
\centering
\small
\begin{tabular}{lccc}
\toprule
& \multicolumn{3}{c}{\textit{DEM Vote Share}} \\
\cmidrule(lr){2-4}
& R-Leaning & Swing & D-Leaning \\
\midrule
Smoke PM\textsubscript{2.5} (60d) & 0.00066*** & 0.00049*** & 0.00082*** \\
& (0.00021) & (0.00014) & (0.00013) \\[6pt]
$N$ & 4,144 & 4,141 & 4,143 \\
\bottomrule
\end{tabular}
\end{table}

\vspace{0.3cm}
\begin{itemize}
  \item Effect is \textbf{present in all terciles} of prior partisanship
  \item Somewhat larger in D-leaning counties
  \item Contrast with \citet{hazlett2020wildfire}: fire proximity affects \textit{only} Democratic areas
  \item Smoke is a broader, less politically sorted treatment
\end{itemize}

\end{frame}

%----------------------------------------------------------------------
\begin{frame}{Temporal Dynamics}

\begin{columns}[T]
\begin{column}{0.55\textwidth}
\begin{figure}
\centering
\includegraphics[width=\textwidth]{event_study_windows.png}
\end{figure}
\end{column}
\begin{column}{0.42\textwidth}
\vspace{1cm}
\begin{itemize}
  \item Effect significant at all windows
  \item Strongest at 30 days
  \item Consistent with recency / salience mechanism
  \item Not just election-day disruption
\end{itemize}
\end{column}
\end{columns}

\end{frame}

%----------------------------------------------------------------------
\begin{frame}{Binscatter: Smoke and Democratic Vote Share}

\begin{figure}
\centering
\includegraphics[width=0.65\textwidth]{binscatter_smoke_dem_share.png}
\end{figure}

\vspace{-0.2cm}
{\small County and year FE residualized. 50 equal-sized bins of smoke exposure.}

\end{frame}

%----------------------------------------------------------------------
\begin{frame}{Robustness: Excluding 2020}

\begin{table}
\centering
\small
\begin{tabular}{lcc}
\toprule
& Full Sample & Excluding 2020 \\
\midrule
\multicolumn{3}{l}{\textit{Panel A: DEM Vote Share}} \\[3pt]
Smoke PM\textsubscript{2.5} (60d) & 0.00087*** & $-$0.00041 \\
& (0.00009) & (0.00025) \\[6pt]
\multicolumn{3}{l}{\textit{Panel B: Incumbent Vote Share}} \\[3pt]
Smoke PM\textsubscript{2.5} (60d) & $-$0.00399*** & $-$0.01446*** \\
& (0.00044) & (0.00173) \\[6pt]
\multicolumn{3}{l}{\textit{Panel C: Log Total Votes}} \\[3pt]
Smoke PM\textsubscript{2.5} (60d) & 0.00242*** & 0.00108*** \\
& (0.00018) & (0.00028) \\[3pt]
\midrule
$N$ & 12,429 & 9,321 \\
\bottomrule
\multicolumn{3}{l}{\footnotesize Presidential elections. County and year FE. SEs clustered by county.}
\end{tabular}
\end{table}

\end{frame}

%----------------------------------------------------------------------
\begin{frame}{Interpreting the 2020 Sensitivity}

\begin{itemize}
  \item The pro-DEM effect is not robust to excluding 2020 --- the extreme Western fire season provides much of the identifying variation
  \item The \textbf{anti-incumbent effect is robust and strengthens} without 2020 ($-$0.014 vs.\ $-$0.004), suggesting it is not an artifact of that single year
  \item The turnout effect attenuates but remains significant
  \item Incumbent punishment is the most robust finding across specifications
\end{itemize}

\end{frame}

%----------------------------------------------------------------------
\begin{frame}{Limitations and Next Steps}

\textbf{Current limitations:}
\begin{itemize}
  \item Only 4 presidential elections; 3 House elections (smoke data: 2006--2020)
  \item Pro-DEM shift is leveraged by the 2020 fire season
  \item County-level aggregation; no individual-level variation
  \item Turnout measure is crude (log total votes without population denominator)
\end{itemize}

\vspace{0.3cm}
\textbf{Planned extensions:}
\begin{itemize}
  \item NOAA HMS smoke plumes for extended coverage through 2024
  \item State legislative elections
  \item Wind direction as instrument for smoke exposure
  \item Conley spatial SEs for inference robust to spatial correlation
\end{itemize}

\end{frame}

%----------------------------------------------------------------------
\begin{frame}{Summary}

\begin{enumerate}
  \item Wildfire smoke \textbf{punishes incumbents} --- the most robust finding
  \item Smoke also \textbf{shifts votes toward Democrats}, but this is driven by the 2020 fire season
  \item Effects are \textbf{nationally representative} and \textbf{cross the partisan spectrum}
  \item Smoke is \textbf{plausibly exogenous} (wind-driven) and affects \textbf{far more people} than fire proximity
\end{enumerate}

\end{frame}

%----------------------------------------------------------------------
\appendix
\begin{frame}[allowframebreaks]{References}
\small
\bibliography{references}
\end{frame}

\end{document}
