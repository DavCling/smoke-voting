\documentclass[aspectratio=169,12pt]{beamer}
\usetheme{metropolis}
\usepackage{booktabs}
\usepackage{graphicx}
\usepackage{natbib}
\usepackage{appendixnumberbeamer}
\usepackage{tikz}

\graphicspath{{../output/figures/}}
\bibliographystyle{apalike}

% Smaller footnote citations
\renewcommand{\footnotesize}{\scriptsize}

\title{Wildfire Smoke and Voter Turnout\\in the United States}
\author{David Clingingsmith\\\small Case Western Reserve University}
\date{February 2026}

\begin{document}

\maketitle

%======================================================================
% PART 1: INTRODUCTION & MOTIVATION
%======================================================================

%----------------------------------------------------------------------
\begin{frame}{Motivation}

\begin{itemize}
  \item Wildfire smoke is the most \textit{widespread experiential} consequence of climate change in the U.S.
  \begin{itemize}
    \item Tens of millions of Americans experience unhealthy air from wildfire smoke each year
    \item Unlike ambient air pollution: episodic, \textbf{visible}, and directly attributable
    \item Hazy skies, orange sunsets, the smell of burning --- sensory channels that make smoke \textbf{salient}
  \end{itemize}
  \vspace{0.3cm}
  \item Does pre-election smoke exposure affect \textbf{voter turnout}? Vote choice?
  \vspace{0.3cm}
  \item Key advantage: smoke plume direction is determined by \textbf{wind}, not by local politics or demographics $\Rightarrow$ plausibly exogenous
\end{itemize}

\end{frame}

%----------------------------------------------------------------------
\begin{frame}{Preview of Findings}

\begin{enumerate}
  \item[\textbf{1.}] \textbf{Turnout increases} with smoke exposure
  \begin{itemize}
    \item Survives TWFE $\rightarrow$ controls $\rightarrow$ state trends
    \item Present in both presidential and House elections
    \item Not driven by 2020
  \end{itemize}
  \vspace{0.2cm}
  \item[\textbf{2.}] \textbf{Vote choice effects are suggestive but fragile}
  \begin{itemize}
    \item Pro-Democratic shift: driven by 2020 fire season
    \item Incumbent punishment: significant only at close-in (1--7 day) windows
  \end{itemize}
  \vspace{0.2cm}
  \item[\textbf{3.}] \textbf{State$\times$year FE eliminate all effects}
  \begin{itemize}
    \item Absorb 41\% of identifying variation
    \item The key limitation --- discussed honestly in the paper
  \end{itemize}
\end{enumerate}

\end{frame}

%----------------------------------------------------------------------
\begin{frame}{Contribution}

\begin{itemize}
  \item \textbf{Relative to \citet{hazlett2020wildfire}:}
  \begin{itemize}
    \item Smoke exposure (not fire proximity) $\Rightarrow$ plausibly exogenous
    \item National scope (not California only)
    \item \textit{Turnout}, not just vote choice on environmental ballot measures
  \end{itemize}
  \vspace{0.2cm}
  \item \textbf{Relative to \citet{bellani2024air}:}
  \begin{itemize}
    \item Visible smoke (not imperceptible PM\textsubscript{10})
    \item They find \textit{no} turnout effect from election-day pollution
    \item I find a \textit{positive} turnout effect from pre-election smoke exposure
  \end{itemize}
  \vspace{0.2cm}
  \item \textbf{Relative to \citet{gomez2007weather}:}
  \begin{itemize}
    \item Rain on election day \textit{suppresses} turnout (cost of voting)
    \item Pre-election smoke \textit{increases} turnout (salience/mobilization)
  \end{itemize}
  \vspace{0.2cm}
  \item \textbf{First paper} linking wildfire-specific smoke to U.S.\ electoral outcomes
\end{itemize}

\end{frame}

%======================================================================
% PART 2: DATA
%======================================================================

%----------------------------------------------------------------------
\begin{frame}{Data: Wildfire Smoke PM\textsubscript{2.5}}

\textbf{Source:} \citet{childs2022daily} --- daily county-level estimates, 2006--2023

\vspace{0.3cm}
\begin{itemize}
  \item NOAA Hazard Mapping System satellite smoke plume classifications + machine learning
  \item Isolates \textit{wildfire-attributed} PM\textsubscript{2.5} from background pollution
  \item Treatment: \textbf{mean smoke PM\textsubscript{2.5} in the 30 days before election}
\end{itemize}

\vspace{0.3cm}
\textbf{Second treatment variable:}
\begin{itemize}
  \item Fraction of days with smoke PM\textsubscript{2.5} $>$ 20 $\mu$g/m$^3$ (onset of visible haze)
  \item Captures the extensive margin of \textit{perceptible} smoke episodes
  \item 1.9\% of county-elections have $\geq$1 haze day (236 of 12,432)
\end{itemize}

\end{frame}

%----------------------------------------------------------------------
\begin{frame}{Data: Elections and Turnout Rate}

\textbf{Election returns:} MIT Election Data + Science Lab \citep{medsl2024county}
\begin{itemize}
  \item Presidential: county-level, 2008, 2012, 2016, 2020
  \item House: precinct-level $\rightarrow$ aggregated to county, 2016--2022
\end{itemize}

\vspace{0.2cm}
\textbf{Voting age population (VAP):} ACS 5-year estimates (B01001 table)
\begin{itemize}
  \item \textbf{Turnout rate} = total votes / VAP; drop if $>$ 1.5 (allocation artifacts)
\end{itemize}

\vspace{0.2cm}
\textbf{Analysis samples:}
\begin{itemize}
  \item Presidential: 12,432 county $\times$ election obs (4 cycles, 3,108 counties)
  \item House: 12,206 county $\times$ election obs (4 cycles); VAP for 2016, 2020 only
\end{itemize}

\end{frame}

%----------------------------------------------------------------------
\begin{frame}{Smoke Exposure Varies Dramatically Across Elections}

\begin{figure}
\centering
\includegraphics[width=\textwidth]{smoke_exposure_map_panel.png}
\end{figure}

\end{frame}

%----------------------------------------------------------------------
\begin{frame}{Key Variation: 2020 Was Exceptional}

\begin{itemize}
  \item 2016: Minimal pre-election smoke nationwide
  \item 2020: Historic August--September fire season $\Rightarrow$ extreme Western exposure
  \item This matters: the pro-Democratic effect is \textit{driven} by 2020
  \item The turnout effect is \textit{not} driven by 2020 (actually strengthens without it)
\end{itemize}

\vspace{0.5cm}
\textbf{Implication:} Need to carefully assess which results are robust to excluding 2020

\end{frame}

%======================================================================
% PART 3: EMPIRICAL STRATEGY
%======================================================================

%----------------------------------------------------------------------
\begin{frame}{Empirical Strategy}

Two-way fixed effects:
\begin{equation*}
Y_{ct} = \alpha_c + \gamma_t + \beta \cdot \text{SmokePM}_{ct} + \varepsilon_{ct}
\end{equation*}

\begin{itemize}
  \item $\alpha_c$: County FE --- absorb all time-invariant county characteristics
  \item $\gamma_t$: Election year FE --- absorb national swings
  \item SEs clustered by county
  \item Treatment: mean smoke PM\textsubscript{2.5} in the 30 days before election
\end{itemize}

\vspace{0.3cm}
\textbf{Outcomes:}
\begin{enumerate}
  \item \textbf{Turnout rate} (total votes / VAP) --- \textit{primary}
  \item Log total votes --- corroboration (no denominator needed)
  \item DEM two-party vote share --- \textit{secondary}
  \item Incumbent party vote share --- \textit{secondary}
\end{enumerate}

\end{frame}

%----------------------------------------------------------------------
\begin{frame}{Identifying Assumption and Threats}

\textbf{Identifying assumption:} Conditional on county and year FE, variation in smoke is uncorrelated with unobserved determinants of voting

\vspace{0.3cm}
\textbf{Why plausible:} Smoke plume direction and dispersion are determined by atmospheric conditions (wind), not by county politics or demographics

\vspace{0.5cm}
\textbf{Potential threats:}
\begin{itemize}
  \item \textbf{Spatially correlated shocks} (drought $\rightarrow$ fires + local economy)
  \begin{itemize}
    \item Mitigated: smoke travels hundreds of miles from fire origin
  \end{itemize}
  \item \textbf{Region-specific trends} (fire-prone vs.\ non-fire-prone states)
  \begin{itemize}
    \item Addressed in build-up: state linear trends (Spec 4) and state$\times$year FE
  \end{itemize}
  \item \textbf{TWFE with continuous treatment} \citep{callaway2024continuous}
  \begin{itemize}
    \item Treatment is atmospherically assigned $\Rightarrow$ limits selection into dose levels
    \item Results robust to dichotomization and dose quintiles
  \end{itemize}
\end{itemize}

\end{frame}

%----------------------------------------------------------------------
\begin{frame}{Build-Up Strategy}

Progressively more demanding specifications:

\vspace{0.3cm}
\begin{enumerate}
  \item \textbf{Raw OLS} --- cross-sectional correlation (benchmark)
  \item \textbf{TWFE} --- county + year FE (baseline)
  \item \textbf{+ Controls} --- unemployment, income, population, October weather
  \item \textbf{+ State trends} --- state-specific linear time trends
\end{enumerate}

\vspace{0.5cm}
\textbf{Logic:} If the effect survives progressively demanding controls, it is less likely to reflect confounding

\vspace{0.3cm}
\textbf{Key question:} Which outcomes survive all four specifications?

\end{frame}

%======================================================================
% PART 4: MAIN RESULTS - TURNOUT
%======================================================================

%----------------------------------------------------------------------
\begin{frame}{Build-Up Table: All Four Outcomes}

\begin{table}
\centering
\scriptsize
\begin{tabular}{lcccc}
\toprule
& (1) Raw OLS & (2) TWFE & (3) +Controls & (4) +St.\ Trends \\
\midrule
\multicolumn{5}{l}{\textit{Panel A: Turnout Rate}} \\[2pt]
Smoke PM\textsubscript{2.5} (30d) & 0.00536*** & 0.00100*** & 0.00085*** & \alert{0.00065***} \\
& (0.00068) & (0.00020) & (0.00019) & (0.00016) \\[4pt]
\multicolumn{5}{l}{\textit{Panel B: Log Total Votes}} \\[2pt]
Smoke PM\textsubscript{2.5} (30d) & $-$0.02911*** & 0.00314*** & 0.00245*** & \alert{0.00124***} \\
& (0.00504) & (0.00052) & (0.00051) & (0.00024) \\[4pt]
\multicolumn{5}{l}{\textit{Panel C: DEM Vote Share}} \\[2pt]
Smoke PM\textsubscript{2.5} (30d) & $-$0.00831*** & 0.00135*** & 0.00125*** & $-$0.00016 \\
& (0.00132) & (0.00021) & (0.00023) & (0.00021) \\[4pt]
\multicolumn{5}{l}{\textit{Panel D: Incumbent Vote Share}} \\[2pt]
Smoke PM\textsubscript{2.5} (30d) & 0.02028*** & $-$0.00172* & 0.00070 & 0.00080 \\
& (0.00259) & (0.00091) & (0.00094) & (0.00108) \\
\midrule
\multicolumn{5}{l}{\footnotesize $N \approx 12{,}400$. County + year FE. SEs clustered by county. Controls: unemployment, income, pop., weather.}
\end{tabular}
\end{table}

\end{frame}

%----------------------------------------------------------------------
\begin{frame}{Turnout is the Primary Finding}

\begin{itemize}
  \item \textbf{Turnout rate:} positive and significant across \textit{all four specifications}
  \begin{itemize}
    \item TWFE: 0.00100*** $\rightarrow$ +Controls: 0.00085*** $\rightarrow$ +State trends: 0.00065***
    \item Raw OLS is also positive --- the sign doesn't flip when adding FE
  \end{itemize}
  \vspace{0.3cm}
  \item \textbf{Log total votes} corroborates (no VAP denominator needed)
  \begin{itemize}
    \item State trends estimate (0.00124***) is $\sim$1/3 of TWFE baseline
  \end{itemize}
  \vspace{0.3cm}
  \item \textbf{DEM vote share:} killed by state trends
  \item \textbf{Incumbent:} killed by controls
  \vspace{0.3cm}
  \item[$\Rightarrow$] \textbf{Turnout is the only outcome that survives all specifications}
\end{itemize}

\end{frame}

%----------------------------------------------------------------------
\begin{frame}{Magnitude}

Under the preferred specification (TWFE + controls):

\vspace{0.3cm}
\begin{itemize}
  \item A \textbf{1 $\mu$g/m$^3$} increase in 30-day mean smoke PM\textsubscript{2.5}:\\[3pt]
  $\Rightarrow$ +0.085 percentage point increase in turnout rate
  \vspace{0.3cm}
  \item A \textbf{10 $\mu$g/m$^3$} increase (roughly a major smoke event):\\[3pt]
  $\Rightarrow$ +0.85 percentage point increase in turnout rate
  \vspace{0.3cm}
  \item For context: U.S.\ presidential turnout rate $\approx$ 55--65\%\\[3pt]
  $\Rightarrow$ 0.85 pp is about a 1.3--1.5\% relative increase
\end{itemize}

\vspace{0.5cm}
\textbf{The positive sign rules out a suppression/disruption mechanism}

\end{frame}

%----------------------------------------------------------------------
\begin{frame}{Temporal Dynamics: Cumulative Windows}

\begin{figure}
\centering
\includegraphics[width=\textwidth,height=0.78\textheight,keepaspectratio]{temporal_cumulative_controls.png}
\end{figure}

\end{frame}

%----------------------------------------------------------------------
\begin{frame}{Interpreting Temporal Dynamics}

\begin{itemize}
  \item \textbf{Turnout rate} (top row): positive and significant at \textit{all} window lengths
  \begin{itemize}
    \item Stable from shortest windows through full fire season
    \item Consistent with persistent mobilization effect
  \end{itemize}
  \vspace{0.2cm}
  \item \textbf{DEM vote share:} builds over 14--35 days, stabilizes at 30d
  \begin{itemize}
    \item Recent smoke drives the effect $\Rightarrow$ salience/recency
  \end{itemize}
  \vspace{0.2cm}
  \item \textbf{Incumbent:} concentrated in first 14 days, then washes out
  \begin{itemize}
    \item Suggests short-lived negative-affect mechanism
  \end{itemize}
  \vspace{0.3cm}
  \item[$\Rightarrow$] Motivates the 30-day window: short enough for recency signal, long enough to smooth noise
\end{itemize}

\end{frame}

%----------------------------------------------------------------------
\begin{frame}{Close-In Daily Dynamics (1--7 Days)}

\begin{figure}
\centering
\includegraphics[width=\textwidth,height=0.78\textheight,keepaspectratio]{temporal_closein_daily.png}
\end{figure}

\end{frame}

%----------------------------------------------------------------------
\begin{frame}{Close-In: Mechanistic Evidence}

\begin{itemize}
  \item \textbf{Incumbent punishment builds monotonically:} 0 at 1 day $\rightarrow$ $-$0.020*** at 7 days
  \begin{itemize}
    \item Most mechanistically compelling result in the paper
    \item Consistent with recency-driven blame attribution
  \end{itemize}
  \vspace{0.2cm}
  \item \textbf{Turnout rate} positive and marginally significant from day 1
  \begin{itemize}
    \item Grows steadily to 0.118 ($p < 0.10$) at 7 days
  \end{itemize}
  \vspace{0.2cm}
  \item \textbf{Log total votes} significant from day 1 ($p < 0.01$), strengthens monotonically
  \vspace{0.2cm}
  \item[$\Rightarrow$] Different mechanisms at different time horizons
\end{itemize}

\end{frame}

%----------------------------------------------------------------------
\begin{frame}{House Elections: Turnout is Robust Across Race Types}

\begin{table}
\centering
\scriptsize
\begin{tabular}{lcc}
\toprule
& County House & Presidential \\
& (2016--2022) & (2008--2020) \\
\midrule
\multicolumn{3}{l}{\textit{Panel A: Turnout Rate}} \\[3pt]
Smoke PM\textsubscript{2.5} (30d) & \alert{0.00173**} & \alert{0.00100***} \\
& (0.00076) & (0.00020) \\[4pt]
\multicolumn{3}{l}{\textit{Panel B: Log Total Votes}} \\[3pt]
Smoke PM\textsubscript{2.5} (30d) & 0.00356*** & 0.00314*** \\
& (0.00119) & (0.00052) \\[4pt]
\multicolumn{3}{l}{\textit{Panel C: DEM Vote Share}} \\[3pt]
Smoke PM\textsubscript{2.5} (30d) & $-$0.00033 & 0.00135*** \\
& (0.00029) & (0.00021) \\[4pt]
\multicolumn{3}{l}{\textit{Panel D: Incumbent Vote Share}} \\[3pt]
Smoke PM\textsubscript{2.5} (30d) & 0.00304* & $-$0.00172* \\
& (0.00178) & (0.00091) \\
\midrule
\multicolumn{3}{l}{\footnotesize County + year FE. SEs clustered by county. House VAP: 2016, 2020 only ($N=6{,}088$).}
\end{tabular}
\end{table}

\end{frame}

%----------------------------------------------------------------------
\begin{frame}{Robustness: Excluding 2020}

\begin{figure}
\centering
\includegraphics[width=\textwidth,height=0.78\textheight,keepaspectratio]{temporal_drop2020.png}
\end{figure}

\end{frame}

%----------------------------------------------------------------------
\begin{frame}{Drop-2020: What We Learn}

\begin{itemize}
  \item \textbf{Turnout:} positive and significant in \textit{both} samples
  \begin{itemize}
    \item Actually \textit{stronger} without 2020 (0.0040*** vs.\ 0.0011**)
    \item Not driven by the extreme fire season
  \end{itemize}
  \vspace{0.3cm}
  \item \textbf{DEM vote share:} \textit{reverses sign} without 2020
  \begin{itemize}
    \item Full sample: +0.0023*** $\rightarrow$ Excl.\ 2020: $-$0.0034***
    \item The pro-Democratic result is leveraged by the 2020 Western fire season
  \end{itemize}
  \vspace{0.3cm}
  \item \textbf{Incumbent punishment:} dramatically \textit{stronger} without 2020
  \begin{itemize}
    \item Full sample: $\approx$0 $\rightarrow$ Excl.\ 2020: $-$0.033***
    \item 2020 \textit{masked} a strong underlying punishment effect
  \end{itemize}
\end{itemize}

\end{frame}

%----------------------------------------------------------------------
\begin{frame}{Summary So Far}

\begin{table}
\centering
\small
\begin{tabular}{lcccc}
\toprule
& TWFE & +Controls & +Trends & Excl.\ 2020 \\
\midrule
Turnout rate & \alert{\checkmark ***} & \alert{\checkmark ***} & \alert{\checkmark ***} & \alert{\checkmark ***} \\
Log total votes & \checkmark *** & \checkmark *** & \checkmark *** & \checkmark *** \\
DEM vote share & \checkmark *** & \checkmark *** & $\times$ & reverses \\
Incumbent & \checkmark * & $\times$ & $\times$ & \checkmark *** \\
\bottomrule
\end{tabular}
\end{table}

\vspace{0.3cm}
\begin{itemize}
  \item \textbf{Turnout is the only outcome that passes every test}
  \item DEM vote share is fragile; incumbent only close-in or excl.\ 2020
  \item But: none of this addresses state$\times$year FE\ldots
\end{itemize}

\end{frame}

%======================================================================
% PART 5: STATE×YEAR FE
%======================================================================

%----------------------------------------------------------------------
\begin{frame}{The Key Limitation: State-by-Year Fixed Effects}

\textbf{The concern:} State-level time-varying shocks (campaigns, ballot measures, economic conditions) could be correlated with smoke exposure

\vspace{0.3cm}
\textbf{The test:} Replace year FE with state$\times$year FE
\begin{itemize}
  \item All identification from \textit{within-state} variation across counties in the same election
  \item Substantially more demanding than state trends
\end{itemize}

\vspace{0.5cm}
\textbf{The result:} \alert{Everything goes to zero}

\end{frame}

%----------------------------------------------------------------------
\begin{frame}{State$\times$Year FE: The Numbers}

\begin{table}
\centering
\scriptsize
\begin{tabular}{lcc}
\toprule
& Presidential & County House \\
\midrule
\multicolumn{3}{l}{\textit{Turnout Rate}} \\[3pt]
Smoke PM\textsubscript{2.5} (30d) & $-$0.00005 & 0.00060 \\
& (0.00018) & (0.00092) \\[4pt]
\multicolumn{3}{l}{\textit{Log Total Votes}} \\[3pt]
Smoke PM\textsubscript{2.5} (30d) & 0.00023 & 0.00161 \\
& (0.00030) & (0.00135) \\[4pt]
\multicolumn{3}{l}{\textit{DEM Vote Share}} \\[3pt]
Smoke PM\textsubscript{2.5} (30d) & 0.00001 & 0.00011 \\
& (0.00025) & (0.00025) \\[4pt]
\multicolumn{3}{l}{\textit{Incumbent Vote Share}} \\[3pt]
Smoke PM\textsubscript{2.5} (30d) & $-$0.00079 & 0.00214 \\
& (0.00086) & (0.00140) \\
\midrule
\multicolumn{3}{l}{\footnotesize \textbf{No coefficient is significant at the 10\% level.}}
\end{tabular}
\end{table}

\end{frame}

%----------------------------------------------------------------------
\begin{frame}{Where Does the Identifying Variation Go?}

\begin{figure}
\centering
\includegraphics[width=\textwidth,height=0.72\textheight,keepaspectratio]{smoke_exposure_map_residualized.png}
\end{figure}

{\small After removing state$\times$year means, within-state variation is concentrated in a few Western states in high-fire years}

\end{frame}

%----------------------------------------------------------------------
\begin{frame}{Temporal Overlay: Spec 3 vs.\ State$\times$Year FE}

\begin{figure}
\centering
\includegraphics[width=\textwidth,height=0.78\textheight,keepaspectratio]{temporal_cumulative_stateyear.png}
\end{figure}

\end{frame}

%----------------------------------------------------------------------
\begin{frame}{Two Interpretations}

\textbf{Interpretation (a): State$\times$year FE are too demanding}
\begin{itemize}
  \item They absorb \textbf{41\%} of the identifying variation
  \item Between-state smoke variation is wind-driven $\Rightarrow$ genuinely exogenous
  \item \citet{hilbig2024floods}: German floods had no \textit{local} effect but a \textit{national} Green shift --- environmental shocks may operate through diffuse channels
  \item With only 4 elections, remaining within-state variation may be too thin
\end{itemize}

\vspace{0.3cm}
\textbf{Interpretation (b): Between-state variation is confounded}
\begin{itemize}
  \item State-level shocks (drought, campaigns) correlate with smoke
  \item Baseline estimates may be partly driven by omitted state$\times$year factors
\end{itemize}

\vspace{0.3cm}
\textbf{I lean toward (a)} but \textbf{cannot rule out (b)} with 4 elections

\end{frame}

%----------------------------------------------------------------------
\begin{frame}{Evidence Favoring Exogeneity}

\begin{itemize}
  \item Atmospheric dispersion is physically independent of local political/economic conditions
  \vspace{0.2cm}
  \item \citet{childs2022daily} isolates wildfire-specific PM\textsubscript{2.5} from other sources
  \vspace{0.2cm}
  \item \citet{borgschulte2022air}: controlling for wind direction does not change smoke-based labor market estimates
  \vspace{0.2cm}
  \item Turnout effect survives state \textit{linear} trends (which absorb differential secular trends)
  \vspace{0.2cm}
  \item The 2024 election --- with Canadian wildfire smoke affecting the \textit{Eastern} U.S.\ --- would provide geographically novel variation and substantially more within-state power
\end{itemize}

\end{frame}

%======================================================================
% PART 6: SECONDARY RESULTS
%======================================================================

%----------------------------------------------------------------------
\begin{frame}{Threshold Comparison: Dose-Response}

\begin{table}
\centering
\scriptsize
\begin{tabular}{lccc}
\toprule
& Haze ($>$20) & USG ($>$35.5) & Unhealthy ($>$55.5) \\
\midrule
\multicolumn{4}{l}{\textit{Turnout Rate}} \\[2pt]
Frac.\ days above & 0.04497*** & $-$0.00065 & 0.04909** \\
& (0.01427) & (0.03037) & (0.02279) \\[3pt]
\multicolumn{4}{l}{\textit{DEM Vote Share}} \\[2pt]
Frac.\ days above & 0.06162*** & 0.09929** & 0.15047*** \\
& (0.01611) & (0.04410) & (0.03746) \\[3pt]
\multicolumn{4}{l}{\textit{Incumbent Vote Share}} \\[2pt]
Frac.\ days above & 0.16253** & $-$0.16716 & $-$0.24933*** \\
& (0.06755) & (0.10465) & (0.07825) \\
\midrule
Nonzero obs & 236 (1.9\%) & 47 (0.4\%) & 19 (0.2\%) \\
\bottomrule
\multicolumn{4}{l}{\footnotesize TWFE + controls. SEs clustered by county.}
\end{tabular}
\end{table}

\vspace{0.2cm}
\begin{itemize}
  \item DEM: dose-response pattern (larger effects at higher thresholds)
  \item Incumbent: \textbf{sign reversal} --- moderate smoke benefits, severe smoke punishes
\end{itemize}

\end{frame}

%----------------------------------------------------------------------
\begin{frame}{Heterogeneity: Effects Cross the Partisan Spectrum}

\begin{table}
\centering
\small
\begin{tabular}{lccc}
\toprule
& R-Leaning & Swing & D-Leaning \\
\midrule
\multicolumn{4}{l}{\textit{DEM Vote Share (TWFE)}} \\[3pt]
Smoke PM\textsubscript{2.5} (30d) & 0.00155*** & 0.00007 & 0.00126*** \\
& (0.00033) & (0.00055) & (0.00040) \\[4pt]
$N$ & 4,146 & 4,144 & 4,147 \\
\bottomrule
\end{tabular}
\end{table}

\vspace{0.3cm}
\begin{itemize}
  \item Pro-DEM shift in \textit{both} R-leaning and D-leaning counties
  \item Absent in swing counties
  \item \textbf{Contrast with \citet{hazlett2020wildfire}:} fire proximity $\rightarrow$ effects \textit{only} in Democratic areas
  \item Smoke is a broader, less politically sorted treatment
\end{itemize}

\end{frame}

%======================================================================
% PART 7: MECHANISMS & DISCUSSION
%======================================================================

%----------------------------------------------------------------------
\begin{frame}{Mechanism: Mobilization, Not Suppression}

\begin{itemize}
  \item \textbf{The positive turnout sign is informative:}
  \begin{itemize}
    \item Rules out \textit{disruption} (smoke makes it harder to vote)
    \item Rules out \textit{differential suppression} (smoke discourages one party)
    \item Points toward \textit{mobilization} through salience
  \end{itemize}
  \vspace{0.3cm}
  \item \textbf{Consistent with:}
  \begin{itemize}
    \item \citet{jusko2024motivated}: severe flooding \textit{increases} turnout in Slovakia
    \item \citet{burke2021exposures}: smoke events increase air quality searches, behavioral responses --- people are \textit{paying attention}
  \end{itemize}
  \vspace{0.3cm}
  \item \textbf{Contrasts with:}
  \begin{itemize}
    \item \citet{gomez2007weather}: rain \textit{suppresses} turnout (cost of voting channel)
    \item \citet{bellani2024air}: PM\textsubscript{10} has \textit{no} turnout effect (subconscious mood only)
  \end{itemize}
\end{itemize}

\end{frame}

%----------------------------------------------------------------------
\begin{frame}{Salience $\neq$ Preference Change}

\textbf{Key insight from \citet{andrews2025wildfire}:}
\begin{itemize}
  \item Wildfire experience increases climate change \textit{belief} (even among Republicans)
  \item But does \textit{not} increase willingness to \textit{spend} on mitigation
\end{itemize}

\vspace{0.2cm}
\textbf{This maps onto my results:}
\begin{itemize}
  \item \textbf{Turnout} (engagement/attention): \alert{robust}
  \item \textbf{DEM vote share} (preference): fragile, driven by 2020
  \item Smoke makes voters more \textit{attentive and motivated} without systematically shifting policy views
\end{itemize}

\vspace{0.2cm}
\textbf{Temporal decay:} \citet{arias2024hurricane} find hurricane-driven attitude shifts last $\sim$1 month --- consistent with effects at 7--30 day windows

\end{frame}

%----------------------------------------------------------------------
\begin{frame}{The Incumbent Puzzle}

\begin{itemize}
  \item \textbf{At 30 days:} No significant effect (washes out with controls)
  \item \textbf{At 1--7 days:} Strong monotonic anti-incumbent effect
  \item \textbf{Without 2020:} Large anti-incumbent effect at all windows
  \vspace{0.2cm}
  \item \textbf{Interpretation:}
  \begin{itemize}
    \item Close-in punishment $\Rightarrow$ short-lived negative-affect channel \citep{bellani2024air,du2024smoke}
    \item Cross-national replication: \citet{kronborg2024wildfires} find incumbent punishment from Sweden's 2018 wildfires
    \item 2020 masked the effect: smoke-exposed Western counties favored incumbent
  \end{itemize}
  \vspace{0.2cm}
  \item \textbf{Threshold nuance:} moderate smoke $\rightarrow$ pro-incumbent; severe $\rightarrow$ anti-incumbent
\end{itemize}

\end{frame}

%----------------------------------------------------------------------
\begin{frame}{Limitations}

\begin{itemize}
  \item \textbf{State$\times$year FE kill everything} --- the most important caveat
  \begin{itemize}
    \item Cannot determine if this is overly demanding or reveals confounding
    \item 41\% of variation is between-state$\times$year
  \end{itemize}
  \vspace{0.2cm}
  \item \textbf{Small sample:} 4 presidential elections, 4 House elections
  \vspace{0.2cm}
  \item \textbf{Pro-DEM effect driven by 2020} fire season
  \vspace{0.2cm}
  \item \textbf{County-level aggregation} may mask within-county heterogeneity
  \vspace{0.2cm}
  \item \textbf{Small within-$R^2$:} smoke explains limited variation after absorbing FE
  \vspace{0.2cm}
  \item \textbf{Mechanism:} I observe turnout increases but cannot directly test the salience channel (no survey data, no search data matched to counties)
\end{itemize}

\end{frame}

%----------------------------------------------------------------------
\begin{frame}{Summary}

\begin{enumerate}
  \item Wildfire smoke \textbf{increases voter turnout} --- the most robust finding
  \begin{itemize}
    \item Survives controls, state trends, excluding 2020
    \item Present in presidential and House elections
    \item Positive sign $\Rightarrow$ mobilization, not suppression
  \end{itemize}
  \vspace{0.2cm}
  \item \textbf{Vote choice effects are suggestive:} pro-DEM (driven by 2020), close-in incumbent punishment (recency mechanism), dose-response across thresholds
  \vspace{0.2cm}
  \item \textbf{State$\times$year FE produce null results} --- the key limitation
  \begin{itemize}
    \item Absorbs 41\% of variation; cannot resolve with 4 elections
    \item More election cycles (especially 2024) would help
  \end{itemize}
  \vspace{0.2cm}
  \item Wildfire smoke is a \textbf{promising research design}: plausibly exogenous, broadly experienced, and captures the salience channel that ambient pollution cannot
\end{enumerate}

\end{frame}

%----------------------------------------------------------------------
\begin{frame}[standout]

Thank you

\vspace{0.5cm}
{\normalsize \texttt{d.clingingsmith@case.edu}}

\end{frame}

%======================================================================
% APPENDIX
%======================================================================

\appendix

%----------------------------------------------------------------------
\begin{frame}{Appendix Slides}
\end{frame}

%----------------------------------------------------------------------
\begin{frame}{Controls Robustness}

\begin{table}
\centering
\scriptsize
\begin{tabular}{lcccc}
\toprule
& \multicolumn{2}{c}{Presidential} & \multicolumn{2}{c}{County House} \\
\cmidrule(lr){2-3} \cmidrule(lr){4-5}
& Baseline & + Controls & Baseline & + Controls \\
\midrule
\multicolumn{5}{l}{\textit{Turnout Rate}} \\[2pt]
Smoke PM\textsubscript{2.5} (30d) & 0.00100*** & 0.00085*** & 0.00173** & 0.00126* \\
& (0.00020) & (0.00019) & (0.00076) & (0.00072) \\[3pt]
\multicolumn{5}{l}{\textit{Log Total Votes}} \\[2pt]
Smoke PM\textsubscript{2.5} (30d) & 0.00314*** & 0.00245*** & 0.00356*** & 0.00323*** \\
& (0.00052) & (0.00051) & (0.00119) & (0.00114) \\[3pt]
\multicolumn{5}{l}{\textit{DEM Vote Share}} \\[2pt]
Smoke PM\textsubscript{2.5} (30d) & 0.00135*** & 0.00125*** & $-$0.00033 & $-$0.00043 \\
& (0.00021) & (0.00023) & (0.00029) & (0.00028) \\[3pt]
\multicolumn{5}{l}{\textit{Incumbent Vote Share}} \\[2pt]
Smoke PM\textsubscript{2.5} (30d) & $-$0.00172* & 0.00070 & 0.00304* & 0.00375** \\
& (0.00091) & (0.00094) & (0.00178) & (0.00151) \\
\midrule
\multicolumn{5}{l}{\footnotesize Controls: unemployment rate, log median income, log population, Oct.\ temperature, Oct.\ precipitation.}
\end{tabular}
\end{table}

\end{frame}

%----------------------------------------------------------------------
\begin{frame}{Exclusive 7-Day Windows: Mean Smoke PM\textsubscript{2.5}}

\begin{figure}
\centering
\includegraphics[width=\textwidth,height=0.78\textheight,keepaspectratio]{temporal_7day_mean_controls.png}
\end{figure}

\end{frame}

%----------------------------------------------------------------------
\begin{frame}{Exclusive 7-Day Windows: Fraction Haze Days}

\begin{figure}
\centering
\includegraphics[width=\textwidth,height=0.78\textheight,keepaspectratio]{temporal_7day_frac_controls.png}
\end{figure}

\end{frame}

%----------------------------------------------------------------------
\begin{frame}{District-Level House Estimates}

\begin{table}
\centering
\scriptsize
\begin{tabular}{lcc}
\toprule
& District House & County House \\
\midrule
\multicolumn{3}{l}{\textit{DEM Vote Share}} \\[3pt]
Smoke PM\textsubscript{2.5} (30d) & $-$0.00081 & $-$0.00033 \\
& (0.00086) & (0.00029) \\[4pt]
\multicolumn{3}{l}{\textit{Incumbent Vote Share}} \\[3pt]
Smoke PM\textsubscript{2.5} (30d) & $-$0.00162 & 0.00304* \\
& (0.00162) & (0.00178) \\[4pt]
\multicolumn{3}{l}{\textit{Log Total Votes}} \\[3pt]
Smoke PM\textsubscript{2.5} (30d) & $-$0.01022* & 0.00356*** \\
& (0.00565) & (0.00119) \\
\midrule
Unit & District & County \\
FE & District + Year & County + Year \\
$N$ & 3,406 / 3,879 & 11,155 / 12,197 \\
\bottomrule
\multicolumn{3}{l}{\footnotesize District-level uses Census crosswalk. SEs clustered by unit.}
\end{tabular}
\end{table}

\end{frame}

%----------------------------------------------------------------------
\begin{frame}{Binscatter: Smoke and Democratic Vote Share}

\begin{figure}
\centering
\includegraphics[width=0.65\textwidth]{binscatter_smoke_dem_share.png}
\end{figure}

{\small County and year FE residualized. 50 equal-sized bins.}

\end{frame}

%----------------------------------------------------------------------
\begin{frame}[allowframebreaks]{References}
\small
\bibliography{references}
\end{frame}

\end{document}
