\documentclass[12pt]{article}
\usepackage[margin=1in]{geometry}
\usepackage{booktabs}
\usepackage{graphicx}
\usepackage{natbib}
\usepackage{amsmath}
\usepackage{amssymb}
\usepackage{setspace}
\usepackage{caption}
\usepackage[hidelinks]{hyperref}
\graphicspath{{../output/figures/}}

\bibliographystyle{apalike}
\onehalfspacing

\title{Wildfire Smoke and Voter Turnout in the United States\thanks{Preliminary draft. Please do not cite or circulate without permission.}}
\author{David Clingingsmith\\Case Western Reserve University}
\date{February 17, 2026}

\begin{document}
\maketitle

\begin{abstract}
\noindent Does wildfire smoke exposure affect voter participation? I exploit the quasi-random spatial dispersion of wildfire smoke plumes---driven by wind patterns rather than local conditions---to estimate the effect of smoke-derived PM\textsubscript{2.5} on county-level voter turnout and voting patterns. Using daily county-level wildfire smoke PM\textsubscript{2.5} estimates \citep{childs2022daily} merged with election returns across multiple cycles (2008--2020 for presidential, 2016--2022 for House), I find that higher pre-election smoke exposure increases voter turnout. Under the preferred specification with county and year fixed effects plus time-varying controls, a 1~$\mu$g/m$^3$ increase in mean smoke PM\textsubscript{2.5} over the 30 days before the election is associated with a 0.085 percentage point increase in the turnout rate ($p < 0.001$). This effect survives state-specific linear time trends ($p < 0.001$) and is present in both presidential and House elections. Suggestive evidence also points to a pro-Democratic shift in presidential voting and to incumbent punishment, though these effects are less robust: the pro-Democratic shift is driven by the 2020 fire season, and the incumbent effect does not survive the addition of controls. A more demanding specification with state-by-year fixed effects produces null results for all outcomes---an important limitation I discuss in the main text rather than relegating to an appendix. The positive sign of the turnout effect rules out a simple disruption or suppression mechanism and is consistent with smoke increasing the salience of environmental concerns, mobilizing voters to participate.

\medskip
\noindent \textit{JEL:} D72, Q54 \quad \textit{Keywords:} Wildfire smoke, voter turnout, air pollution, climate salience
\end{abstract}

%----------------------------------------------------------------------
\section{Introduction}
%----------------------------------------------------------------------

Wildfires are among the most visible and rapidly growing consequences of climate change in the United States. Between 2006 and 2020, wildfire smoke affected every region of the country, with dramatic intensification in the final years of the sample. Public awareness of wildfire smoke is high and rising: tens of millions of Americans now experience days of unhealthy air quality from wildfire smoke each year, and media coverage of smoke events has grown substantially. Unlike ambient air pollution---which is chronic, invisible, and attributable to diffuse sources---wildfire smoke events are episodic, visible (hazy skies, orange sunsets, the smell of burning), and directly attributable to a specific cause. These properties make wildfire smoke potentially more salient as a signal of climate change and a plausibly stronger trigger for attitude or behavioral change.

This paper bridges several strands of the literature by using \textit{wildfire-specific} smoke PM\textsubscript{2.5} as a treatment variable across the entire continental United States. Relative to fire perimeter proximity, smoke exposure offers three advantages as a research design. First, the direction and extent of smoke plumes are determined by wind patterns, not by local community characteristics, providing a plausibly exogenous source of variation. Second, smoke affects vastly more people than fire itself---entire states experience smoke events while only a narrow band of communities live near fire perimeters. Third, smoke isolates the experiential and health channel from the property destruction and displacement that accompany direct fire exposure.

The primary finding is that wildfire smoke \textit{increases} voter turnout. Under progressively demanding specifications---from baseline two-way fixed effects (TWFE) through the addition of time-varying controls and state-specific linear trends---the turnout effect remains positive and statistically significant. This positive sign rules out a simple disruption or suppression mechanism, in which smoke deters participation as rain does \citep{gomez2007weather}, and instead points toward a mobilization channel: visible smoke may increase the salience of environmental issues, prompting more voters to go to the polls. Suggestive evidence also points to a pro-Democratic shift in presidential voting and to incumbent punishment, but these effects are less robust. The pro-Democratic shift is driven by the extreme 2020 Western fire season; the incumbent effect does not survive the addition of county-level controls. A more demanding specification with state-by-year fixed effects produces null results for all outcomes, including turnout. I discuss this limitation honestly in the main text: 41\% of the identifying variation is between-state-by-year, and with only four presidential elections, the remaining within-state variation may be insufficient for precise estimation---but the possibility of state-level confounding cannot be ruled out.

%----------------------------------------------------------------------
\section{Related Literature}
\label{sec:literature}
%----------------------------------------------------------------------

This paper connects five literatures: environmental shocks and voter turnout, environmental shocks and vote choice, retrospective incumbent punishment, air pollution and political behavior, and the economics of wildfire smoke. Appendix~\ref{sec:litreview} provides a comprehensive review; here I focus on the papers most directly related to this study's contributions and interpretation.

\paragraph{Environmental shocks and voter turnout.} The most directly relevant precedent for the turnout finding is \citet{gomez2007weather}, who show that rain on election day suppresses voter turnout, with differential effects by party. My finding runs in the opposite direction: smoke \textit{increases} turnout. The key difference is timing---I measure smoke exposure over the 30 days \textit{before} the election, not on election day itself. Pre-election smoke plausibly operates through a salience channel, making environmental concerns more vivid and motivating participation, rather than through the logistical cost channel by which election-day rain deters voting. \citet{jusko2024motivated} provides the closest precedent for the mobilization mechanism I propose: severe flooding in Slovakia \textit{increased} voter turnout, with effects concentrated in the most damaged areas, consistent with disaster salience motivating participation. \citet{bellani2024air} study the effect of ambient PM\textsubscript{10} on German elections and find no effect on turnout, attributing their vote-share results to a subconscious emotional channel. The contrast with my finding may reflect the visibility of wildfire smoke: voters can see haze, smell burning, and feel respiratory irritation, which could activate deliberate engagement rather than only subconscious mood shifts.

\paragraph{Wildfires and voting.} The most closely related paper on vote choice is \citet{hazlett2020wildfire}, who use proximity to California wildfire perimeters to study effects on pro-environment ballot proposition voting. They find that fire proximity increases pro-environment voting, but only in already-Democratic areas; Republican areas show no response. \citet{kronborg2024wildfires} study Sweden's severe 2018 wildfire season and find that affected municipalities showed approximately 1~percentage point higher Green Party vote share, with concurrent incumbent punishment---the only other paper documenting wildfire-specific electoral effects. My paper extends this literature in three ways: I use smoke exposure rather than fire proximity or fire occurrence, providing a plausibly exogenous treatment determined by atmospheric dispersion; I study the entire continental United States; and I document effects on \textit{turnout}, not just vote choice. \citet{liao2022extreme} provide complementary evidence showing that natural disasters hurt incumbents with anti-environment records in House races, using campaign contribution data.

\paragraph{Air pollution and voting.} \citet{bellani2024air} is the closest analogue for the pollution--voting mechanism. Using 60 German federal and state elections, they exploit within-county variation in PM\textsubscript{10} on election day to show that higher pollution shifts votes from incumbent to opposition parties. They argue this operates through a subconscious emotional channel: day-to-day PM\textsubscript{10} fluctuations are imperceptible, yet they increase negative emotions that reduce support for the status quo. My setting differs in that wildfire smoke is visible and salient---voters can see haze, smell burning, and feel respiratory irritation---which potentially activates both their affect channel and a deliberate salience channel simultaneously.

\paragraph{Incumbent punishment and retrospective voting.} My secondary anti-incumbent findings contribute to a long literature on retrospective voting. \citet{healy2010myopic} show that voters punish incumbents for disaster damage but reward disaster relief spending, implying underinvestment in preparedness. \citet{achen2016democracy} argue more provocatively that voters engage in ``blind retrospection,'' punishing incumbents for events beyond governmental control---though \citet{fowler2018sharks} and \citet{gasper2011make} show that voters do partly distinguish between harm and government response. \citet{healy2010substituting} demonstrate that even irrelevant events (college football outcomes) shift incumbent evaluations through mood contamination, while \citet{huber2012sources} confirm experimentally that voters overweight recent experiences.

\paragraph{Wildfire smoke economics.} \citet{borgschulte2022air} is the leading paper using wildfire smoke as an exogenous source of PM\textsubscript{2.5} variation, estimating that each additional smoke day reduces quarterly earnings by about 0.1\%. Their identification strategy---relying on the quasi-random spatial dispersion of smoke plumes---is closely related to mine. \citet{burke2021exposures} document that during smoke events, people express more negative sentiment and alter behavior, providing direct evidence for the mood channel I invoke. \citet{miller2024wildfire} and \citet{jung2025mental} establish that wildfire smoke harms physical and mental health, supporting the negative-affect mechanism. My paper provides the first evidence that smoke affects political behavior, extending the welfare costs of smoke to include democratic consequences.

\paragraph{Identification.} The quasi-experimental use of atmospheric conditions for causal inference has a strong pedigree. \citet{deryugina2019mortality} pioneered using wind direction as an instrument for PM\textsubscript{2.5}, estimating mortality effects among the elderly. \citet{rangel2019agricultural} use upwind/downwind variation from agricultural fires in Brazil, conceptually very close to my approach. Rather than instrumenting for total PM\textsubscript{2.5} with wind, I use the \citet{childs2022daily} data that has already separated wildfire smoke PM\textsubscript{2.5} from other sources, and argue that atmospheric dispersion is quasi-random conditional on county and year fixed effects.

%----------------------------------------------------------------------
\section{Data}
%----------------------------------------------------------------------

\paragraph{Wildfire smoke PM\textsubscript{2.5}.} I use daily county-level estimates of wildfire-attributed PM\textsubscript{2.5} from \citet{childs2022daily}, covering all U.S.\ counties from January 2006 through December 2023 (v2.0 of the dataset). These estimates use NOAA Hazard Mapping System satellite smoke plume classifications combined with machine learning to separate wildfire-derived PM\textsubscript{2.5} from background pollution.

\paragraph{Election returns.} County-level presidential election returns for 2000--2024 come from the MIT Election Data + Science Lab \citep{medsl2024county}. I use the two-party vote share (Democratic votes / [Democratic + Republican votes]) as one outcome. For House elections, I use precinct-level returns with county identifiers (2016, 2018, 2020, 2022) from the same source, aggregating precinct votes to the county level to enable analysis at the same geographic unit as the presidential regressions.

\paragraph{Analysis samples.} The overlap of smoke data (2006--2020) and presidential elections yields four election cycles: 2008, 2012, 2016, and 2020. After merging on county FIPS codes, the presidential analysis sample contains 12,432 county-election observations spanning 3,108 counties. The county-level House sample covers four election cycles (2016, 2018, 2020, 2022) with 12,206 county-election observations.

\paragraph{Voting age population and turnout rate.} To construct a turnout rate with a proper population denominator, I obtain county-level voting age population (VAP) from the American Community Survey (ACS) 5-year estimates, using the B01001 age-by-sex table to sum the population aged 18 and over. I match each election year to the closest ACS vintage (e.g., 2008 to the 2005--2009 ACS, 2020 to the 2016--2020 ACS). The turnout rate is defined as total votes cast divided by VAP. County-election observations with turnout rates exceeding 1.5 are dropped as allocation artifacts (affecting fewer than 0.1\% of observations). This yields 12,432 county-election observations for presidential elections. For House elections, VAP data are available only for the 2016 and 2020 cycles (6,088 observations), as the off-year elections (2018, 2022) do not align with the decennial-anchored ACS vintages used.

\paragraph{Smoke exposure measures.} For each county and election, I aggregate daily smoke PM\textsubscript{2.5} over pre-election windows: 7, 30, 60, and 90 days before election day, plus the full fire season (June 1 to election day). The primary treatment variable is the mean daily smoke PM\textsubscript{2.5} in the 30 days before the election, motivated by the temporal dynamics analysis showing that smoke in the weeks closest to the election has the largest effect (Section~\ref{sec:temporal}).

\paragraph{Perceptibility and threshold measures.} A key distinction between wildfire smoke and the ambient PM\textsubscript{10} studied by \citet{bellani2024air} is that smoke is perceptible: people can see haze, smell burning, and feel respiratory irritation. PM\textsubscript{2.5} is the primary driver of reduced visibility, and haze from wildfire smoke becomes noticeable at concentrations of roughly 20--40~$\mu$g/m$^3$, depending on background conditions \citep{burke2021exposures}. To capture the extensive margin of salient smoke exposure, I construct a second treatment variable: the fraction of days in the pre-election window on which daily smoke PM\textsubscript{2.5} exceeded 20~$\mu$g/m$^3$, corresponding to the onset of visible haze. In the 30-day presidential sample, 1.9\% of county-election observations (236 of 12,432) have at least one day above this threshold, providing substantially more variation than higher thresholds: only 0.4\% exceed the EPA ``Unhealthy for Sensitive Groups'' cutoff of 35.5~$\mu$g/m$^3$, and only 0.2\% exceed the ``Unhealthy'' cutoff of 55.5~$\mu$g/m$^3$. Appendix~\ref{sec:thresholds} compares results across all three thresholds.

%----------------------------------------------------------------------
\section{Empirical Strategy}
%----------------------------------------------------------------------

I estimate two-way fixed effects models of the form:
\begin{equation}
Y_{ct} = \alpha_c + \gamma_t + \beta \cdot \text{SmokePM}_{ct} + \varepsilon_{ct}
\label{eq:twfe}
\end{equation}
where $Y_{ct}$ is the outcome in county $c$ in election year $t$, $\alpha_c$ are county fixed effects absorbing all time-invariant county characteristics, $\gamma_t$ are election-year fixed effects absorbing national swings, and $\text{SmokePM}_{ct}$ is the mean wildfire smoke PM\textsubscript{2.5} in the pre-election window. Standard errors are clustered by county.

The primary outcome is the \textit{turnout rate} (total votes / voting age population). Secondary outcomes are log total votes, the Democratic two-party vote share, and the incumbent party's vote share.

\paragraph{Identifying assumption.} The key assumption is that, conditional on county and year fixed effects, variation in wildfire smoke exposure is uncorrelated with unobserved determinants of voting. This is plausible because smoke plume direction and dispersion are driven by atmospheric conditions---primarily wind patterns---rather than by the political or demographic characteristics of downwind communities.

\paragraph{Threats to identification.} Two potential concerns merit discussion. First, spatially correlated shocks such as drought could affect both fire activity and local economic conditions. This is mitigated by the fact that smoke travels hundreds of miles from fire origins, so downwind counties experience smoke without experiencing the local conditions that generated the fires. Second, secular trends in fire-prone versus non-fire-prone regions could confound the estimates; county fixed effects absorb level differences, and year fixed effects absorb national trends, but region-specific trends remain a potential concern.

\paragraph{Continuous treatment and TWFE.} Recent work by \citet{callaway2024continuous} shows that TWFE regressions with a continuous treatment variable can produce coefficients that lack a clear causal interpretation when the dose--response function is heterogeneous across units. In our setting, several features limit these concerns. First, treatment intensity (smoke PM\textsubscript{2.5}) is determined by atmospheric dispersion rather than by choices of the treated units, which sharply limits the scope for selection into dose levels. Second, because we estimate a linear specification, the TWFE coefficient corresponds to the ACRT decomposition in which weights on unit-level slopes are non-negative, provided the conditional mean of treatment given fixed effects is approximately linear. Third, as a direct robustness check, I verify that results are qualitatively similar when the continuous treatment is replaced with a binary indicator or discretized into dose quintiles.

%----------------------------------------------------------------------
\section{Results}
%----------------------------------------------------------------------

\subsection{Main Results}

Table~\ref{tab:main} presents presidential election estimates using the 30-day pre-election window under progressively more demanding specifications. Column~(1) shows the raw pooled OLS correlation with no fixed effects. Column~(2) adds county and year fixed effects---the baseline TWFE specification. Column~(3) adds five time-varying county controls (unemployment rate, log median household income, log population, October mean temperature, and October total precipitation). Column~(4) further adds state-specific linear time trends, which absorb differential secular trends across states.

The primary finding is that wildfire smoke increases voter turnout. Panel~A shows that the turnout rate is positively and significantly associated with smoke exposure across all specifications. The raw correlation in Column~(1) is positive (0.00536, $p < 0.001$), and unlike the other outcomes, does not change sign when fixed effects are added---suggesting that the turnout effect is not an artifact of cross-sectional confounders. Under TWFE (Column~2), a 1~$\mu$g/m$^3$ increase in 30-day mean smoke PM\textsubscript{2.5} raises the turnout rate by 0.10 percentage points ($p < 0.001$). Adding time-varying controls attenuates the estimate only slightly to 0.085 percentage points (Column~3, $p < 0.001$), and adding state trends reduces it further to 0.065 percentage points (Column~4, $p < 0.001$). The turnout effect is the \textit{only} outcome that survives all four specifications.

Panel~B corroborates the turnout finding using log total votes, which does not require a population denominator. The pattern is qualitatively identical: positive and highly significant across all specifications, with the state-trends estimate (0.00124, $p < 0.001$) roughly one-third the size of the TWFE baseline.

For Democratic vote share (Panel~C), the raw correlation is negative, reflecting the fact that smoke-prone Western counties tend to be more Republican. TWFE reverses the sign, revealing that \textit{within} counties, higher smoke exposure is associated with a 0.13 percentage point increase in Democratic vote share ($p < 0.001$). Controls attenuate this slightly, but state trends eliminate it ($-$0.00016, $p > 0.10$). For incumbent vote share (Panel~D), the marginally significant TWFE anti-incumbent effect ($-$0.00172, $p < 0.10$) disappears with controls.

\begin{table}[htbp]
\centering
\caption{Effect of Wildfire Smoke on Presidential Voting: Build-Up Specifications}
\label{tab:main}
\small
\begin{tabular}{lcccc}
\toprule
& (1) & (2) & (3) & (4) \\
\midrule
\multicolumn{5}{l}{\textit{Panel A: Turnout Rate}} \\[3pt]
Mean Smoke PM\textsubscript{2.5} (30d) & 0.00536*** & 0.00100*** & 0.00085*** & 0.00065*** \\
& (0.00068) & (0.00020) & (0.00019) & (0.00016) \\[6pt]
\multicolumn{5}{l}{\textit{Panel B: Log Total Votes}} \\[3pt]
Mean Smoke PM\textsubscript{2.5} (30d) & $-$0.02911*** & 0.00314*** & 0.00245*** & 0.00124*** \\
& (0.00504) & (0.00052) & (0.00051) & (0.00024) \\[6pt]
\multicolumn{5}{l}{\textit{Panel C: DEM Vote Share}} \\[3pt]
Mean Smoke PM\textsubscript{2.5} (30d) & $-$0.00831*** & 0.00135*** & 0.00125*** & $-$0.00016 \\
& (0.00132) & (0.00021) & (0.00023) & (0.00021) \\[6pt]
\multicolumn{5}{l}{\textit{Panel D: Incumbent Vote Share}} \\[3pt]
Mean Smoke PM\textsubscript{2.5} (30d) & 0.02028*** & $-$0.00172* & 0.00070 & 0.00080 \\
& (0.00259) & (0.00091) & (0.00094) & (0.00108) \\[3pt]
\midrule
County FE & & \checkmark & \checkmark & \checkmark \\
Year FE & & \checkmark & \checkmark & \checkmark \\
Controls & & & \checkmark & \checkmark \\
State trends & & & & \checkmark \\
Observations & 12,432 & 12,432 & 12,400 & 12,400 \\
\bottomrule
\multicolumn{5}{l}{\footnotesize *** $p<0.01$, ** $p<0.05$, * $p<0.10$. Standard errors clustered by county.} \\
\multicolumn{5}{l}{\footnotesize Controls: unemployment rate, log median household income, log population,} \\
\multicolumn{5}{l}{\footnotesize October mean temperature, October total precipitation.} \\
\multicolumn{5}{l}{\footnotesize Turnout rate = total votes / voting age population (ACS).} \\
\end{tabular}
\end{table}

\subsection{State-by-Year Fixed Effects}
\label{sec:stateyear}

The baseline TWFE specification identifies the smoke effect from both within-state and between-state variation in smoke exposure across elections. A natural concern is that state-level time-varying shocks---such as statewide political campaigns, ballot measures, or economic conditions---could be correlated with smoke exposure and drive the results. To address this, I replace year fixed effects with state-by-year fixed effects, so that identification comes only from \textit{within-state} variation in smoke across counties within the same election.

Table~\ref{tab:stateyear} presents the results. No coefficient is statistically significant for any outcome in either the presidential or county House specifications. The turnout rate effect shrinks from 0.00100 to essentially zero ($-$0.00005, $p = 0.79$) in the presidential sample. The Democratic vote share effect shrinks from 0.00135 to 0.00001, and the log total votes effect from 0.00314 to 0.00023.

\begin{table}[htbp]
\centering
\caption{State-by-Year Fixed Effects: Presidential and County House}
\label{tab:stateyear}
\small
\begin{tabular}{lcc}
\toprule
& (1) & (2) \\
& Presidential & County House \\
\midrule
\multicolumn{3}{l}{\textit{Panel A: Turnout Rate}} \\[3pt]
Mean Smoke PM\textsubscript{2.5} (30d) & $-$0.00005 & 0.00060 \\
& (0.00018) & (0.00092) \\[6pt]
\multicolumn{3}{l}{\textit{Panel B: Log Total Votes}} \\[3pt]
Mean Smoke PM\textsubscript{2.5} (30d) & 0.00023 & 0.00161 \\
& (0.00030) & (0.00135) \\[6pt]
\multicolumn{3}{l}{\textit{Panel C: DEM Vote Share}} \\[3pt]
Mean Smoke PM\textsubscript{2.5} (30d) & 0.00001 & 0.00011 \\
& (0.00025) & (0.00025) \\[6pt]
\multicolumn{3}{l}{\textit{Panel D: Incumbent Vote Share}} \\[3pt]
Mean Smoke PM\textsubscript{2.5} (30d) & $-$0.00079 & 0.00214 \\
& (0.00086) & (0.00140) \\[6pt]
\midrule
FE & County + State$\times$Year & County + State$\times$Year \\
Observations & 12,432 & 6,088 / 12,197 \\
\bottomrule
\multicolumn{3}{l}{\footnotesize No coefficient is significant at the 10\% level. SEs clustered by county.} \\
\multicolumn{3}{l}{\footnotesize Panels A and C--D use contested races only for County House.} \\
\end{tabular}
\end{table}

Figure~\ref{fig:resid_map} provides visual context for these null results. After residualizing county-level smoke on state-by-year means, the remaining within-state variation is concentrated in a small number of Western states during high-fire years (especially 2020). Figure~\ref{fig:stateyear_overlay} further illustrates this by overlaying the cumulative temporal dynamics under the preferred Specification~(3) and the state-by-year FE specification: the divergence between the two is stark, with the state-by-year estimates centered on zero at all window lengths.

\begin{figure}[htbp]
\centering
\includegraphics[width=\textwidth]{smoke_exposure_map_residualized.png}
\caption{Within-state variation in pre-election smoke exposure. County-level smoke PM\textsubscript{2.5} (30-day window) residualized on state-by-year means. The diverging scale shows counties with more (red) or less (blue) smoke than their state average. Meaningful within-state variation is concentrated in Western states during high-fire years.}
\label{fig:resid_map}
\end{figure}

\begin{figure}[htbp]
\centering
\includegraphics[width=\textwidth]{temporal_cumulative_stateyear.png}
\caption{Temporal dynamics: Specification~(3) vs.\ state-by-year fixed effects. Each point is a separate regression using the indicated cumulative window. The divergence illustrates the large share of identifying variation absorbed by state-by-year fixed effects.}
\label{fig:stateyear_overlay}
\end{figure}

How should these null results be interpreted? Two views are possible. Under the first, state-by-year fixed effects are overly demanding: they absorb 41\% of the identifying variation in smoke exposure, including legitimate exogenous variation driven by atmospheric dispersion patterns that differ across states. Because smoke plumes are wind-driven, between-state differences in smoke conditional on year may be genuinely exogenous, and removing this variation discards real signal. Under the second view, between-state smoke variation is correlated with state-level shocks---such as drought, economic conditions, or political campaigns---and the baseline estimates are partly confounded by these omitted factors.

I cannot definitively adjudicate between these interpretations with four presidential elections. Several considerations favor the first view: the atmospheric dispersion mechanism that generates smoke exposure is physically independent of local political and economic conditions; the smoke data from \citet{childs2022daily} isolates wildfire-specific PM\textsubscript{2.5} from other pollution sources; and \citet{borgschulte2022air} show that controlling for wind direction does not change their smoke-based labor market estimates, suggesting smoke variation is genuinely exogenous. A related finding from \citet{hilbig2024floods} is instructive: the 2021 German floods had little \textit{local} electoral effect but produced a brief \textit{national} increase in Green Party support, suggesting that environmental shocks may operate through diffuse awareness channels rather than geographically concentrated exposure. If smoke similarly affects voters through broad media coverage and regional salience rather than county-specific dosage, within-state variation may be the wrong margin to isolate the effect. However, I cannot rule out the second view, and the state-by-year null should be taken seriously as a limitation. Appendix~\ref{sec:statebystate} presents a complementary diagnostic: TWFE+controls regressions estimated separately within each state. Only 15 of 36 states yield positive turnout coefficients, and the meta-analytic mean is near zero, consistent with the between-state variation driving the pooled result---though severe power limitations (four elections per state) complicate interpretation. Additional election cycles would provide the statistical power needed to resolve this ambiguity.

\subsection{Temporal Dynamics}
\label{sec:temporal}

Figure~\ref{fig:windows} presents the temporal dynamics of the smoke--voting relationship using cumulative windows of expanding length, estimated under the preferred Specification~(3) with controls. The left column uses mean smoke PM\textsubscript{2.5} as the treatment; the right column uses the fraction of days exceeding the 20~$\mu$g/m$^3$ visible haze threshold. Each point is a separate regression in which the treatment variable is the average over the indicated window.

For turnout rate (top row), positive effects are stable and significant across all window lengths under both treatments, confirming that the turnout finding is not an artifact of a particular window choice. The mean PM\textsubscript{2.5} estimate is significant from the shortest windows and remains stable as the window expands, consistent with a persistent mobilization effect. For log total votes (second row), the pattern is similar, with positive and significant effects at all windows.

For Democratic vote share (third row), the mean PM\textsubscript{2.5} effect builds over the first 14--35 days and then stabilizes around the 30-day estimate, confirming that recent smoke drives the effect. The fraction-above-haze measure tells a similar story with larger point estimates. For incumbent vote share (bottom row), the mean PM\textsubscript{2.5} effect is concentrated in the first 14 days and washes out at longer windows, while the haze fraction shows a persistent positive effect---suggesting that visible smoke exposure may benefit incumbents at moderate levels, unlike the punishment effect found at higher thresholds (Appendix~\ref{sec:thresholds}).

The temporal pattern---strongest effects at short lags, stabilizing as the window expands---is consistent with a salience or recency mechanism and difficult to explain with confounders. It also motivates the use of the 30-day window as the base specification: short enough to capture the recency-weighted signal, long enough to smooth over week-to-week noise.

\begin{figure}[htbp]
\centering
\includegraphics[width=\textwidth]{temporal_cumulative_controls.png}
\caption{Temporal dynamics of smoke effects using expanding cumulative windows. Left column: mean smoke PM\textsubscript{2.5}. Right column: fraction of days exceeding the visible haze threshold (20~$\mu$g/m$^3$). Rows: turnout rate, log total votes, DEM vote share, incumbent vote share. Each point is a separate regression with the indicated cumulative window as treatment. All specifications include county and year FE plus time-varying controls (Specification~3). Bars are 95\% confidence intervals; SEs clustered by county.}
\label{fig:windows}
\end{figure}

\subsection{Close-In Daily Dynamics}
\label{sec:closein}

Figure~\ref{fig:closein} examines the very-short-run dynamics at daily resolution, using cumulative windows from 1 to 7 days before the election. These estimates speak to the question of how quickly smoke affects electoral behavior.

For turnout rate, the effect is positive and marginally significant from a single day of pre-election exposure (0.021, $p < 0.10$), growing steadily to 0.118 ($p < 0.10$) at the 7-day window. For log total votes, significance emerges from day 1 (0.049, $p < 0.01$) and strengthens monotonically.

The most mechanistically compelling close-in result is for incumbent vote share. The anti-incumbent effect builds monotonically from near zero at 1 day to $-$0.020 ($p < 0.01$) at 7 days, consistent with a recency-driven blame attribution mechanism in which voters punish the incumbent for discomfort experienced in the days immediately before casting their ballot. This close-in pattern---absent in the 30-day results where the effect washes out---suggests that incumbent punishment operates through a short-lived negative-affect channel, while the turnout mobilization effect is more persistent.

\begin{figure}[htbp]
\centering
\includegraphics[width=\textwidth]{temporal_closein_daily.png}
\caption{Close-in daily dynamics: cumulative windows from 1 to 7 days before the election. Each point is a separate regression (Specification~3). The monotonic build-up of the anti-incumbent effect is consistent with a recency-driven blame attribution mechanism. Bars are 95\% confidence intervals; SEs clustered by county.}
\label{fig:closein}
\end{figure}

\subsection{Geographic Variation in Smoke Exposure}

Figure~\ref{fig:maps} displays county-level mean smoke PM\textsubscript{2.5} in the 30 days before each election. The maps illustrate both the geographic scope and temporal variation that identify the main estimates: 2016 saw minimal pre-election smoke nationwide, while 2020 produced extreme exposure across the Western states following the historic August--September fire season. The 2022 midterm also shows substantial smoke exposure.

\begin{figure}[htbp]
\centering
\includegraphics[width=\textwidth]{smoke_exposure_map_panel.png}
\caption{Pre-election wildfire smoke exposure by county, 30-day window before election day. Color scale is identical across all panels.}
\label{fig:maps}
\end{figure}

\subsection{House Elections}

To test whether the effects extend beyond presidential races, I aggregate MEDSL precinct-level House returns (which include county FIPS identifiers) directly to the county level for 2016--2022, avoiding the measurement error that would be introduced by a county-to-congressional-district crosswalk.

Table~\ref{tab:house} presents the county-level House results alongside the presidential estimates, all using the 30-day mean smoke PM\textsubscript{2.5} treatment and TWFE. The turnout effect is strong in both presidential and House races. County-level House turnout shows a larger point estimate (0.00173, $p < 0.05$) than presidential turnout (0.00100, $p < 0.001$), though the House estimate is less precise due to the smaller sample (VAP data are available only for 2016 and 2020 House elections). Log total votes are also significantly positive in both samples. The pro-Democratic effect is not significant in House races, while the incumbent effect flips sign (positive in House races), possibly reflecting the advantage of name recognition for House incumbents in smoke-affected areas.

\begin{table}[htbp]
\centering
\caption{Effect of Wildfire Smoke: County-Level House vs.\ Presidential}
\label{tab:house}
\small
\begin{tabular}{lcc}
\toprule
& (1) & (2) \\
& County House & Presidential \\
\midrule
\multicolumn{3}{l}{\textit{Panel A: Turnout Rate}} \\[3pt]
Mean Smoke PM\textsubscript{2.5} (30d) & 0.00173** & 0.00100*** \\
& (0.00076) & (0.00020) \\[6pt]
\multicolumn{3}{l}{\textit{Panel B: Log Total Votes}} \\[3pt]
Mean Smoke PM\textsubscript{2.5} (30d) & 0.00356*** & 0.00314*** \\
& (0.00119) & (0.00052) \\[6pt]
\multicolumn{3}{l}{\textit{Panel C: DEM Vote Share}} \\[3pt]
Mean Smoke PM\textsubscript{2.5} (30d) & $-$0.00033 & 0.00135*** \\
& (0.00029) & (0.00021) \\[6pt]
\multicolumn{3}{l}{\textit{Panel D: Incumbent Vote Share}} \\[3pt]
Mean Smoke PM\textsubscript{2.5} (30d) & 0.00304* & $-$0.00172* \\
& (0.00178) & (0.00091) \\[6pt]
\midrule
Unit & County & County \\
FE & County + Year & County + Year \\
Elections & 2016--2022 & 2008--2020 \\
\bottomrule
\multicolumn{3}{l}{\footnotesize *** $p<0.01$, ** $p<0.05$, * $p<0.10$. SEs clustered by county.} \\
\multicolumn{3}{l}{\footnotesize Panels C--D use contested races only (House: 11,155 obs; Presidential: 12,432).} \\
\multicolumn{3}{l}{\footnotesize Panel B uses all races (House: 12,197; Presidential: 12,432).} \\
\multicolumn{3}{l}{\footnotesize Panel A: House 6,088 obs (VAP available 2016, 2020 only); Presidential 12,432.} \\
\end{tabular}
\end{table}

As a further robustness check, I also estimate the House specifications at the congressional district level using Census county-to-district crosswalks (Appendix Table~\ref{tab:district_house}). The district-level estimates are noisier due to the measurement error introduced by the crosswalk.

\subsection{Robustness to Excluding 2020}

The 2020 election coincided with historically extreme wildfire smoke across the Western United States, raising the question of whether the main results are driven by this single year. Figure~\ref{fig:drop2020} addresses this by overlaying the cumulative temporal dynamics for the full sample and the sample excluding 2020, using the same Specification~(3) as Figure~\ref{fig:windows}.

The turnout effect is positive and significant in both samples. At the 28-day window, the turnout rate coefficient is 0.0011 ($p < 0.05$) in the full sample and 0.0040 ($p < 0.001$) excluding 2020---the turnout effect is actually \textit{stronger} without 2020, confirming that it is not driven by the extreme fire season. The same pattern holds for log total votes.

For the secondary outcomes, the divergence between samples is more dramatic. The pro-Democratic effect reverses sign when 2020 is excluded: the full-sample positive effect at 28 days ($\beta = 0.0023$, $p < 0.001$) becomes negative without 2020 ($\beta = -0.0034$, $p < 0.001$). This confirms that the pro-Democratic result is leveraged by the 2020 Western fire season. The anti-incumbent effect, however, is dramatically \textit{stronger} without 2020: near zero in the full sample but large and significant excluding 2020 ($\beta = -0.033$, $p < 0.001$), suggesting that the 2020 election---in which smoke-exposed areas happened to favor the incumbent---masked a strong underlying punishment effect.

\begin{figure}[htbp]
\centering
\includegraphics[width=\textwidth]{temporal_drop2020.png}
\caption{Temporal dynamics: full sample vs.\ excluding 2020. Solid lines with circles show the full sample; dashed lines with squares exclude the 2020 election. Left column: mean smoke PM\textsubscript{2.5}. Right column: fraction of days exceeding the visible haze threshold (20~$\mu$g/m$^3$). All specifications include county and year FE plus time-varying controls (Specification~3). Bars are 95\% confidence intervals; SEs clustered by county.}
\label{fig:drop2020}
\end{figure}

\subsection{Heterogeneity by Prior Partisanship}

Table~\ref{tab:hetero} splits the presidential sample by terciles of lagged Democratic vote share to test whether the pro-Democratic shift from smoke varies across the partisan spectrum. The effect is present in R-leaning and D-leaning counties but is absent in swing counties. Unlike \citet{hazlett2020wildfire}, who find effects \textit{only} in Democratic areas for fire proximity, smoke exposure moves both R-leaning and D-leaning counties toward the Democrats, with the strongest effect in R-leaning counties (0.155~pp). This cross-partisan pattern is consistent with smoke operating as a general salience signal rather than reinforcing existing preferences.

\begin{table}[htbp]
\centering
\caption{Heterogeneity by Prior Partisanship (DEM Vote Share)}
\label{tab:hetero}
\small
\begin{tabular}{lccc}
\toprule
& R-Leaning & Swing & D-Leaning \\
\midrule
Mean Smoke PM\textsubscript{2.5} (30d) & 0.00155*** & 0.00007 & 0.00126*** \\
& (0.00033) & (0.00055) & (0.00040) \\[6pt]
Observations & 4,146 & 4,144 & 4,147 \\
\bottomrule
\multicolumn{4}{l}{\footnotesize *** $p<0.01$, ** $p<0.05$, * $p<0.10$. County and year FE. SEs clustered by county.} \\
\multicolumn{4}{l}{\footnotesize Outcome: DEM vote share. Terciles defined by lagged DEM vote share.} \\
\end{tabular}
\end{table}

\subsection{Mechanism: Engagement Without Preference Change}
\label{sec:nominate}

The results so far establish that smoke robustly increases turnout but has fragile effects on vote choice. A natural question is whether these marginal shifts in vote shares translate into changes in the \textit{type} of representative elected---that is, whether smoke affects not just how many people vote, but the ideological composition of government. \citet{autor2020importing} show that trade shocks shifted the U.S.\ House toward more ideologically extreme members, as moderates lost to candidates further from the center. If smoke similarly polarizes the electorate, we should see it reflected in the roll-call voting behavior of elected representatives.

To test this, I merge DW-NOMINATE first-dimension scores from VoteView \citep{lewis2025voteview} with the district-level House analysis data for Congresses 110--117 (elections 2006--2020; $N = 3{,}449$ district-year observations). The first-dimension score captures the primary liberal--conservative axis of roll-call voting, with negative values indicating more liberal positions and positive values more conservative. I construct three outcome variables: $|$NOMINATE dim1$|$ (ideological extremity, where higher values indicate greater polarization), the signed NOMINATE dim1 score (directional ideology), and an indicator for moderate representatives ($|$dim1$|$ below the sample median).

Table~\ref{tab:nominate} reports TWFE estimates with district and year fixed effects, standard errors clustered by district. The results are unambiguously null. Smoke exposure has no significant effect on ideological extremity ($\beta = -0.00035$, $p = 0.71$), directional ideology ($\beta = 0.00178$, $p = 0.33$), or the probability of electing a moderate ($\beta = 0.00263$, $p = 0.41$). The null is not driven by offsetting effects across parties: splitting the sample by winner party, smoke does not push Democratic representatives further left ($\beta = 0.00047$, $p = 0.71$) or Republican representatives further right ($\beta = -0.00130$, $p = 0.16$). State$\times$year fixed effects do not alter this conclusion.

\begin{table}[htbp]
\centering
\caption{Effect of Wildfire Smoke on Congressional Polarization (DW-NOMINATE)}
\label{tab:nominate}
\small
\begin{tabular}{lccc}
\toprule
& (1) & (2) & (3) \\
& $|$NOMINATE dim1$|$ & NOMINATE dim1 & Is Moderate (LPM) \\
\midrule
\multicolumn{4}{l}{\textit{Panel A: All Members}} \\[3pt]
Mean Smoke PM\textsubscript{2.5} (30d) & $-$0.00035 & 0.00178 & 0.00263 \\
& (0.00092) & (0.00184) & (0.00317) \\
$R^2$ (within) & $-$0.0004 & 0.0003 & $-$0.0005 \\[6pt]
\multicolumn{4}{l}{\textit{Panel B: DEM Winners Only}} \\[3pt]
Mean Smoke PM\textsubscript{2.5} (30d) & $-$0.00047 & 0.00047 & \\
& (0.00127) & (0.00127) & \\[6pt]
\multicolumn{4}{l}{\textit{Panel C: REP Winners Only}} \\[3pt]
Mean Smoke PM\textsubscript{2.5} (30d) & $-$0.00130 & $-$0.00130 & \\
& (0.00093) & (0.00093) & \\[6pt]
\multicolumn{4}{l}{\textit{Panel D: State$\times$Year FE}} \\[3pt]
Mean Smoke PM\textsubscript{2.5} (30d) & $-$0.00017 & 0.00266 & $-$0.00052 \\
& (0.00109) & (0.00255) & (0.00454) \\[6pt]
\midrule
FE & District + Year & District + Year & District + Year \\
Observations (Panel A) & 3,449 & 3,449 & 3,449 \\
\bottomrule
\multicolumn{4}{l}{\footnotesize No coefficient is significant at the 10\% level.} \\
\multicolumn{4}{l}{\footnotesize SEs clustered by district. DW-NOMINATE scores from VoteView, Congresses 110--117.} \\
\multicolumn{4}{l}{\footnotesize $|$NOMINATE dim1$|$ measures polarization (extremity); Is Moderate = 1 if $|$dim1$|$ $<$ median.} \\
\end{tabular}
\end{table}

This null is informative. It means that even where smoke marginally shifts vote shares, these shifts do not alter the ideological composition of the legislature. The contrast with \citet{autor2020importing} is instructive: trade shocks---which are persistent, geographically concentrated, and economically devastating---changed \textit{who gets elected}. Smoke---which is transient, diffuse, and primarily a salience shock---changes \textit{whether people vote} but not the type of candidate who wins. This pattern pins down the mechanism as engagement without preference change: smoke mobilizes voters roughly symmetrically across the partisan spectrum, consistent with the robust turnout effect and the fragility of the vote-choice results.

%----------------------------------------------------------------------
\section{Discussion}
%----------------------------------------------------------------------

The primary finding of this paper is that wildfire smoke increases voter turnout. This effect is robust across specifications: it survives the addition of time-varying county controls and state-specific linear trends, is present in both presidential and House elections, and is not driven by the extreme 2020 fire season. The positive sign of the effect rules out a \textit{disruption} mechanism---in which smoke deters participation as election-day rain does \citep{gomez2007weather}---and instead points toward \textit{mobilization}. Visible smoke may increase the salience of environmental and public health concerns, prompting voters who might otherwise abstain to participate. This interpretation is consistent with \citet{jusko2024motivated}, who finds that severe flooding in Slovakia increases turnout, and with the broader pattern that acute environmental shocks can mobilize rather than suppress participation when they raise the perceived stakes of collective action.

An important distinction is between salience-driven \textit{engagement} and salience-driven \textit{preference change}. \citet{andrews2025wildfire} show that wildfire experience in the American West increases climate change belief---especially among Republicans---but does not increase willingness to spend on mitigation. This dissociation between ``paying attention'' and ``paying the costs'' maps onto my results: the turnout effect (engagement) is robust, while the pro-Democratic effect (preference) is fragile. The NOMINATE analysis (Section~\ref{sec:nominate}) provides the strongest evidence for this channel: smoke does not change the ideological composition of the House, meaning that even where vote shares shift marginally, the shifts are too small or too symmetric to alter who wins. Smoke makes voters more attentive and motivated to participate without systematically shifting which candidates hold office.

Two secondary findings provide suggestive evidence about how smoke affects vote \textit{choice}. First, smoke exposure is associated with a pro-Democratic shift in presidential voting, consistent with climate salience benefiting the party perceived as more pro-environment \citep{hazlett2020wildfire, kahn2007voting}. However, this effect is fragile: it is driven by the 2020 fire season (reversing sign without it) and does not survive state trends. \citet{arias2024hurricane} find a parallel pattern for Hurricane Ian: climate attitudes shifted across partisan lines in the immediate aftermath, but effects decayed within months---consistent with my temporal dynamics showing the strongest effects at short pre-election windows. Second, close-in incumbent punishment---building monotonically from 0 to $-$0.020 over the 7 days before the election---is consistent with the negative-affect channel identified by \citet{bellani2024air}, in which experienced discomfort reduces support for the status quo. \citet{du2024smoke} provide population-scale evidence for this mechanism, showing that transboundary wildfire smoke significantly reduces expressed sentiment on social media. \citet{kronborg2024wildfires} document a similar incumbent punishment effect for Sweden's 2018 wildfire season. This close-in effect is strongest when 2020 is excluded, suggesting it is a general phenomenon not specific to one election cycle.

The state-by-year fixed effects null (Section~\ref{sec:stateyear}) is the most important limitation. As discussed there, two interpretations are possible: the null could reflect an overly demanding specification that discards genuine exogenous variation, or it could indicate that state-level confounders partly drive the baseline results. I lean toward the first interpretation given the atmospheric dispersion mechanism, but cannot rule out the second with only four presidential elections. The fact that 41\% of the identifying variation is between-state-by-year makes this a binding constraint. Future work with additional election cycles---the 2024 election, with its major Canadian wildfire smoke events affecting the Eastern United States, would be particularly valuable---could resolve this ambiguity by providing enough within-state variation for precise estimation under state-by-year FE.

\paragraph{Alternative smoke measurement.} As a further robustness check, Appendix~\ref{sec:hms} compares the Childs et al.\ ML-predicted PM\textsubscript{2.5} with an entirely independent smoke exposure measure: NOAA's Hazard Mapping System (HMS), in which human analysts visually identify smoke plumes from satellite imagery \citep{borgschulte2022air}. The two datasets agree strongly on the extensive margin of smoke presence (county-day Cohen's $\kappa = 0.86$; 30-day smoke days $r = 0.91$) but diverge at intensity thresholds ($r \approx 0.2$--$0.3$), reflecting the fundamental difference between satellite-observed visual density and ground-level PM\textsubscript{2.5} concentration. In regressions, the broad HMS ``any smoke'' treatment does not reproduce the positive turnout finding. However, when restricted to \textit{heavy} HMS smoke, both measures converge: HMS heavy smoke produces positive and significant effects on turnout ($+$0.077, $p < 0.01$) and Democratic vote share ($+$0.108, $p < 0.01$), paralleling the Childs results at comparable thresholds. This convergence at high intensity---despite weak correlation between the threshold variables---suggests that severe smoke events produce real behavioral effects regardless of measurement approach, while the divergence at low intensity reflects the superior ability of continuous PM\textsubscript{2.5} to capture the ground-level exposure relevant to human behavior.

\paragraph{Limitations.} Several additional limitations merit discussion. The analysis covers only four presidential elections and four House elections, and the sample is small by the standards of the environmental economics literature. County-level aggregation may mask within-county heterogeneity. And the negative within-$R^2$ values in some specifications suggest that the smoke variable alone explains limited within-county variation after absorbing fixed effects, underscoring that these are small effects on a noisy outcome.

%----------------------------------------------------------------------
\section{Conclusion}
%----------------------------------------------------------------------

Wildfire smoke exposure in the 30 days before an election increases voter turnout, a finding that survives progressively demanding specifications including state-specific linear time trends and is present in both presidential and House elections. The positive sign rules out disruption and points toward a mobilization mechanism driven by the salience of visible smoke. Crucially, the null effect on DW-NOMINATE scores of elected House members shows that smoke does not change the ideological composition of the legislature, pinning down the mechanism as symmetric engagement---more people voting---rather than partisan preference change. Suggestive evidence also points to a pro-Democratic shift and close-in incumbent punishment, though these vote-choice effects are less robust.

The most important caveat is that state-by-year fixed effects---which absorb 41\% of the identifying variation---eliminate all effects. Whether this reflects insufficient within-state power or genuine state-level confounding cannot be resolved with four elections. Additional election cycles, particularly the 2024 cycle with its geographically novel smoke patterns, offer the most promising path to sharper identification. Wildfire smoke---which is plausibly exogenous and affects a far larger population than fire proximity---remains a promising research design for studying how environmental experience shapes democratic participation.

\bibliography{references}

%----------------------------------------------------------------------
\appendix
\section{Supplementary Results}
\setcounter{table}{0}
\renewcommand{\thetable}{A\arabic{table}}
\setcounter{figure}{0}
\renewcommand{\thefigure}{A\arabic{figure}}

%----------------------------------------------------------------------
\subsection{Exclusive 7-Day Windows}
\label{sec:exclusive_windows}
%----------------------------------------------------------------------

Figures~\ref{fig:excl_mean} and~\ref{fig:excl_frac} present the exclusive-window counterparts to the cumulative results in Figure~\ref{fig:windows}. In the exclusive specification, all thirteen non-overlapping 7-day bins are entered simultaneously as regressors. These estimates identify the marginal contribution of each week's smoke exposure holding all other weeks constant. The sign-flipping between adjacent bins reflects multicollinearity among temporally proximate smoke measures; the cumulative windows in the main text provide the more interpretable summary.

\begin{figure}[htbp]
\centering
\includegraphics[width=\textwidth]{temporal_7day_mean_controls.png}
\caption{Temporal dynamics: mean smoke PM\textsubscript{2.5}. Left column: exclusive 7-day windows (all bins entered simultaneously). Right column: cumulative windows (same as left column of Figure~\ref{fig:windows}). Rows: turnout rate, log total votes, DEM vote share, incumbent vote share. County and year FE with controls. 95\% CIs; SEs clustered by county.}
\label{fig:excl_mean}
\end{figure}

\begin{figure}[htbp]
\centering
\includegraphics[width=\textwidth]{temporal_7day_frac_controls.png}
\caption{Temporal dynamics: fraction of days exceeding the visible haze threshold (20~$\mu$g/m$^3$). Left column: exclusive 7-day windows. Right column: cumulative windows (same as right column of Figure~\ref{fig:windows}). Rows: turnout rate, log total votes, DEM vote share, incumbent vote share. County and year FE with controls. 95\% CIs; SEs clustered by county.}
\label{fig:excl_frac}
\end{figure}

%----------------------------------------------------------------------
\clearpage
\subsection{Threshold Comparison}
\label{sec:thresholds}
%----------------------------------------------------------------------

Table~\ref{tab:thresholds} compares the fraction-above-threshold treatment variable at three cutoffs: 20~$\mu$g/m$^3$ (onset of visible haze, used in the main text), 35.5~$\mu$g/m$^3$ (EPA ``Unhealthy for Sensitive Groups''), and 55.5~$\mu$g/m$^3$ (EPA ``Unhealthy''). All specifications use the 30-day pre-election window with county and year fixed effects plus time-varying controls (Specification~3).

The haze threshold has substantially more variation (236 nonzero observations, 1.9\%) than the higher thresholds (47 and 19 nonzero, respectively). For turnout rate, the haze threshold produces a significant positive effect (0.045, $p < 0.01$), the unhealthy threshold is also significant (0.049, $p < 0.05$), but the intermediate USG threshold is not---likely reflecting the small number of nonzero observations at that cutoff. For Democratic vote share, all three thresholds produce significant positive estimates, with larger coefficients at higher thresholds---consistent with a dose-response relationship. For incumbent vote share, the sign flips across thresholds: the haze measure shows a significant \textit{positive} effect while the unhealthy measure shows a significant \textit{negative} effect, consistent with dual mechanisms operating at different exposure intensities.

\begin{table}[htbp]
\centering
\caption{Threshold Comparison: Fraction of Days Above Cutoff (30-Day Window)}
\label{tab:thresholds}
\small
\begin{tabular}{lccc}
\toprule
& Haze ($>$20) & USG ($>$35.5) & Unhealthy ($>$55.5) \\
\midrule
\multicolumn{4}{l}{\textit{Panel A: Turnout Rate}} \\[3pt]
Frac.\ days above threshold & 0.04497*** & $-$0.00065 & 0.04909** \\
& (0.01427) & (0.03037) & (0.02279) \\[6pt]
\multicolumn{4}{l}{\textit{Panel B: Log Total Votes}} \\[3pt]
Frac.\ days above threshold & 0.10585*** & 0.04803 & 0.09839 \\
& (0.03298) & (0.06271) & (0.08007) \\[6pt]
\multicolumn{4}{l}{\textit{Panel C: DEM Vote Share}} \\[3pt]
Frac.\ days above threshold & 0.06162*** & 0.09929** & 0.15047*** \\
& (0.01611) & (0.04410) & (0.03746) \\[6pt]
\multicolumn{4}{l}{\textit{Panel D: Incumbent Vote Share}} \\[3pt]
Frac.\ days above threshold & 0.16253** & $-$0.16716 & $-$0.24933*** \\
& (0.06755) & (0.10465) & (0.07825) \\[3pt]
\midrule
Nonzero obs (of 12,432) & 236 (1.9\%) & 47 (0.4\%) & 19 (0.2\%) \\
County + Year FE & \checkmark & \checkmark & \checkmark \\
Controls & \checkmark & \checkmark & \checkmark \\
Observations & 12,400 & 12,400 & 12,400 \\
\bottomrule
\multicolumn{4}{l}{\footnotesize *** $p<0.01$, ** $p<0.05$, * $p<0.10$. SEs clustered by county. Presidential elections 2008--2020.} \\
\end{tabular}
\end{table}

%----------------------------------------------------------------------
\clearpage
\subsection{District-Level House Estimates}
\label{sec:district_house}
%----------------------------------------------------------------------

\begin{table}[htbp]
\centering
\caption{Robustness: District-Level House Estimates}
\label{tab:district_house}
\small
\begin{tabular}{lcc}
\toprule
& (1) & (2) \\
& District House & County House \\
\midrule
\multicolumn{3}{l}{\textit{Panel A: DEM Vote Share}} \\[3pt]
Mean Smoke PM\textsubscript{2.5} (30d) & $-$0.00081 & $-$0.00033 \\
& (0.00086) & (0.00029) \\[6pt]
\multicolumn{3}{l}{\textit{Panel B: Incumbent Vote Share}} \\[3pt]
Mean Smoke PM\textsubscript{2.5} (30d) & $-$0.00162 & 0.00304* \\
& (0.00162) & (0.00178) \\[6pt]
\multicolumn{3}{l}{\textit{Panel C: Log Total Votes}} \\[3pt]
Mean Smoke PM\textsubscript{2.5} (30d) & $-$0.01022* & 0.00356*** \\
& (0.00565) & (0.00119) \\[6pt]
\midrule
Unit & District & County \\
FE & District + Year & County + Year \\
Observations & 3,406 / 3,879 & 11,155 / 12,197 \\
Elections & 2006--2022 & 2016--2022 \\
\bottomrule
\multicolumn{3}{l}{\footnotesize *** $p<0.01$, ** $p<0.05$, * $p<0.10$. SEs clustered by unit.} \\
\multicolumn{3}{l}{\footnotesize District-level uses Census crosswalk to map county smoke to districts.} \\
\end{tabular}
\end{table}

%----------------------------------------------------------------------
\clearpage
\subsection{Controls Robustness: Presidential and County House}
\label{sec:controls}
%----------------------------------------------------------------------

Table~\ref{tab:controls} presents the controls robustness check for both presidential and county House specifications side by side. Controls are the unemployment rate (BLS LAUS), log median household income (Census SAIPE), log population (Census Population Estimates), and October mean temperature and total precipitation (PRISM).

\begin{table}[htbp]
\centering
\caption{Robustness: Adding Time-Varying County Controls}
\label{tab:controls}
\small
\begin{tabular}{lcccc}
\toprule
& \multicolumn{2}{c}{Presidential} & \multicolumn{2}{c}{County House} \\
\cmidrule(lr){2-3} \cmidrule(lr){4-5}
& Baseline & + Controls & Baseline & + Controls \\
\midrule
\multicolumn{5}{l}{\textit{Panel A: Turnout Rate}} \\[3pt]
Mean Smoke PM\textsubscript{2.5} (30d) & 0.00100*** & 0.00085*** & 0.00173** & 0.00126* \\
& (0.00020) & (0.00019) & (0.00076) & (0.00072) \\[6pt]
\multicolumn{5}{l}{\textit{Panel B: Log Total Votes}} \\[3pt]
Mean Smoke PM\textsubscript{2.5} (30d) & 0.00314*** & 0.00245*** & 0.00356*** & 0.00323*** \\
& (0.00052) & (0.00051) & (0.00119) & (0.00114) \\[6pt]
\multicolumn{5}{l}{\textit{Panel C: DEM Vote Share}} \\[3pt]
Mean Smoke PM\textsubscript{2.5} (30d) & 0.00135*** & 0.00125*** & $-$0.00033 & $-$0.00043 \\
& (0.00021) & (0.00023) & (0.00029) & (0.00028) \\[6pt]
\multicolumn{5}{l}{\textit{Panel D: Incumbent Vote Share}} \\[3pt]
Mean Smoke PM\textsubscript{2.5} (30d) & $-$0.00172* & 0.00070 & 0.00304* & 0.00375** \\
& (0.00091) & (0.00094) & (0.00178) & (0.00151) \\[3pt]
\midrule
County FE & \checkmark & \checkmark & \checkmark & \checkmark \\
Year FE & \checkmark & \checkmark & \checkmark & \checkmark \\
Controls & & \checkmark & & \checkmark \\
Observations & 12,432 & 12,400 & 6,088 / 12,197 & 6,072 / 12,165 \\
\bottomrule
\multicolumn{5}{l}{\footnotesize *** $p<0.01$, ** $p<0.05$, * $p<0.10$. SEs clustered by county.} \\
\multicolumn{5}{l}{\footnotesize Controls: unemployment rate, log median household income, log population,} \\
\multicolumn{5}{l}{\footnotesize October mean temperature, October total precipitation.} \\
\multicolumn{5}{l}{\footnotesize Panels C--D use contested races only; Panels A--B include all.} \\
\end{tabular}
\end{table}

%----------------------------------------------------------------------
\clearpage
\subsection{State-by-State Analysis}
\label{sec:statebystate}
%----------------------------------------------------------------------

A concern with the state$\times$year fixed effects result (Section~\ref{sec:stateyear}) is that the pooled turnout effect may reflect cross-state confounding rather than within-state smoke variation. We address this by estimating TWFE+controls regressions separately within each state and examining the distribution of coefficients.

For each of the 36 CONUS states with at least 10 counties and sufficient non-missing data, we estimate the baseline specification (county and year FE plus time-varying controls) for turnout rate on 30-day mean smoke PM\textsubscript{2.5}. Figure~\ref{fig:state_forest} presents the resulting forest plot, and Figure~\ref{fig:state_histogram} shows the coefficient distribution.

Of 36 states, 15 (42\%) yield positive coefficients---not significantly different from 50\% under a binomial sign test ($p = 0.879$). The inverse-variance weighted meta-analytic mean is 0.00010 ($p = 0.329$), substantially below the pooled TWFE+controls estimate of 0.00085. Cochran's $Q$ test rejects homogeneity ($I^2 = 63\%$), confirming substantial cross-state heterogeneity.

However, the exercise has important power limitations. With only four election years and 30--250 counties per state, most individual state regressions lack the statistical power to detect an effect of the magnitude found in the pooled specification. Consistent with this, the 10 most precisely estimated states---those with the greatest within-state smoke variation---show a 7-to-3 positive-to-negative split. Geographically, western states (which experience the most intense smoke) are nearly uniformly negative (1/8 positive), while southern and eastern states tend positive. This heterogeneity may reflect adaptation in fire-prone regions or differential salience of novel versus recurring smoke events, but it also means the pooled result draws disproportionately on cross-state comparisons.

We interpret these results as reinforcing the conclusion from Section~\ref{sec:stateyear}: the turnout effect is robust under county and year fixed effects with controls and even state-specific linear trends, but relies in part on between-state variation in smoke exposure. The complementary California tract-level analysis---where county$\times$year fixed effects preserve the turnout effect using within-county variation across thousands of tracts---provides the strongest evidence that the effect is not purely an artifact of cross-state confounding.

\begin{figure}[htbp]
\centering
\includegraphics[width=\textwidth,height=0.85\textheight,keepaspectratio]{state_forest_turnout.png}
\caption{Forest plot of state-by-state TWFE+controls estimates for turnout rate on 30-day mean smoke PM\textsubscript{2.5}. States ordered by precision (most precise at top). Colors indicate significance: dark blue ($p < 0.05$), medium blue ($p < 0.10$), light blue ($p \geq 0.10$). Dashed red line: inverse-variance weighted meta-analytic mean. Dotted green line: pooled TWFE+controls estimate from the full sample. Of 36 states, 15 (42\%) yield positive coefficients (sign test $p = 0.879$).}
\label{fig:state_forest}
\end{figure}

\begin{figure}[htbp]
\centering
\includegraphics[width=0.85\textwidth]{state_histogram_turnout.png}
\caption{Distribution of state-level coefficients for turnout rate on 30-day mean smoke PM\textsubscript{2.5}. Dashed red line: meta-analytic mean. Dotted orange line: median.}
\label{fig:state_histogram}
\end{figure}

%----------------------------------------------------------------------
\clearpage
\subsection{HMS Satellite Smoke Plume Comparison}
\label{sec:hms}
%----------------------------------------------------------------------

As a validation exercise, I compare the Childs et al.\ ML-predicted smoke PM\textsubscript{2.5} used throughout this paper with an entirely independent smoke exposure dataset: NOAA's Hazard Mapping System (HMS). HMS analysts visually identify smoke plumes in satellite imagery and digitize polygon boundaries, classifying each as Light, Medium, or Heavy density. I construct county-level HMS exposure using the standard centroid method \citep{borgschulte2022air}: a county is classified as smoke-exposed on a given day if its centroid falls within any HMS polygon.

\paragraph{Validation.} At the county-day level (1.99 million observations across the four presidential election windows), the two measures show strong extensive-margin agreement: Cohen's $\kappa = 0.86$, with 98.7\% sensitivity and 94.8\% specificity. At the 30-day window level, smoke day counts correlate at $r = 0.91$ and mean PM\textsubscript{2.5} vs.\ mean density at $r = 0.67$. However, intensity threshold variables---which serve as treatment variables in the regressions---correlate weakly ($r \approx 0.2$--$0.3$), reflecting the fundamental difference between satellite-observed visual plume density and ground-level PM\textsubscript{2.5} concentration.

\paragraph{Regression comparison.} Table~\ref{tab:hms_buildup} presents the build-up specification table using the fraction of 30-day pre-election days with any HMS smoke as treatment, alongside the Childs PM\textsubscript{2.5} results for comparison. The broad HMS ``any smoke'' measure does not reproduce the positive turnout finding: the coefficient is negative and significant ($-$0.032 in Specification~3), opposite in sign to the Childs result ($+$0.00085). This divergence likely reflects that HMS ``any smoke'' is an extremely broad measure---over the full fire season, 98--100\% of counties register at least one day of satellite-observed smoke---and the within-county variation in this variable after absorbing fixed effects may capture analyst classification decisions rather than meaningful exposure differences.

However, when restricted to \textit{heavy} HMS smoke (density $\geq 3$), the results converge with the Childs findings. HMS heavy smoke produces positive and significant effects on turnout rate ($+$0.077, $p < 0.01$) and Democratic vote share ($+$0.108, $p < 0.01$). This convergence at high intensity---despite the weak correlation between the underlying threshold variables ($r = 0.30$)---provides a form of triangulation: two independent measurement systems identify the same behavioral response to unambiguously severe smoke events.

The overall pattern supports using continuous ground-level PM\textsubscript{2.5} as the preferred treatment variable: it captures the dose of pollution voters actually experience, whereas satellite visual density reflects column-integrated optical properties that depend on plume altitude and viewing geometry. For studying human behavioral responses, ground-level concentration is the theoretically superior measure.

\begin{table}[htbp]
\centering
\caption{Build-Up Table: Childs ML-Predicted PM\textsubscript{2.5} vs.\ HMS Satellite Smoke}
\label{tab:hms_buildup}
\small
\begin{tabular}{lcccc}
\toprule
& (1) Raw OLS & (2) TWFE & (3) +Controls & (4) +St.\ Trends \\
\midrule
\multicolumn{5}{l}{\textit{Panel A: Turnout Rate}} \\
Childs PM\textsubscript{2.5} (30d) & 0.00536*** & 0.00100*** & 0.00085*** & 0.00065*** \\
& (0.00068) & (0.00020) & (0.00019) & (0.00016) \\
HMS frac days (30d) & 0.12050*** & $-$0.04203*** & $-$0.03238*** & $-$0.01454* \\
& (0.00447) & (0.00946) & (0.00862) & (0.00823) \\[6pt]
\multicolumn{5}{l}{\textit{Panel B: Log Total Votes}} \\
Childs PM\textsubscript{2.5} (30d) & $-$0.02911*** & 0.00314*** & 0.00245*** & 0.00124*** \\
& (0.00504) & (0.00052) & (0.00051) & (0.00024) \\
HMS frac days (30d) & $-$0.11497** & $-$0.06687*** & $-$0.00807 & 0.04425*** \\
& (0.05857) & (0.02559) & (0.02468) & (0.01482) \\[6pt]
\multicolumn{5}{l}{\textit{Panel C: DEM Vote Share}} \\
Childs PM\textsubscript{2.5} (30d) & $-$0.00831*** & 0.00135*** & 0.00125*** & $-$0.00016 \\
& (0.00132) & (0.00021) & (0.00023) & (0.00021) \\
HMS frac days (30d) & $-$0.20738*** & $-$0.02617** & 0.00646 & $-$0.04905*** \\
& (0.00733) & (0.01118) & (0.01102) & (0.00847) \\[6pt]
\multicolumn{5}{l}{\textit{Panel D: Incumbent Vote Share}} \\
Childs PM\textsubscript{2.5} (30d) & 0.02028*** & $-$0.00172* & 0.00070 & 0.00080 \\
& (0.00259) & (0.00091) & (0.00094) & (0.00108) \\
HMS frac days (30d) & 0.71349*** & 0.24357*** & 0.21560*** & 0.23721*** \\
& (0.01567) & (0.05609) & (0.04689) & (0.07053) \\
\midrule
County FE & & Yes & Yes & Yes \\
Year FE & & Yes & Yes & Yes \\
Controls & & & Yes & Yes \\
State trends & & & & Yes \\
\bottomrule
\multicolumn{5}{l}{\footnotesize *** $p<0.01$, ** $p<0.05$, * $p<0.10$. SEs clustered by county.} \\
\multicolumn{5}{l}{\footnotesize HMS treatment: fraction of 30 pre-election days with any satellite-observed smoke.} \\
\end{tabular}
\end{table}

%----------------------------------------------------------------------
\clearpage
\section{Literature Review}
\label{sec:litreview}
%----------------------------------------------------------------------

This appendix surveys the literatures most relevant to this paper, organized into five strands: environmental shocks and voting, the theory of anti-incumbent and retrospective voting, air pollution and political behavior, wildfire smoke and economic/health outcomes, and identification strategies using atmospheric variation.

\subsection{Environmental Shocks, Natural Disasters, and Voting}

A large literature examines whether environmental shocks alter political behavior. The findings cluster around two mechanisms: retrospective incumbent punishment and climate salience.

\subsubsection{Incumbent Punishment}

\citet{healy2010myopic} provide the foundational analysis of myopic retrospective voting in response to disasters. Voters punish incumbents for disaster damage but reward them for disaster relief spending, and the asymmetry implies that governments underinvest in preparedness relative to relief. \citet{achen2016democracy} extend the argument, contending that voters engage in ``blind retrospection,'' punishing incumbents for events entirely beyond governmental control, including droughts, floods, and famously, shark attacks---though the shark attack finding is contested by \citet{fowler2018sharks}. \citet{gasper2011make} show that voters punish governors and presidents differently after natural disasters depending on whether a disaster declaration was issued, suggesting voters can partly distinguish between experiencing harm and receiving a government response. \citet{bechtel2011lasting} study the 2002 Elbe River flood in Germany and find that voters rewarded the incumbent government for effective flood management, demonstrating that the sign of the disaster--voting relationship depends on the government's response. \citet{cole2012incumbency} use rainfall variation in India to show that voters reward governments for disaster relief, providing a developing-country parallel.

\subsubsection{Wildfires and Voting}

\citet{hazlett2020wildfire} is the most directly relevant paper. Using proximity to California wildfire perimeters, they find that fire proximity increases pro-environment ballot proposition voting, but only in already-Democratic areas. Republican areas show no response. \citet{kronborg2024wildfires} study Sweden's 2018 wildfire season and find that affected municipalities showed higher Green Party vote share alongside incumbent punishment---the only other paper documenting wildfire-specific electoral effects, and one whose incumbent punishment finding directly parallels mine. \citet{andrews2025wildfire} show that wildfire experience in the American West increases climate change belief, particularly among Republicans, but does not increase willingness to spend on mitigation---a dissociation between salience and policy preference that maps onto my finding that turnout (engagement) is robust while vote share shifts (preference) are fragile. My paper extends this literature by studying smoke exposure rather than fire proximity, covering the entire continental United States, and documenting effects on turnout. \citet{liao2022extreme} study how extreme weather and natural disasters affect campaign contributions and House elections, finding that natural disasters hurt incumbents with anti-environment records, providing complementary evidence using campaign finance data.

\subsubsection{Climate Salience}

\citet{herrnstadt2014weather} use Google search intensity to show that unusual weather increases searches for ``climate change'' and that members of Congress vote more pro-environment when their home states experience unusual weather---the clearest evidence for a salience channel linking environmental experience to political behavior. \citet{hoffmann2022climate} use panel data across Europe to show that personal experience of climate-related events raises environmental concerns and increases Green party voting. \citet{baccini2021salience} study Swiss referendum voting after floods and find that flood exposure increases support for pro-climate ballot measures by up to 20 percent. \citet{arias2024hurricane} use Hurricane Ian (2022) to show that hurricane exposure increases belief in climate science, support for climate migrants, and salience of climate migration---effects that cross-cut partisanship but decay within months, consistent with the temporal dynamics in my results. \citet{hilbig2024floods} find that Germany's 2021 floods had little \textit{local} effect on Green Party support but produced a brief \textit{nationwide} increase, suggesting that environmental shocks may operate through diffuse media channels rather than geographically concentrated exposure. \citet{elliott2023climate} examine whether U.S.\ senators vote more pro-environment after climate-related natural disasters, finding effects that are short-lived (about two years).

\subsubsection{Weather and Turnout}

\citet{gomez2007weather} show that rain on election day suppresses voter turnout with differential effects by party. This is relevant for the turnout channel: my positive turnout coefficient could be in tension with their finding if smoke operated like bad weather, but smoke exposure in my specification is measured over the pre-election window rather than on election day itself, and the positive sign indicates mobilization rather than suppression. \citet{jusko2024motivated} provides the most direct parallel to my turnout finding: severe flooding in Slovakia increased voter turnout, with effects concentrated in the most damaged areas, consistent with disaster salience motivating participation rather than suppressing it.

\subsection{Anti-Incumbent Voting: Theoretical Foundations}

The robust anti-incumbent finding places this paper within one of the oldest debates in democratic theory.

\subsubsection{Sanctioning and Selection Models}

The formal literature on electoral accountability begins with \citet{ferejohn1986incumbent}, who models elections as a sanctioning mechanism: voters retain the incumbent if performance exceeds some threshold and replace her otherwise. The key insight is that retrospective voting can sustain accountability even when voters are poorly informed about policy. \citet{ashworth2012electoral} reviews how modern formal models blend sanctioning with selection (voters using past performance to infer incumbent quality), noting that the empirical literature has struggled to separate the two mechanisms.

\subsubsection{Blind Retrospection and Its Critics}

\citet{achen2016democracy} argue that retrospective voting is often ``blind''---voters punish incumbents for events beyond governmental control. \citet{gasper2011make} push back, showing voters distinguish between harm and government response. \citet{fowler2018sharks} challenge the shark attack result specifically. The emerging consensus, as articulated in \citet{healy2013retrospective}, is a middle ground: voters are neither fully rational accountability agents nor purely blind avengers, but decision makers who sometimes apply coherent logic and sometimes fall prey to psychological biases.

\subsubsection{Psychological Mechanisms}

Three mechanisms are especially relevant. First, negative affect: \citet{healy2010substituting} show that irrelevant events (college football outcomes) shift incumbent evaluations, suggesting negative emotional states reduce support for the status quo regardless of source. \citet{huber2012sources} provide experimental confirmation that voters overweight recent performance and are influenced by irrelevant lotteries. Second, attribution: partisanship powerfully shapes blame assignment after disasters, which may partly explain the partisan heterogeneity I find. Third, issue salience: negative experiences can shift the issues voters weight, activating voters who already care about climate while others simply experience negative affect. This dual-channel interpretation---salience for the pro-Democratic shift, affect for the anti-incumbent shift---maps onto my results where both channels appear to operate simultaneously through different pathways.

\subsection{Air Pollution and Political Behavior}

\citet{bellani2024air} is the closest analogue for the pollution--voting mechanism. Using 60 German federal and state elections, they exploit within-county variation in PM\textsubscript{10} on election day to show that a 10~$\mu$g/m$^3$ increase reduces the incumbent vote share by two percentage points. They argue this operates through a subconscious emotional channel: PM\textsubscript{10} fluctuations are imperceptible, yet they increase negative emotions. My setting differs critically in that wildfire smoke is visible and salient, potentially activating both their affect channel and a deliberate salience channel. \citet{kahn2007voting} provides background for the heterogeneity pattern: areas with stronger baseline pro-environment preferences show larger smoke effects.

\subsection{Wildfire Smoke: Economic, Health, and Behavioral Effects}

A rapidly growing literature documents the broad consequences of wildfire smoke, establishing the health and mood channels that could mediate voting effects.

\subsubsection{Labor Markets and Economic Costs}

\citet{borgschulte2022air} is the leading paper using wildfire smoke as an exogenous source of PM\textsubscript{2.5} variation to study economic outcomes, estimating that each additional smoke day reduces quarterly per capita earnings by about 0.1 percent. Their identification strategy---relying on the quasi-random spatial dispersion of smoke plumes using the same NOAA HMS data underlying \citet{childs2022daily}---is closely related to mine. Importantly, they show that controlling flexibly for wind direction does not change their estimates, strengthening the case that smoke rather than correlated wind patterns drives the variation.

\subsubsection{Mental Health and Mood}

\citet{burke2021exposures} document behavioral and sentiment responses to wildfire smoke: during smoke events, people search more for air quality information, stay home more, and express more negative sentiment. This provides direct evidence for the mood mechanism I invoke for incumbent punishment. \citet{du2024smoke} extend this evidence to a large-scale social media setting, showing that transboundary wildfire smoke in Southeast Asia significantly reduces expressed sentiment on Twitter, with stronger effects during peak fire season and when smoke originates from a neighboring country. \citet{jung2025mental} isolate short-term mental health effects of wildfire-specific PM\textsubscript{2.5}, finding that a 10~$\mu$g/m$^3$ increase significantly increases emergency department visits for depression and anxiety, with effects lasting up to seven days. \citet{miller2024wildfire} estimate that wildfire smoke accounts for 18 percent of ambient PM\textsubscript{2.5} concentrations and 0.42 percent of deaths among adults 65 and older, establishing the health harm of the exact pollution source I study.

\subsection{Identification: Wind Direction and Atmospheric Dispersion}

\citet{deryugina2019mortality} pioneered using wind direction as an instrument for PM\textsubscript{2.5}, estimating effects of acute exposure on elderly mortality. The innovation is that the approach does not require knowing the location of pollution sources---it simply exploits the fact that wind direction shifts nonlocal pollution in and out of a county. My approach differs in that rather than using wind as an instrument for total PM\textsubscript{2.5}, I use data from \citet{childs2022daily} that has already separated wildfire smoke PM\textsubscript{2.5} from other sources, and argue that atmospheric dispersion is quasi-random conditional on county and year fixed effects. \citet{rangel2019agricultural} use upwind/downwind variation from agricultural fires in Brazil to study birth outcomes, a conceptually very close design. \citet{borgschulte2022air} demonstrate as a robustness check that controlling for wind direction does not change their smoke estimates, a test that could be applied in my setting as well.

\end{document}
