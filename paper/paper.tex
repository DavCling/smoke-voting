\documentclass[12pt]{article}
\usepackage[margin=1in]{geometry}
\usepackage{booktabs}
\usepackage{graphicx}
\usepackage{natbib}
\usepackage{amsmath}
\usepackage{setspace}
\usepackage{caption}
\usepackage[hidelinks]{hyperref}
\graphicspath{{../output/figures/}}

\bibliographystyle{apalike}
\onehalfspacing

\title{Wildfire Smoke and Voting Behavior in the United States\thanks{Preliminary draft. Please do not cite or circulate without permission.}}
\author{David Clingingsmith\\Case Western Reserve University}
\date{\today}

\begin{document}
\maketitle

\begin{abstract}
\noindent Does wildfire smoke exposure shift political behavior? I exploit the quasi-random spatial dispersion of wildfire smoke plumes---driven by wind patterns rather than local conditions---to estimate the effect of smoke-derived PM\textsubscript{2.5} on county-level presidential and House election voting. Using daily county-level wildfire smoke PM\textsubscript{2.5} estimates \citep{childs2022daily} merged with election returns across multiple cycles (2008--2020 for presidential, 2016--2020 for House), I find that higher pre-election smoke exposure increases the Democratic two-party vote share and decreases the incumbent party's vote share. A 10~$\mu$g/m$^3$ increase in mean smoke PM\textsubscript{2.5} over the 60 days before the election is associated with a 0.9 percentage point increase in the Democratic vote share in presidential races. The incumbent punishment effect is roughly four times larger. Effects are present across the partisan spectrum but somewhat stronger in Democratic-leaning counties. These results extend findings on fire proximity \citep{hazlett2020wildfire} and general air pollution \citep{bellani2024air} to a nationally representative setting where treatment assignment is plausibly exogenous.

\medskip
\noindent \textit{JEL:} D72, Q54 \quad \textit{Keywords:} Wildfire smoke, voting behavior, air pollution, climate salience
\end{abstract}

%----------------------------------------------------------------------
\section{Introduction}
%----------------------------------------------------------------------

Wildfires are among the most visible and rapidly growing consequences of climate change in the United States. Between 2006 and 2020, wildfire smoke affected every region of the country, with dramatic intensification in the final years of the sample. Unlike ambient air pollution, wildfire smoke events are visible, sudden, and directly attributable to a specific cause---making them potentially more salient as signals of climate change. A growing literature investigates whether environmental shocks alter political behavior: \citet{hazlett2020wildfire} find that proximity to California wildfires increases pro-environment voting, but only in already-Democratic areas; \citet{bellani2024air} show that overall PM\textsubscript{10} pollution on election day shifts German voters against the incumbent; and \citet{gomez2007weather} demonstrate that rain suppresses voter turnout.

This paper bridges these strands by using \textit{wildfire-specific} smoke PM\textsubscript{2.5} as a treatment variable across the entire continental United States. Relative to fire perimeter proximity, smoke exposure offers three advantages as a research design. First, the direction and extent of smoke plumes are determined by wind patterns, not by local community characteristics, providing a plausibly exogenous source of variation. Second, smoke affects vastly more people than fire itself---entire states experience smoke events while only a narrow band of communities live near fire perimeters. Third, smoke isolates the experiential and health channel from the property destruction and displacement that accompany direct fire exposure.

%----------------------------------------------------------------------
\section{Data}
%----------------------------------------------------------------------

\paragraph{Wildfire smoke PM\textsubscript{2.5}.} I use daily county-level estimates of wildfire-attributed PM\textsubscript{2.5} from \citet{childs2022daily}, covering all U.S.\ counties from January 2006 through December 2020. These estimates use NOAA Hazard Mapping System satellite smoke plume classifications combined with machine learning to separate wildfire-derived PM\textsubscript{2.5} from background pollution. The data are available at Harvard Dataverse (doi:10.7910/DVN/DJVMTV).

\paragraph{Election returns.} County-level presidential election returns for 2000--2024 come from the MIT Election Data + Science Lab \citep{medsl2024county}. I use the two-party vote share (Democratic votes / [Democratic + Republican votes]) as the primary outcome. For House elections, I use precinct-level returns with county identifiers (2016, 2018, 2020) from the same source, aggregating precinct votes to the county level to enable analysis at the same geographic unit as the presidential regressions.

\paragraph{Analysis samples.} The overlap of smoke data (2006--2020) and presidential elections yields four election cycles: 2008, 2012, 2016, and 2020. After merging on county FIPS codes, the presidential analysis sample contains 12,429 county-election observations spanning 3,108 counties. The county-level House sample covers three election cycles (2016, 2018, 2020) with 9,171 county-election observations.

\paragraph{Smoke exposure measures.} For each county and election, I aggregate daily smoke PM\textsubscript{2.5} over pre-election windows: 7, 30, 60, and 90 days before election day, plus the full fire season (June 1 to election day). The primary treatment variable is the mean daily smoke PM\textsubscript{2.5} in the 60 days before the election.

%----------------------------------------------------------------------
\section{Empirical Strategy}
%----------------------------------------------------------------------

I estimate two-way fixed effects models of the form:
\begin{equation}
Y_{ct} = \alpha_c + \gamma_t + \beta \cdot \text{SmokePM}_{ct} + \varepsilon_{ct}
\label{eq:twfe}
\end{equation}
where $Y_{ct}$ is the outcome in county $c$ in election year $t$, $\alpha_c$ are county fixed effects absorbing all time-invariant county characteristics, $\gamma_t$ are election-year fixed effects absorbing national swings, and $\text{SmokePM}_{ct}$ is the mean wildfire smoke PM\textsubscript{2.5} in the pre-election window. Standard errors are clustered by county.

\paragraph{Identifying assumption.} The key assumption is that, conditional on county and year fixed effects, variation in wildfire smoke exposure is uncorrelated with unobserved determinants of voting. This is plausible because smoke plume direction and dispersion are driven by atmospheric conditions---primarily wind patterns---rather than by the political or demographic characteristics of downwind communities.

\paragraph{Threats to identification.} Two potential concerns merit discussion. First, spatially correlated shocks such as drought could affect both fire activity and local economic conditions. This is mitigated by the fact that smoke travels hundreds of miles from fire origins, so downwind counties experience smoke without experiencing the local conditions that generated the fires. Second, secular trends in fire-prone versus non-fire-prone regions could confound the estimates; county fixed effects absorb level differences, and year fixed effects absorb national trends, but region-specific trends remain a potential concern.

\paragraph{Continuous treatment and TWFE.} Recent work by \citet{callaway2024continuous} shows that TWFE regressions with a continuous treatment variable can produce coefficients with ambiguous causal interpretation due to heterogeneous weighting across dose levels. In our setting, several features mitigate these concerns: treatment is atmospherically assigned (limiting selection into dose levels), and we estimate a linear slope corresponding to the average causal response (ACRT) decomposition in which weights are non-negative. As a robustness check, I verify that results are qualitatively similar when the treatment is dichotomized or discretized into dose bins.

%----------------------------------------------------------------------
\section{Results}
%----------------------------------------------------------------------

\subsection{Main Results}

Table~\ref{tab:main} presents the main presidential election estimates. Column (1) shows that a 1~$\mu$g/m$^3$ increase in mean smoke PM\textsubscript{2.5} over the 60 days before the election increases the Democratic two-party vote share by 0.087 percentage points ($p < 0.001$). At the sample mean of 2.7~$\mu$g/m$^3$, this implies that moving from zero smoke to mean exposure shifts the Democratic vote share by roughly 0.24 percentage points. A county experiencing the 2020 Western fire season levels of smoke ($\sim$40~$\mu$g/m$^3$) would see a shift of approximately 3.5 percentage points.

Column (2) shows that the incumbent punishment effect is substantially larger: a 1~$\mu$g/m$^3$ increase reduces the incumbent party's vote share by 0.40 percentage points. This is consistent with the negative-affect mechanism identified by \citet{bellani2024air} in the German context---smoke makes voters feel worse, and they punish the party in power regardless of its partisan identity.

\begin{table}[htbp]
\centering
\caption{Effect of Wildfire Smoke on Presidential Voting Outcomes}
\label{tab:main}
\small
\begin{tabular}{lccc}
\toprule
& (1) & (2) & (3) \\
& DEM Vote Share & Incumbent Vote Share & Log Total Votes \\
\midrule
Mean Smoke PM\textsubscript{2.5} (60d) & 0.00087*** & $-$0.00399*** & 0.00242*** \\
& (0.00009) & (0.00044) & (0.00018) \\[6pt]
County FE & Yes & Yes & Yes \\
Year FE & Yes & Yes & Yes \\
Observations & 12,429 & 12,429 & 12,429 \\
$R^2$ (within) & $-$0.007 & $-$0.024 & 0.070 \\
\bottomrule
\multicolumn{4}{l}{\footnotesize *** $p<0.01$, ** $p<0.05$, * $p<0.10$. Standard errors clustered by county.} \\
\end{tabular}
\end{table}

\subsection{Heterogeneity by Prior Partisanship}

Table~\ref{tab:hetero} splits the sample by terciles of lagged Democratic vote share. The pro-Democratic shift from smoke is present in all three groups, with a somewhat larger effect in D-leaning counties (0.082 pp) than in R-leaning counties (0.066 pp). Unlike \citet{hazlett2020wildfire}, who find effects \textit{only} in Democratic areas for fire proximity, I find that smoke exposure moves all county types toward the Democrats, though the effect is modestly larger where pro-environment attitudes are presumably more prevalent.

\begin{table}[htbp]
\centering
\caption{Heterogeneity by Prior Partisanship}
\label{tab:hetero}
\small
\begin{tabular}{lccc}
\toprule
& R-Leaning & Swing & D-Leaning \\
\midrule
Mean Smoke PM\textsubscript{2.5} (60d) & 0.00066*** & 0.00049*** & 0.00082*** \\
& (0.00021) & (0.00014) & (0.00013) \\[6pt]
Observations & 4,144 & 4,141 & 4,143 \\
\bottomrule
\multicolumn{4}{l}{\footnotesize *** $p<0.01$. County and year FE. SEs clustered by county.} \\
\end{tabular}
\end{table}

\subsection{Temporal Dynamics}

Figure~\ref{fig:windows} plots the estimated effect of mean smoke PM\textsubscript{2.5} on Democratic vote share across different pre-election windows. The effect is statistically significant at all windows, with the largest point estimate at the 30-day window. This suggests that smoke exposure in the weeks most proximate to the election has the greatest electoral impact, consistent with a salience or recency mechanism.

\begin{figure}[htbp]
\centering
\includegraphics[width=0.75\textwidth]{event_study_windows.png}
\caption{Effect of smoke PM\textsubscript{2.5} on Democratic vote share by pre-election window length. Points are coefficient estimates from separate TWFE regressions; bars are 95\% confidence intervals.}
\label{fig:windows}
\end{figure}

\subsection{Geographic Variation in Smoke Exposure}

Figure~\ref{fig:maps} displays county-level mean smoke PM\textsubscript{2.5} in the 30 days before each election. The maps illustrate both the geographic scope and temporal variation that identify the main estimates: 2016 saw minimal pre-election smoke nationwide, while 2020 produced extreme exposure across the Western states following the historic August--September fire season.

\begin{figure}[htbp]
\centering
\includegraphics[width=\textwidth]{smoke_exposure_map_panel.png}
\caption{Pre-election wildfire smoke exposure by county, 30-day window before election day. Color scale is identical across all panels.}
\label{fig:maps}
\end{figure}

\subsection{House Elections}

To test whether the effects extend beyond presidential races, I aggregate MEDSL precinct-level House returns (which include county FIPS identifiers) directly to the county level for 2016--2020, avoiding the measurement error that would be introduced by a county-to-congressional-district crosswalk.

Table~\ref{tab:house} presents the county-level House results alongside the presidential estimates, all using the 60-day mean smoke PM\textsubscript{2.5} treatment. The county-level House analysis covers approximately 3,000 counties per election across three cycles (2016, 2018, 2020). Multi-district counties have votes from all House races aggregated, measuring overall House candidate performance in each county rather than individual district outcomes. The pro-Democratic and anti-incumbent effects are both statistically significant, though smaller in magnitude than the presidential estimates, consistent with the more candidate-driven nature of House races.

\begin{table}[htbp]
\centering
\caption{Effect of Wildfire Smoke: County-Level House vs.\ Presidential}
\label{tab:house}
\small
\begin{tabular}{lcc}
\toprule
& (1) & (2) \\
& County House & Presidential \\
\midrule
\multicolumn{3}{l}{\textit{Panel A: DEM Vote Share}} \\[3pt]
Mean Smoke PM\textsubscript{2.5} (60d) & 0.00038*** & 0.00087*** \\
& (0.00013) & (0.00009) \\[6pt]
\multicolumn{3}{l}{\textit{Panel B: Incumbent Vote Share}} \\[3pt]
Mean Smoke PM\textsubscript{2.5} (60d) & $-$0.00153*** & $-$0.00399*** \\
& (0.00045) & (0.00044) \\[6pt]
\multicolumn{3}{l}{\textit{Panel C: Log Total Votes}} \\[3pt]
Mean Smoke PM\textsubscript{2.5} (60d) & 0.00177*** & 0.00242*** \\
& (0.00066) & (0.00018) \\[6pt]
\midrule
Unit & County & County \\
FE & County + Year & County + Year \\
Observations & 8,391 / 9,165 & 12,429 \\
Elections & 2016--2020 & 2008--2020 \\
\bottomrule
\multicolumn{3}{l}{\footnotesize *** $p<0.01$, ** $p<0.05$, * $p<0.10$. SEs clustered by county.} \\
\multicolumn{3}{l}{\footnotesize Panels A--B use contested races only; Panel C includes all.} \\
\end{tabular}
\end{table}

As a further robustness check, I also estimate the House specifications at the congressional district level using Census county-to-district crosswalks (Appendix Table~\ref{tab:district_house}). The district-level estimates are noisier due to the measurement error introduced by the crosswalk, but the anti-incumbent effect remains statistically significant.

%----------------------------------------------------------------------
\section{Discussion}
%----------------------------------------------------------------------

Three mechanisms could drive these results. First, a \textit{salience} channel: smoke makes climate change tangible, increasing the weight voters place on environmental issues and benefiting the party perceived as more pro-environment \citep{hazlett2020wildfire, kahn2007voting}. Second, a \textit{negative affect} channel: smoke degrades well-being and mood, and voters punish incumbents for experienced discomfort regardless of policy responsibility \citep{bellani2024air, healy2010myopic}. Third, a \textit{disruption} channel: smoke could differentially suppress turnout among certain voter groups \citep{gomez2007weather, burke2021exposures}.

The data are more consistent with the first two mechanisms than the third. The pro-Democratic shift points toward salience, while the larger anti-incumbent effect points toward negative affect. The positive turnout coefficient (Column 3 of Table~\ref{tab:main}) is surprising and may reflect confounding from 2020's historically high turnout; this result should be interpreted cautiously.

\paragraph{Limitations.} This proof of concept has several limitations that subsequent work should address. The analysis covers only four presidential elections and three House elections. The turnout measure (log total votes) is a crude proxy without a proper population denominator. County-level aggregation may mask within-county heterogeneity. And the negative within-$R^2$ values in some specifications suggest that the smoke variable alone explains limited within-county variation after absorbing fixed effects, underscoring that these are small effects on a noisy outcome.

%----------------------------------------------------------------------
\section{Conclusion}
%----------------------------------------------------------------------

Wildfire smoke exposure shifts votes toward the Democratic Party and against the incumbent in both presidential and House elections, with effects that are statistically significant, present across the partisan spectrum, and strongest in the weeks immediately preceding the election. These preliminary results suggest that wildfire smoke---which is plausibly exogenous and affects a far larger population than fire proximity---offers a compelling research design for studying how environmental experience shapes political behavior.

\bibliography{references}

%----------------------------------------------------------------------
\appendix
\section*{Appendix}
\setcounter{table}{0}
\renewcommand{\thetable}{A\arabic{table}}

\begin{table}[htbp]
\centering
\caption{Robustness: District-Level House Estimates}
\label{tab:district_house}
\small
\begin{tabular}{lcc}
\toprule
& (1) & (2) \\
& District House & County House \\
\midrule
\multicolumn{3}{l}{\textit{Panel A: DEM Vote Share}} \\[3pt]
Mean Smoke PM\textsubscript{2.5} (60d) & $-$0.00027 & 0.00038*** \\
& (0.00045) & (0.00013) \\[6pt]
\multicolumn{3}{l}{\textit{Panel B: Incumbent Vote Share}} \\[3pt]
Mean Smoke PM\textsubscript{2.5} (60d) & $-$0.00186** & $-$0.00153*** \\
& (0.00089) & (0.00045) \\[6pt]
\multicolumn{3}{l}{\textit{Panel C: Log Total Votes}} \\[3pt]
Mean Smoke PM\textsubscript{2.5} (60d) & $-$0.00223 & 0.00177*** \\
& (0.00150) & (0.00066) \\[6pt]
\midrule
Unit & District & County \\
FE & District + Year & County + Year \\
Observations & 3,014 / 3,450 & 8,391 / 9,165 \\
Elections & 2006--2020 & 2016--2020 \\
\bottomrule
\multicolumn{3}{l}{\footnotesize *** $p<0.01$, ** $p<0.05$, * $p<0.10$. SEs clustered by unit.} \\
\multicolumn{3}{l}{\footnotesize District-level uses Census crosswalk to map county smoke to districts.} \\
\end{tabular}
\end{table}

\end{document}
